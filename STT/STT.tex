\documentclass[a4paper,NoNotes,GeneralMath]{stdmdoc}

\newcommand{\sle}{\sqsubseteq}
\newcommand{\dom}{\mbox{dom}}
\newcommand{\leftcurly}[1]{{\left\{{#1}\right.}}
\newcommand{\fix}{\mbox{fix}}

\begin{document}
\title{Semantica e Teoria dei Tipi}

\section*{Teoria dei Domini}
Per introdurre la semantica operazionale abbiamo bisogno di un po' di
questa teoria.

\subsection*{Definizioni di domini e cpo}
\begin{itemize}
\item ({\bf Poset}) Dato un insieme $D$, una relazione $\sle$ su di esso si dice un ordine parziale se valgono:
  \begin{enumerate}
  \item ({\bf Riflessiva}) $\forall d \in D \quad d \sle d$
  \item ({\bf Transitiva}) $\forall d, e, f \in D \quad d \sle e \sle f \implies d \sle f$
  \item ({\bf Antisimmetrica}) $\forall d, e \in D \quad d \sle e, e \sle d \implies d = e$
  \end{enumerate}
\item ({\bf Minimo di un sottoinsieme}) Dato un sottoinsieme $S \subseteq D$ di un poset, si dice che $s \in S$ è un elemento minimo se $\forall d \in D \quad s \sle d$.
  Un elemento minimo, se esiste, è unico per l'assioma di antisimmetria, e lo denotiamo con il simbolo $\bot$
\item ({\bf Estremo superiore di una catena}) Una catena numerabile crescente (CIC - Countable Increasing Chain) è una sequenza $\{ d_n \}_{n \in \bbN}$ di elementi di un poset $D$ tale che
  $$ d_0 \sle d_1 \sle d_2 \sle \ldots $$
  Un maggiorante per la catena è un qualunque $d \in D$ tale che $\forall n \in \bbN \quad d_n \sle d$.
  
  Se esiste il minimo degli elementi maggioranti, esso viene chiamato sup, e verrà indicato con $\sqcup_{n \ge 0} d_n$, o anche con $\sup_n d_n$.
  Per definizione vale quindi che un elemento $d \in D$ è il sup della catena se:
  \begin{enumerate}
  \item $\forall n \in \bbN \quad d_n \sle \sup_n d_n$
  \item $\forall d \in D \quad (\forall m \in \bbN \quad d_m \sle d) \implies \sup_n d_n \sle d$
  \end{enumerate}
  
  Notiamo inoltre che togliendo un numero finito di elementi alla catena, non alteriamo l'insieme dei suoi maggioranti, e quindi non cambia nemmeno il suo sup.
\item ({\bf cpo - Chain Complete Poset}) Un cpo è un poset $D$ in cui tutte le CIC hanno un sup.
\item ({\bf Dominio}) Un dominio è un cpo con un elemento minimo $\bot$
\item ({\bf Dominio delle funzioni parziali}) L'insieme delle funzioni parziali $X \rar Y$ può essere dotato della struttura di dominio:
  \begin{enumerate}
  \item $f \sle g$ sse $\dom f \subseteq \dom g$ e vale $\forall x \in \dom f \quad f(x) = g(x)$
  \item Data una CIC $f_0 \sle f_1 \sle \ldots$ il suo sup è la funzione $f$ con $\dom f = \cup_{n \ge 0} \dom(f_n)$ definita da:
    $$f(x) = \leftcurly{\begin{array}{cc}
                          f_n(x) & \mbox{se } x \in \dom(f_n) \mbox{ per qualche } n \\
                          \mbox{undefined} & \mbox{otherwise} \\
                        \end{array}{cc}}$$
  \item $\bot$ è la funzione totalmente non definita, ovvero $\dom f = \emptyset$
  \end{enumerate}
\end{itemize}

\subsection*{Funzioni continue e monotone}
\begin{itemize}
\item ({\bf Funzioni monotone}) Una funzione $f: D \rar E$ tra poset è monotona sse $\forall x, y \in D \quad x \sle y \implies f(x) \sle f(y)$
\item ({\bf Funzioni continue}) Se entrambi i poset sono cpo ha senso parlare di funzioni continue: $f: D \rar E$ si dice continua sse è monotona e preserva i sup delle CIC, ovvero se $\forall \mbox{catena} d_0 \sle d_1 \sle \ldots$ in $D$ si ha che $f(\sup_n d_n) = sup_n f(d_n)$ dentro ad $E$.
\item ({\bf Funzioni rigide}) Se $D$ ed $E$ hanno un minimo, si dice che $f$ è rigida se $f(\bot_D) = \bot_E$
\item ({\bf Criterio per la continuità}) Notiamo che, data una cic in $D$: $d_0 \sle d_1 \sle \ldots$ ed $f$ monotona, si ha sempre che $f(d_0) \sle f(d_1) \sle \ldots$, ovvero $f(d_n)$ è una catena in $E$. Inoltre se $d$ è un maggiorante della catena in $D$ deve essere che $f(d)$ è un maggiorante della catena in $E$: ovvero se $f$ è monotona tra cpo si ha sempre che:
  $$\sup_n f(d_n) \sle f(\sup_n d_n)$$
  Quindi per controllare se una funzione è continua tra cpo basterà controllare solo l'inclusione opposta.
\item ({\bf Funzioni costanti}) Dati due cpo $D$ ed $E$ ed $e \in E$, la funzione costante $c_e: D \rar E$ data da $c_e(d) = e$ $\forall d \in D$ è continua
\end{itemize}

\subsection*{Teorema del punto fisso di Tarski}
\begin{itemize}
\item ({\bf Punto fisso}) Un punto fisso per una funzione $f: D \rar D$ è un elemento $d \in D$ tale che $f(d) = d$
\item ({\bf Punto prefisso}) Se $D$ è un poset, un punto $d$ si dice prefisso di $f$ se $f(d) \sle d$
  Il punto prefisso più piccolo (pppp) di $f$, se esiste, verrà scritto $\fix f$. Esso è quindi univocamente specificato dalle due proprietà:
  \begin{enumerate}
  \item $f(\fix f) \sle \fix f$
  \item $\forall d \in D \quad f(d) \sle d \implies \fix f \sle d$
  \end{enumerate}

  Il motivo per cui ci interessano è che, se $f$ è monotona, ed ha un pppp $\fix f$ allora $\fix f$ è in particolare anche un punto fisso.
\item ({\bf Punto fisso di Tarski}) Sia $f: D \rar D$ una funzione continua su un dominio $D$. Allora $f$ ha un pppp, dato da $\fix f = \sup_n f^n(\bot)$. Inoltre $\fix f$ è un punto fisso di $f$, ed è quindi il più piccolo punto fisso di $f$.
\end{itemize}

\subsection*{Costruzioni sui Domini}
\begin{itemize}
\item ({\bf Prodotto di domini}) Dati due cpo $D_1$ e $D_2$, definiamo $D_1 \times D_2$ il cpo che ha come insieme il prodotto cartesiano degli insiemi, con la relazione d'ordine $(d_1, d_2) \sle (d_1', d_2') \sse d_1 \sle_1 d_1', d_2 \sle_2 d_2'$.

  Il sup di una catena si può vedere essere fatto componente per componente: $sup_n (d_{1,n}, d_{2,n}) = (sup_i d_{1,i}, sup_j d_{1,j})$.

  Inoltre se $D_1$ e $D_2$ sono domini, lo è anche $D_1 \times D_2$ con $\bot_{D_1 \times D_2} = (\bot_{D_1}, \bot_{D_2})$
\item ({\bf Proiezioni}) Dati $D$ ed $E$ cpo, la proiezione $\pi: D \times E \rar E$ definita da $\pi(d, e) = e$ è una funzione continua.
\item ({\bf Pairing}) Dati $C, D, E$ cpo e funzioni $f: C \rar D$ e $g: C \rar E$ continue, allora si ha che $\langle f, g \rangle : C \rar D \times E$ definita da $\langle f, g \rangle (c) = (f(c), g(c))$ è una funzione continua.
\item ({\bf Prodotto arbitrario}) Siano $(D_i, \sle_i)_{i \in I}$ una famiglia di cpo. Il prodotto di questa famiglia ha come insieme il prodotto cartesiano degli insiemi $\prod_i D_i = \{ f: I \rar D\}$, e come ordine parziale si ha $p \sle q \sse \forall i \in I \quad p(i) \sle_i q(i)$.
\end{itemize}

\end{document}

