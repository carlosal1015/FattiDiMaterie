\documentclass[a4paper,NoNotes,GeneralMath]{stdmdoc}
\usepackage{pgf}
\usepackage{tikz}
\usetikzlibrary{arrows,automata}
\newcommand{\MCD}{\text{MCD }}
\newcommand{\Lt}{\text{lt }}
\newcommand{\Lc}{\text{lc }}
\newcommand{\BdG}{\text{BdG }}
\newcommand{\Ris}{\text{Ris }}
\newcommand{\srar}{\twoheadrightarrow}
\newcommand{\hrar}{\hookrightarrow}
\newcommand{\Hom}{\text{Hom }}
\newcommand{\coKer}{\text{coKer }}
\newcommand{\gen}[1]{\langle {#1} \rangle}

\begin{document}
	\title{Algebra 2}
	
	\section*{Anelli}
	\begin{itemize}
		\item Se $A$ è un anello finito allora $A = A^* \sqcup \cD(A)$
		\item $f: A \rar B$ allora $\Img f \cong \frac{A}{\Ker f}$
		\item $I \subseteq A$ ideale, $B \subseteq A$ sottoanello allora vale $\frac{I+B}{I} \cong \frac{B}{I\cap B}$
		\item $I, J \subseteq A$ ideali e $I \subseteq J$. Allora vale $\frac{\frac{A}{I}}{\frac{J}{I}} \cong \frac{A}{J}$ \\
			Si ha inoltre la corrispondenza tra gli ideali di $\frac{A}{I}$ e gli ideali $J \subseteq A$ tali che $I \subseteq J$. In questa corrispondenza i primi ed i massimali si corrispondono
		\item $IJ \subseteq I \cap J$. Se vale $I + J = 1$ allora $IJ = I \cap J$
		\item È FALSO che $I \cap (J + K) = (I \cap J) + (I \cap K)$. FALSO
		\item $I \subseteq \sqrt{I}$
		\item ($A$ dominio) $a$ primo $\implies$ $a$ irriducibile
		\item ($A$ UFD) $a$ irriducibile $\implies$ $a$ primo
		\item Se $H \subseteq A \times B$ è ideale allora $H = I \times J$ con $I \subseteq A$, $J \subseteq B$ ideali
		\item $A \cong A_1 \times A_2 \sse \exists e \in A, e \neq 0,1 \quad e^2 = e$
		\item $\cD(A) = \cup_{a \notin A^*} (0 : a) = \cup_{a \notin A^*} \sqrt{(0 : a)}$ e $\sqrt{\cD(A)} = \cD(A)$, anche se non è necessariamente un ideale
		\item $\{ E_\lambda \}_{\lambda \in \Lambda}$ sottoinsiemi di $A$. Allora $\cup_{\lambda \in \Lambda} \sqrt{E_\lambda} = \sqrt{\cup_{\lambda \in \Lambda} E_\lambda}$
		\item Sia $A$ dominio con un numero infinito di elementi e $\mid A^* \mid < \infty$ allora $A$ possiede infiniti ideali massimali
		\item $I$ massimale $\implies I$ primo $\implies I$ primario. Inoltre $A$ dominio $\sse (0)$ ideale primo
		\item Sono equivalenti:
			\begin{itemize}
				\item $A$ ha un unico ideale massimale
				\item $\exists \mathfrak{m} \subseteq A$ ideale massimale $\tc \forall a \in A \setminus \mathfrak{m} \implies a \notin A^*$
				\item $\exists \km \subseteq A$ ideale massimale $\tc$ ogni elemento della forma $1 + \km$ è invertibile
			\end{itemize}
		\item $a \in \cJ(A) \sse \forall b \in A \quad 1-ab \in A^*$
		\item $\sqrt{I} = \cap_{I \subseteq P \text{ primi }} P$
		\item ({\bf Lemma di Scansamento}) $P_1, \ldots, P_n$ ideali primi. Sia $I \subseteq A$ ideale $\tc I \subseteq \cup_{i=1}^n P_i$. Allora $\exists j \tc I \subseteq P_j$
		\item $I_1, \ldots, I_n$ ideali e $P$ ideale primo. $\cap_{i=1}^n I_i \subseteq P \implies \exists j \tc I_j \subset P$. Inoltre se $P = \cap_i I_i$ allora $\exists j \tc I_j = P$
		\item ({\bf Teorema cinese}) Siano $I_1, \ldots, I_n \subseteq A$ ideali tali che $I_i + I_j = 1$. Allora $\forall a_1, \ldots, a_n \in A \quad \exists a \in A \tc a \equiv a_i (I_i)$
		\item $A$ anello c.u. Allora si ha che
			\begin{itemize}
				\item $f \in A[x]$ è un'unità $\sse$ $f = \sum_{i=0}^n a_i x^i$ con $a_i \in A$ tali che $a_0 \in A^*$ e $a_i \in \cN(A) \quad \forall i \ge 1$
				\item $f \in A[x]$ è nilpotente $\sse$ $\forall i \quad a_i \in \cN(A)$
				\item $f \in A[x]$ è divisore di zero $\sse$ $\exists c \in A, c \neq 0 \tc cf = 0$
			\end{itemize}
			Si ha inoltre per gli anelli di polinomi che
			\begin{itemize}
				\item $I$ primo $\sse I[x]$ primo
				\item $I$ primario $\sse I[x]$ primario
			\end{itemize}
			NON è vero che tutti gli ideali di $A[x]$ sono del tipo $I[x]$, come ad esempio $(x)$
		\item Gli ideali primi di $\bbZ[x]$ sono dei seguenti tipi:
			\begin{itemize}
				\item $(0)$
				\item $(p)[x]$ con $p \in \bbP$
				\item $(f(x))$ con $f$ irriducibile
				\item $(p, f(x))$ con $p \in \bbP$ e $f$ irriducibile modulo $p$ (Questi sono anche massimali)
			\end{itemize}
		\item $u \in A^*$, $a \in \cN(A)$, allora $u + a \in A^*$ (Somma di un nilpotente e di un invertibile è invertibile)
		\item $I$ primo $\implies I $ irriducibile
		\item In $A[x]$ si ha $\cN(A[x]) = \cJ(A[x])$ (Mentre in generale vale solo che $\cN(A) \subseteq \cJ(A)$)
		\item Sia $\phi: A \rar B$ omomorfismo di anelli. Allora
			\begin{itemize}
				\item $\phi(\cN(A)) \subseteq \cN(B)$
				\item Se $\phi$ è surgettivo allora $\phi(\cJ(A)) \subseteq \cJ(B)$
				\item $A$ semilocale (con un numero finito di ideali massimali) $\implies \phi(\cJ(A)) = \cJ(B)$
			\end{itemize}
		\item $A$ PID $\implies \cJ(A) = \cN(A)$
		\item $A \tc$ ogni ideale è primo $\implies$ $A$ è un campo
		\item $A \tc$ ogni ideale primo è principale $\implies A$ è un anello ad ideali principali
		\item $\sqrt{I}$ massimale $\implies I$ primario.
		\item $I$ primario, $J \not\subseteq \sqrt{I} \implies \sqrt{I : J^i} = \sqrt{I} \forall i$
		\item $I = \sqrt{I}$ e $h \notin I \implies I:h$ è radicale
		\item ({\bf Teorema della base di Hilbert}) Se $A$ è un anello Nötheriano, allora $A[x]$ è Nötheriano
		\item Se $A$ è locale, $\km$ il suo ideale massimale e $Q$ è $\km$-primario, allora si ha $(\frac{A}{Q})_{\frac{\km}{Q}} \cong \frac{A}{Q}$
	\end{itemize}
	
	\section*{Basi di Gröbner}
	\subsection*{Ideali Monomiali}
	Un ideale monomiale in $K[x_1, \ldots, x_n]$ è un ideale generato dai monomi
	\begin{itemize}
		\item ({\bf Criterio di appartenenza}) Sia $I$ un ideale monomiale e $f \in K[x_1, \ldots, x_n]$, $f = \sum_\beta c_\beta x^\beta$ con $c_\beta \in K$. Allora $f \in K \sse \forall \beta x^\beta \in I$
		\item ({\bf Lemma di Dickson}) Ogni ideale monomiale è finitamente generato. (La frontiera minimale di un ideale monomiale è unica, e viene detta Escalièr)
		\item ({\bf Operazioni con ideali monomiali}) Siano $I_1 = (m_1, \ldots, m_k)$ e $I_2 = (n_1, \ldots, n_s)$ con $m_i, n_j$ monomi. Allora si ha
			\begin{itemize}
				\item $I_1 + I_2 = (m_1, \ldots, m_k, n_1, \ldots, n_s)$
				\item $I_1 \cap I_2 = (\MCD_{i,j} (m_i, n_j))$
				\item $I_1 \cdot I_2 = (m_i \cdot n_j)_{i,j}$
				\item ({\bf Iatto}) $(I, m \cdot n) = (I, m) \cap (I, n)$ se $\MCD(m, n) = 1$ come monomi
				\item $I$ primo $\sse I = (x_{i_1}, \ldots, x_{i_k})$ (ed è massimale solo se le variabili compaiono tutte, ma DEVE essere monomiale)
				\item $I = \sqrt{I}$ (ovvero $I$ è radicale) $\sse \sqrt{m_i} = m_i \forall i$
				\item $I$ è primario $\sse I = (x_{i_1}^{\alpha_1}, \ldots, x_{i_k}^{\alpha_k}, m_1, \ldots, m_s)$ dove $m_1, \ldots, m_s \in K[x_{i_1}, \ldots, x_{i_k}]$
				\item $I$ è irriducibile $\sse$ $I = (x_{i_1}^{\alpha_1}, \ldots, x_{i_k}^{\alpha_k})$
				\item $I\cdot J = I \cap J \sse \forall i,j \quad \MCD(m_i, n_j) = 1$
				\item $I : J = \cap_i (I : n_i)$ e $I : (n_i) = (\frac{m_j}{\MCD(n_i, m_j)})_{j}$
			\end{itemize}
		\item Notare che usando la terza relazione del punto precedente possiamo spezzare ogni ideale monomiale in ideali primari e utilizzando $\sqrt{I \cap J} = \sqrt{I} \cap \sqrt{J}$ si possono calcolare anche gli ideali primi associati. \\
			Inoltre con la decomposizione in primari si calcolano bene i divisori di zero, i nilpotenti, etc.
	\end{itemize}
	
	\subsection*{Ordinamenti Monomiali Comuni}
	\begin{itemize}
		\item LEX $x_1 > x_2 > \ldots > x_n$. Dico che $\alpha \ge \beta \sse$ In $\alpha - \beta$ la prima coordinata $\neq 0$ è positiva
		\item DEGLEX Sia $\mid \alpha \mid := \sum_i \alpha_i$. Allora $\alpha \ge \beta \sse$ si ha $\mid \alpha \mid > \mid \beta \mid$ oppure $\mid \alpha \mid = \mid \beta \mid$ e vale $\alpha \ge \beta$ con LEX
		\item DEGREVLEX $\alpha \ge \beta \sse \mid \alpha \mid > \mid \beta \mid$ oppure si ha $\mid \alpha \mid = \mid \beta \mid$ e in $\alpha - \beta$ l'ultima coordinata $\neq 0$ è negativa
	\end{itemize}

	\subsection*{Basi di Gröbner e Algoritmo di Divisione}
	\begin{itemize}
		\item ({\bf Algoritmo di Divisione}) Siano $f_1, \ldots, f_k, f \in K[x_1, \ldots, x_n]$ allora $\exists a_1, \ldots, x_k, r \in K[x_1, \ldots, x_n]$ tali che $f = \sum_i a_i f_i + r$ e $\Deg (a_i f_i) \le \Deg(f)$. Inoltre se $r = \sum_\alpha r_\alpha x^\alpha$ si ha che se $r_\alpha \neq 0$ allora $x^\alpha \in (\Lt(f_1), \ldots, \Lt(f_k))$ \\
		Notiamo che posso fare dei passaggi "a mano" prima di partire con l'algoritmo di divisione e lui funzionerà comunque. La cosa importante è ricordarsi di soddisfare la condizione $\Deg (a_i f_i) \le \Deg(f)$ ad ogni passaggio.
		\item ({\bf Base di Gröbner}) Un insieme di polinomi $g_1, \ldots, g_k$ generatori di un ideale $I$ i cui leading term generano $\Lt(I)$ si dicono base di Gröbner. Sono equivalenti inoltre:
			\begin{itemize}
				\item $\forall f \quad \exists ! r$ resto della divisione di $f$ per $\{g_1, \ldots, g_k\}$
				\item $\forall f \in I = (g_1, \ldots, g_k)$ si ha $r = 0$ dall'algoritmo di divisione
				\item $\forall i,j \quad S(g_i, g_j)$ ha resto $r = 0$ nell'algoritmo di divisione
			\end{itemize}
		Dove per divisione si intende un risultato che soddisfi le ipotesi dell'algoritmo di divisione
		\item ({\bf Base di Gröbner ridotta}) Una BdG $G = \{ g_1, \ldots, g_k \}$ si dice ridotta se è minimale per inclusione e inoltre
			\begin{itemize}
				\item $\Lc(g_i) = 1 \quad \forall i$
				\item $(\Deg(g_1), \ldots, \Deg(g_k))$ sono un'escalièr per $\Deg(I)$
				\item $\forall g_i \quad g_i = \sum_\alpha c_\alpha x^\alpha$ allora $x^\alpha \notin \Lt(G \setminus \{g_i\})$
			\end{itemize}
		Teorema: La base ridotta è unica. Per ridurre una BdG basta prendere ciascun elemento $g$ ed effettuare la divisione per $G \setminus \{g\}$
		\item ({\bf S-polinomio}) Dati $f, g \in K[x_1, \ldots, x_n]$ e supponiamo $f = c_\alpha x^\alpha + f_1$ e $g = d_\beta x^\beta + g_1$ con $\Deg f = \alpha, \Deg g = \beta$. Allora dico S-polinomio tra $f, g$ il polinomio definito da $\gamma = (\gamma_1, \ldots, \gamma_n)$ con $\gamma_i = \max(\alpha_i, \beta_i)$
			$$ S(f, g) = \frac{x^\gamma}{c_\alpha x^\alpha} f - \frac{x^\gamma}{d_\beta x^\beta} g $$
	\end{itemize}
	
	\subsection*{Applicazioni e Computazioni}
	\begin{itemize}
		\item ({\bf Eliminazione di LEX}) $I \subseteq K[x_1, \ldots, x_n]$ allora $I_k = I \cap K[x_{k+1}, \ldots, x_n]$ è il $k$-esimo ideale di eliminazione. Vale il teorema: Se $G$ è una BdG rispetto a LEX con $x_1 \ge \ldots \ge x_n$ allora $\forall k = 1, \ldots, n-1$ si ha che $G_k = G \cap K[x_{k+1}, \ldots, x_n]$ è BdG di $I_k$
		\item ({\bf Cose calcolabili}) Dati $I, J \subseteq K[x_1, \ldots, x_n]$ e note le loro due BdG si ha
			\begin{itemize}
				\item ({\bf Intersezione}) $I \cap J = (tI, (1-t)J) \cap K[x_1, \ldots, x_n]$ dove quindi bisognerà usare l'ordinamento LEX con $t$ come variabile più pesante per poter usare eliminazione
				\item ({\bf Colon}) Se $\BdG(J) = \{h_1, \ldots, h_r\}$ allora $I : J = \cap_{i=1}^r (I : h_i)$. \\
					Se ora ho $f \in K[x_1, \ldots, x_n]$ e voglio calcolare $I : (f) = \{g \mid gf \in I\}$ allora ho che $I : (f) = \frac{1}{f} \cdot (I\cap (f))$, ovvero se $\BdG(I \cap (f)) = \{g_1 f, \ldots, g_k f\}$ allora ho $\BdG(I : (f)) = \{g_1, \ldots, g_k\}$
				\item ({\bf Ker di morfismi}) Sia $\Phi: K[x_1, \ldots, x_n] \rar K[y_1, \ldots, y_n]$ tale che $f_i(Y) := \Phi(x_i)$. Allora si ha $\Ker \Phi = (x_1 - f_1(Y), \ldots, x_n - f_n(Y)) \cap K[x_1, \ldots, x_n]$ ovvero bisogna calcolare l'ideale di eliminazione senza le $Y$
				\item ({\bf Appartenenza al radicale}) $f \in \sqrt{I} \sse 1 \in (I, 1-tf)$ e NON serve $K$ algebricamente chiuso
			\end{itemize}
		\item ({\bf Sistemi di equazioni polinomiali}) Cerchiamo le soluzioni comuni di $f_1 = 0, \ldots, f_n = 0$ in $K^n$. Valgono:
			\begin{itemize}
				\item ({\bf Esistenza di soluzioni}) Se $K$ è algebricamente chiuso, il sistema non ha soluzioni se e solo se $1 \in I = (f_1, \ldots, f_n)$, che si vede subito se c'è o meno con una BdG
				\item ({\bf Teorema di Estensione}) $I = (f_1, \ldots, f_k)$ e supponiamo $K$ algebricamente chiuso. $I_1 = I \cap K[x_2, \ldots, x_n]$ e $\beta \in \cV(I_1)$. $f_i = c_i(x_2, \ldots, x_n) \cdot x_1^{n_1} + \ldots \in K[x_2, \ldots, x_n][x_1]$. Se $\beta \notin \cV(c_1, \ldots, c_k)$ allora $\exists a \in K \tc (a, \beta) \in \cV(I)$. Ovvero se i termini davanti alle potenze più alte di $x_1$ non si annullano tutti su $\beta$ allora posso estendere $\beta$ ad una radice di $I$.
				\item ({\bf Conseguenza di Estensione}) Se la BdG è del tipo $\{x_1^{N_1} + \ldots, x_2^{N_2} + \ldots, \ldots, x_k^{N_k} + \ldots \}$ (deve essere di questa forma in tutte le variabili) allora la varietà è finita.
				\item ({\bf Soluzioni finite}) $K$ algebricamente chiuso. $I \subseteq A$. Allora sono fatti equivalenti:
					\begin{itemize}
						\item $\mid \cV(I) \mid < \infty$ ($\cV(I)$ è costituita da un numero finito di punti)
						\item $\forall i = 1, \ldots, n \quad \exists m_i \tc x_i^{m_i} \in \Lt(I)$
						\item $G = \{ g_1, \ldots, g_r\}$ BdG di $I$ allora $\forall i = 1, \ldots, n \quad \exists h_i \in \bbN \quad \exists g_r \in G \tc \Lt(g_r) \mid x_i^{h_i}$
						\item $\Dim_K \frac{A}{I} < \infty$
						\item $\Dim I = 0$ (come dimensione di Krull)
					\end{itemize}
					Inoltre vale che una $K$-base di $\frac{A}{I}$ è $\{x^\alpha \tc x^\alpha \notin \Lt(I)\}$, e anche $\Dim_K \frac{A}{\sqrt{I}} = \mid \cV(I) \mid$ \\
					Osservazione: Il nullstellensatz serve solo per la freccia che $\mid \cV(I) \mid < \infty$ implica una delle altre. Per le freccie inverse non serve.
			\end{itemize}
	\end{itemize}
	
	\section*{Ideali e Varietà}
	Siano $I, J, H \subseteq K[x_1, \ldots, x_n]$ ideali e $V$ varietà affine. Allora vale
	\begin{itemize}
		\item $I \subseteq J \implies \cV(J) \subseteq \cV(I)$
		\item $I \subseteq \cI(\cV(I))$
		\item $\cV(\cI(V)) = V$
		\item $\cV(I) \subseteq \cV(J) \implies \cI(\cV(J)) \subseteq \cI(\cV(I))$
		\item $\cV(I + J) = \cV(I) \cap \cV(J)$
		\item $\cV(I\cdot J) = \cV(I) \cup \cV(J) = \cV(I \cap J)$
		\item $\cV(I) = \cV(\sqrt{I})$
		\item $\cV(I, JH) = \cV(I, J) \cup \cV(I, H)$
	\end{itemize}
	Valgono inoltre i seguenti fatti:
	\begin{itemize}
		\item $V$ è irriducibile $\implies \exists \kp \text{primo} \tc V = \cV(\kp)$ (il viceversa è vero se $K$ è algebricamente chiuso)
		\item Ogni varietà affine si decompone come unione di un numero finito di varietà irriducibili. Tale decomposizione si può minimizzare nel modo seguente: se compaiono due varietà irriducibili una contenuta dentro l'altra si toglie dall'unione la più piccola. La decomposizione minimalizzata è unica a meno dell'ordine con cui compaiono i fattori irriducibili
		\item $V = \{\alpha\}$ con $\alpha = (\alpha_1, \ldots, \alpha_n)$ allora $\cI(V) = (x_1 - \alpha_1, \ldots, x_n - \alpha_n)$ è un ideale massimale. (Se $K$ è algebricamente chiuso allora $I$ è massimale se e solo se è di quella forma)
		\item ({\bf Nullstellensatz}) $K$ algebricamente chiuso. Allora $I \subseteq K[x_1, \ldots, x_n]$ e si ha:
			\begin{itemize}
				\item ({\bf Forma debole}) $\cV(I) = \emptyset \sse 1 \in I$
				\item ({\bf Forma forte}) $\cI(\cV(I)) = \sqrt{I}$
			\end{itemize}
		\item ({\bf Normalizzazione di Nöther}) $K$ infinito. Se $f$ è un polinomio in $K[x_1, \ldots, x_n] \tc f \notin I_1 = K[x_2, \ldots, x_n]$ (ovvero $x_1$ compare) allora $\exists \phi$ cambio lineare di coordinate tale che $\phi(f) = c \cdot x_1^N + \overline{f}$ con $\Deg_{x_1} \overline{f} < N$ e $c \neq 0$ costante. 
		\item $K$ algebricamente chiuso. Se $I$ è radicale allora $I = \cap_{i=1}^k P_i$ con $P_i$ primi. (Basta decomporre la varietà)
	\end{itemize}
	
	\section*{Risultante}
	\begin{itemize}
		\item ({\bf Definizione di Risultante}) Sia $R$ un dominio d'integrità, $f, g \in R[x]$ e $f = \sum_{i=0}^n a_i x^i$, $g = \sum_{i=0}^m b_i x^i$. Definiamo allora la matrice di Sylvester come
		$$ \text{Sylv}(f, g) = \left[ \begin{array}{cccccccc}
		a_0     & a_1     & \ldots  & \ldots  & a_n     & 0       & \ldots  & 0       \\
		0       & a_0     & a_1     & \ldots  & \ldots  & a_n     & 0       & 0       \\
		\vdots  &         & \ddots  &         &         &         & \ddots  & \vdots  \\
		0       & \ldots  & 0       & a_0     & a_1     & \ldots  & \ldots  & a_n     \\ \hline
		b_0     & b_1     & \ldots  & b_m     & 0       & \ldots  & \ldots  & 0       \\
		0       & b_0     & b_1     & \ldots  & b_m     & 0       & \ldots  & 0       \\
		0       & 0       & b_0     & b_1     & \ldots  & b_m     & 0       & 0       \\
		\vdots  &         &         & \ddots  &         &         & \ddots  & \vdots  \\
		0       & \ldots  & \ldots  & 0       & b_0     & b_1     & \ldots  & b_m     \\
		\end{array} \right]$$
		Ed il risultante di $f$ e $g$ è $\Ris(f, g) = \Det \text{Sylv}(f, g)$
		\item ({\bf Definizione alternativa}) $\Ris(f, g) = a_n^m b_m^n \prod_{i,j} (\alpha_i - \beta_j) = a_n^m \cdot \prod_{f(\alpha_i) = 0} g(\alpha_i) = (-1)^{mn} b_m^n \cdot \prod_{g(\beta_j) = 0} f(\beta_j)$ dove le $\alpha_i$ e le $\beta_j$ sono le radici rispettivamente di $f$ e di $g$, con molteplicità
		\item ({\bf Proprietà del risultante}) Valgono le seguenti proprietà:
			\begin{itemize}
				\item $\Ris(f, g) = (-1)^{mn} \Ris(g, f)$
				\item $\Ris(af, g) = a^m \Ris(f, g)$ con $a \in R$ scalare
				\item $\Ris(f, ag) = a^n \Ris(f, g)$ con $a \in R$ scalare
				\item $\Ris(a, b) = 1$ dove $a, b \in R$ sono scalari
				\item $\Ris(f, g) = 0 \sse \exists \alpha \in \overline{R} \tc f(\alpha) = g(\alpha) = 0$ (ovvero il risultante è nullo se e solo se $f$ e $g$ hanno una radice in comune nella chiusura algebrica del campo delle frazioni di $R$). Inoltre, se $R$ è UFD allora le due precedenti sono equivalenti a $\exists h \in R[x] \tc \Deg h > 0, h \mid f, h \mid g$
				\item $f, g \in R[x]$ e $\Deg f = n, \Deg g = m$, allora $\Ris(f,g) = Af + Bg$ con $A, B \in R[x]$ e $\Deg A < m, \Deg B < n$
				\item $\Ris(f, h_1 \cdot h_2) = \Ris(f, h_1) \cdot \Ris(f, h_2)$
				\item $\Ris(f, hf+g) = a_m^{\Deg (hf + g) - \Deg g} \cdot \Ris(f, g)$ [ATTENZIONE: della formula a fianco non sono completamente sicuro]
				%$\Ris(f, hf+g)=a_m^{\Deg(hf+g)}\Prod_{f(\alpha_i)=0}{(hf+g)(\alpha_i)}=a_m{^\Deg(hf+g)}\Prod_{f(\alpha_i)=0}{g(\alpha_i)}=a_m^{\Deg(hf+g)-\Deg g}\cdot a_m^{\Deg g}\Prod_{f(\alpha_i)=0}{g(\alpha_i)}=a_m^{\Deg (hf + g) - \Deg g} \cdot \Ris(f, g)$
				\item In molti casi vale che $\Ris(f,g) \mid_\alpha = \Ris(f\mid_\alpha, g\mid_\alpha)$ dove con $\mid_\alpha$ si intende la valutazione in $\alpha$. Bisogna solo stare attenti che almeno uno dei coefficienti direttivi valutati sia non nullo, altrimenti cambia la dimensione della matrice di sylvester e di conseguenza anche il polinomio che definisce il risultante
				\item Può essere comodo sapere che, detti $a_i$ e $b_j$ i coefficienti di $f$ e di $g$, si ha che $\Ris(f, g) \in \bbZ[a_i, b_j]$
			\end{itemize}
		\item ({\bf Trucchi utili con il risultante}) Dati $f = \prod_i (x - \alpha_i)$ e $g = \prod_j (x - \beta_j)$, allora si possono costruire i seguenti polinomi:
			\begin{itemize}
				\item $\Ris_y (f(x-y), g(y))$ ha radici $\gamma_{i,j} = \alpha_i + \beta_j$
				\item $\Ris_y (f(x+y), g(y))$ ha radici $\gamma_{i,j} = \alpha_i - \beta_j$
				\item $\Ris_y (y^{\Deg f} f(\frac{x}{y}), g(y))$ ha radici $\gamma_{i,j} = \alpha_i \cdot \beta_j$
				\item Se $g(0) \neq 0$ allora $\Ris_y (f(xy), g(y))$ ha radici $\gamma_{i,j} = \frac{\alpha_i}{\beta_j}$
			\end{itemize}
	\end{itemize}
	
	\section*{Moduli}
	\subsection*{Primi fatti}
	\begin{itemize}
		\item ({\bf Fregatura dei Moduli}) Attenzione che le seguenti cose non sono sempre vere su moduli generici:
			\begin{itemize}
				\item Non sempre esiste una base
				\item Un sistema di generatori minimale non è necessariamente una base
				\item Un insieme libero massimale non è necessariamente una base
				\item Due sistemi di generatori minimali non hanno necessariamente la stessa cardinalità (e nemmeno gli insiemi liberi massimali)
			\end{itemize}
		\item ({\bf Cardinalità di una base di un modulo libero}) Se un modulo $M$ è libero, allora ogni base ha la stessa cardinalità. Inoltre ogni insieme di generatori di $M$ ha cardinalità maggiore o uguale a quella di una base.
		\item ({\bf Omomorfismi di $A$-Moduli}) Dati due $A$-Moduli $M$ ed $N$, allora si ha che anche $\Hom_A(M, N)$ è un $A$-modulo con le operazioni di somma e di prodotto scalare effettuate in arrivo. (Notare che questa proprietà è particolarmente strana e ci tornerà utile più volte). \\
		Inoltre si può notare come dato un omomorfismo $f: M \rar N$ di $A$-moduli si ha che $\Ker f = \{ m \in M \mid f(m) = 0 \}$ ed $\Img f = \{ f(m) \mid m \in M \}$ sono entrambi due sottomoduli rispettivamente di $M$ e di $N$. Allora possiamo anche sempre definire $\coKer f = \frac{N}{\Img f}$
		\item ({\bf Fatti di base e definizioni di operazioni importanti}) Valgono le seguenti cose:
			\begin{itemize}
				\item $\Hom_A(A, M) \cong_{\text{A-mod}} M$. Infatti conoscere il valore di $f(1)$ caratterizza tutto l'omomorfismo $f$, visto che è di $A$-moduli
				\item $L \subseteq N \subseteq M$ allora vale $\frac{M}{N} \cong_{\text{A-mod}} \frac{\frac{M}{L}}{\frac{N}{L}}$
				\item $M_1, M_2 \subseteq M$ sottomoduli. $M_1 + M_2 := \{ m_1 + m_2 \mid m_1 \in M_1, m_2 \in M_2 \}$ allora vale che $\frac{M_1 + M_2}{M_2} \cong_{\text{A-mod}} \frac{M_1}{M_1 \cap M_2}$
				\item ({\bf $\frac{A}{I}$-moduli}) Dato $I \subseteq A$ idale ed $M$ modulo si può definire $IM = \{ \sum_i a_i m_i \mid a_i \in I, m_i \in M \}$ e si verifica che è un sottomodulo di $M$. Inoltre vale che $\frac{M}{IM}$ è anche un $\frac{A}{I}$-modulo. \\
				Possiamo invece notare che $M$ non è sempre un $\frac{A}{I}$-modulo. Ci possiamo però riuscire se $I \subseteq (0 : M) = \{ a \in A \mid aM \subseteq (0) \}$.
				\item ({\bf Somma diretta e prodotto}) Dati $\{M_i\}_{i \in I}$ una famiglia di $A$-moduli si definisce $$\oplus_i M_i = \{ (a_i)_{i \in I} \mid a_i \in M_i, a_i \neq 0 \text{ solo per un numero finito di indici}\}$$ Inoltre si definisce $$\prod_i M_i = \{ (a_i)_{i \in I} \mid a_i \in M_i \}$$ senza la condizione di sopra. \\
				Se l'insieme $I$ di indici è finito allora si ha che $\oplus_i M_i = \prod_i M_i$. Valgono inoltre le seguenti proprietà universali per somma diretta e prodotto:
				\begin{itemize}
					\item Dati $\{M_i\}_{i \in I}$ $A$-moduli, si hanno $M_i \hrar^{j_i} \oplus_i M_i$ date da $m_i \mapsto (0, \ldots, 0, m_i, 0, \ldots)$. Allora per ogni assegnamento di $\{\varphi_i\}_{i \in I}$ con $\varphi_i: M_i \rar N$ omomorfismi di $A$-moduli, esiste unico $\tilde\phi: \oplus_i M_i \rar N$ tale che $\varphi_i = \tilde\phi \circ j_i$
					\item Dati $\{M_i\}_{i \in I}$ $A$-moduli, si hanno $\prod_i M_i \srar^{\pi_i} M_i$ le proiezioni date da $m = (m_j)_{j \in I} \mapsto m_i$. Allora per ogni assegnamento di $\{\varphi_i\}_{i \in I}$ con $\varphi_i: N \rar M_i$ omomorfismi di $A$-moduli, esiste unico $\tilde\phi: N \rar \prod_i M_i$ tale che $\varphi_i = \pi_i \circ \tilde\phi$
				\end{itemize}
			\end{itemize}
		\item ({\bf Morfismi da un modulo libero}) Sia $M$ un $A$-modulo libero e sia $S = \{s_1, \ldots, s_k\}$ una sua base. Allora dati $n_1, \ldots, n_k \in N$ ($N$ è un altro $A$-modulo) si ha che $\exists ! \Phi: M \rar N$ tale che $\Phi(s_i) = n_i$, $\Phi$ morfismo di $A$-moduli
		\item ({\bf Rango di un modulo libero}) Sia $M$ un $A$-modulo libero con base $B = \{b_1, \ldots, b_k\}$ finita. Allora ogni altra base di $M$ ha cardinalità $k$. Se $M$ è libero con base di cardinalità $k$ si dice che $M$ ha rango $k$ ($\Rk M = k$)
		\item $\Hom_A(A^n, M) \cong M^n$.
		\item $M$ è un $A$-modulo finitamente generato $\sse M \cong \frac{A^k}{\Ker \varphi}$ per un certo $k \in \bbN$ e per un certo $\varphi$. Se $M = \gen{m_1, \ldots, m_k}$ si ha $\varphi: A^k \rar M$ definito da $e_i \mapsto m_i$. Allora $M \cong \frac{A^k}{\Ker \varphi}$. Il viceversa è ovvio.
		\item ({\bf Hamilton-Cayley}) Sia $M$ un $A$-modulo finitamente generato, $I \subseteq A$ ideale. Sia $\varphi \in \Hom_A(M, M)$ endomorfismo tale che $\phi(M) \subseteq IM$. Allora $\exists b_0, \ldots, b_{n-1} \in I \tc \phi^n + \sum_{i=0}^{n-1} a_i \phi^i = 0$ in $\Hom_A(M, M)$
		\item ({\bf Nakayama}) Come corollario di Hamilton-Cayley si ottengono le seguenti tre versioni di Nakayama:
			\begin{itemize}
				\item Sia $M$ un $A$-modulo finitamente generato, $I \subseteq A$ ideale $\tc M = IM$. Allora $\exists a \in A \tc a \equiv 1 (\mod I)$ e $a \cdot M = 0$ (Basta applicare HC a $\varphi = \Id$)
				\item Sia $M$ un $A$-modulo finitamente generato, $\cJ(A)$ radicale di Jacobson, $I \subseteq \cJ(A)$ ideale di $A$ tale che $M = IM$. Allora $M = 0$ (Usiamo il Nakayama precedente ed usiamo la caratterizzazione del radicale di Jacobson)
				\item Sia $M$ un $A$-modulo finitamente generato, $N$ un sottomodulo, $I \subseteq \cJ(A)$ ideale di $A$. Se $M = N + IM$ allora $M = N$ (Usando il Nakayama precedente basta mostrare che $\frac{M}{N} = I(\frac{M}{N})$ così che $\frac{M}{N} = (0) \implies M = N$ e questo è piuttosto semplice)
			\end{itemize}
			Come corollario otteniamo che se $A$ è un anello locale e $\km$ un suo ideale massimale, $M$ un $A$-modulo finitamente generato. Allora se $n_1, \ldots, n_k$ sono elementi di $M$ tali che si ha che $\overline{n_1}, \ldots, \overline{n_k}$ generano $\frac{M}{\km M}$ come $\frac{A}{\km}$-modulo (ovvero come spazio vettoriale) allora $n_1, \ldots, n_k$ generano $M$ come $A$-modulo (considerare $N \hrar M \srar \frac{M}{\km M}$ e usare Nakayama 3) \\
			Come altro corollario sia $M$ un $A$-modulo finitamente generato, $f \in \End_A(M)$ surgettivo $\implies f$ è un isomorfismo.
		\item ({\bf Funtori $f^*$ e $g_*$}) Se ho $f: P \rar M$ allora posso considerare $f^*: \Hom_A(M, N) \rar \Hom_A(P, N)$ definito da $\phi \mapsto \phi \circ f$. Notiamo che è contravariante. \\
		Inoltre dato $g: M \rar P$ si ha $g_*: \Hom_A(N, M) \rar \Hom_A(N, P)$ definito da $\psi \mapsto g \circ \psi$, che è covariante. 
	\end{itemize}
	
	\subsection*{Omomorfismi tra moduli liberi e forma normale di Smith}
	\begin{itemize}
		\item Ogni elemento di $\Hom_A(A^m, A^n)$ si può rappresentare in modo unico come matrice, quindi mi basta sapere dove vanno gli $e_i$ base di $A^m$ per sapere dove vanno tutti gli altri elementi. Inoltre una matrice sarà invertibile se e solo se il suo determinante è un elemento invertibile dell'anello (Basta usare l'aggiunta sapendo che $M M^* = (\Det M) \Id$)
		\item $S, T$ matrici si dicono equivalenti per righe se $\exists P$ invertibile tale che $PS = T$, equivalenti per colonne se $\exists Q$ invertibile tale che $SQ = T$ e si dicono equivalenti se $\exists P, Q$ tali che $PSQ = T$
		\item Se $A$ è PID, allora si ha che ogni matrice è equivalente ad una matrice diagonale ($D$ si dice diagonale se $D_{ij} = 0$ quando $i \neq j$). \\
			Il trucco fondamentale è che sui blocchetti $2\times 2$ riesco a triangolarli. Infatti, usando che $A$ è PID si ha $d = \MCD(a, b)$ e quindi $\exists s, t \tc d = as + bt$ ovvero $$ \left( \begin{array}{cc} a & b \\ u & v \\ \end{array} \right) \cdot \left[ \begin{array}{cc} s & - \frac{b}{d} \\ t & \frac{a}{d} \\ \end{array} \right] = \left( \begin{array}{cc} d & 0 \\ w & x \\ \end{array} \right)$$ e trasponendo la relazione si riesce anche a portare in forma triangolare superiore. \\
			Il modo generale di procedere è piuttosto semplice: con il metodo precedente si pongono a zero tutti i numeri sulla prima riga tranne il primo, a questo punto si mettono a zero tutti i numeri sulla prima colonna tranne il primo, e si procede riga-colonna fino a quando non sono nulli sia tutti i numeri sulla prima riga che sulla prima colonna (tranne ovviamente il primo). Questa cosa deve succere prima o poi. Quando accade si ricorre per induzione sulla sottomatrice $(n-1) \times (n-1)$ che si ottiene levando la prima riga e la prima colonna.
		\item ({\bf Forma normale di Smith}) $A$ PID. Vogliamo dare una forma canonica alle matrici che rappresentano gli omomorfismi tra moduli liberi. Una matrice diagonale $D$ si dice in forma di Smith se $d_1 \mid d_2 \mid \ldots \mid d_n$ con $D = \left( \begin{array}{cccc} d_1 & & & \\ & d_2 & & \\ & & \ddots & \\ & & & d_n \\ \end{array} \right)$
		\item ({\bf Ogni matrice diagonale si può portare in forma di Smith}) Infatti data $\left( \begin{array}{cc} a & 0 \\ 0 & b \\ \end{array} \right)$ e detto $d = \MCD(a, b) = as + bt$ si computa $\left( \begin{array}{cc} s & t \\ -\frac{b}{d} & \frac{a}{d} \\ \end{array} \right) \cdot \left( \begin{array}{cc} a & 0 \\ 0 & b \\ \end{array} \right) \cdot \left( \begin{array}{cc} 1 & - \frac{tb}{d} \\ 1 & \frac{sa}{d} \\ \end{array} \right) = \left( \begin{array}{cc} d & 0 \\ 0 & \frac{ab}{d} \\ \end{array} \right)$
		\item ({\bf Caratterizzazione tramite ideali determinanti}) Se $S$ è una matrice definiamo $\Delta_i(S)$ come l'ideale generato dai determinanti delle sottomatrici $i \times i$ di $S$. Se $S, T$ $m \times n$ sono equivalenti allora $\Delta_i S = \Delta_i T \quad \forall i$. Se $D_1$ e $D_2$ sono matrici in forma di Smith allora $D_1$ è equivalente a $D_2$ se e solo se $d_i^{(1)}$ e $d_i^{(2)}$ differiscono di un invertibile (ovvero sono associati). Inoltre si ha che i $d_i$ sono $d_i = \frac{\Delta_{i}}{\Delta_{i-1}}$ per $i \ge 1$ (dove convenzionalmente $\Delta_0 = 1$)
		\item ({\bf Sottomoduli di moduli liberi su PID}) Se $M$ è un $A$-modulo libero con $A$ PID e $N \subseteq M$ sottomodulo, allora $N$ è libero e inoltre vale che $\Rk N \le \Rk M$
		\item ({\bf Teorema di struttura di moduli f.g. su PID}) Ogni modulo finitamente generato su PID si scrive come somma diretta di moduli ciclici. $M$ f.g. su PID (ovvero è quoziente di un modulo libero). $M = \gen{m_1, \ldots, m_k}$. Allora $A^n \rar^f M \rar 0$ con $f(e_i) = m_i$ e $f(a_1, \ldots, a_n) = \sum_i a_i m_i$ ovvero $M \cong \frac{A^n}{\Ker f}$ e $\Ker f \subseteq A^n$ è un sottomodulo di modulo libero. \\
		Sapendo che ogni sottomodulo di modulo libero su PID è libero abbiamo che $A^m \rar^\phi A^k \rar^f M \rar 0$ allora $M \cong \frac{A^m}{\Ker f} \cong \frac{A^k}{\Img \phi} \cong \coKer \phi \cong \oplus_i \frac{A}{(d_i)} \cong \oplus_i \gen{z_i}$ con $d_i = \Ann(z_i)$
		\item Se $M = \gen{m}$ è un $A$-modulo ciclico allora $M \cong \frac{A}{\Ann(m)}$
		\item $M = \frac{A}{J}$ come $A$-modulo. Dato $a \in A$ si ha $(a) \cdot M \cong \frac{A}{(J \cdot (a))}$
		\item $A^n \cong A^m \sse n = m$
		\item $\phi: A^m \srar A^n$ surgettivo e $m < n \implies A = 0$
		\item $M = \frac{A}{J_1} \oplus \frac{A}{J_2}$, con $I \subseteq A$ ideale. Allora valgono:
			\begin{itemize}
				\item $IM \cong \frac{I + J_1}{J_1} \oplus \frac{I + J_2}{J_2}$
				\item $\frac{M}{IM} \cong \frac{A}{I + J_1} \oplus \frac{A}{I + J_2}$
			\end{itemize}
		\item Sia $M$ un $A$-modulo finitamente generato su PID allora $M$ si scrive come somma diretta di moduli ciclici $M = \gen{m_1} \oplus \ldots \oplus \gen{m_k}$
		\item Se ho due catene di ideali $I_n \subseteq \ldots \subseteq I_1$, $J_m \subseteq \ldots \subseteq J_1$ con $n \ge m$, e supponiamo $M = \oplus_{k=1}^n \frac{A}{I_k} = \oplus_{i=1}^m \frac{A}{J_i}$ allora $J_1 = \ldots = J_{n-m} = A$ e $I_i = J_{n-m+i}$
		\item Se $A$ è un dominio ed $M$ un $A$-modulo, allora chiamiamo sottomodulo di torsione $\tau(M) = \{ m \in M \mid \Ann(m) \neq 0 \} \subseteq M$.
			\begin{itemize}
				\item $f \in \Hom_A(M, N) \implies f(\tau(M)) \subseteq \tau(M)$
				\item Data $0 \rar M \rar N \rar P \rar 0$ esatta $\implies 0 \rar \tau(M) \rar \tau(N) \rar \tau(P)$ è esatta ma non a destra
				\item $M$ f.g. su $A$ PID. Allora $M \cong \tau(M) \oplus A^k$ per un qualche $k$
			\end{itemize}
		\item $M$ si dice modulo $p$-primario se $\Ann(M) = (p^s)$
		\item ({\bf Riassunto di tutto}) $M$ f.g. su $A$ PID. allora valgono:
			\begin{itemize}
				\item $M = (\oplus_{i=1}^m \frac{A}{(d_i)}) \oplus A^k$ con $d_1 \mid \ldots \mid d_m$ non necessariamente distinti, unicamente determinati a meno di associati. Tali $d_i$ si chiamo fattori invarianti di $M$.
				\item $M \cong (\oplus_{p_i} M_{p_i}) \oplus A^k$ dove gli $M_{p_i}$ sono moduli ciclici $p_i$-primari di torsione. Tutti i $p_1^{s_1} \ldots p_r^{s_r}$ si chiamano divisori elementari di $M$. \\
				Infatti se $\tau(M) = \oplus_i \frac{A}{(d_i)}$ con $d_i \in A$ PID allora se $d_i = p_{i_1}^{s_1} \cdot \ldots \cdot p_{i_k}^{s_k} \implies \frac{A}{(d_i)} = \oplus_{j=1}^k \frac{A}{p_{ij}^{s_j}}$
			\end{itemize}
	\end{itemize}
	
	\subsection*{Prodotto tensore}
	\begin{itemize}
		\item ({\bf Proprietà universale}) Sia $R$ un anello, $M, N$ due $R$-moduli. Un prodotto tensore di $M$ e $N$ è un $R$-modulo denotato con $M \otimes_R N$ con una mappa $\tau: M \times N \rar M \otimes_R N$ bilineare tale che $\forall \phi: M \times N \rar P$ bilineare (con $P$ un generico $R$-modulo) $\exists ! \tilde\phi: M \otimes_R N \rar P$ tale che $\phi = \tilde\phi \circ \tau$ \\
		Deriva dalla definizione che se un tale modulo esiste allora è unico a meno di unico isomorfismo. \\
		Si può costruire in maniera piuttosto semplice sui moduli prendendo l'$R$-modulo libero generato dagli elementi di $M \times N$ e quozientando per il sottomodulo delle relazioni, ovvero il generato da $i(m_1+m_2, n) - i(m_1, n) - i(m_2,n)$, $i(m, n_1+n_2) - i(m, n_1) - i(m, n_2)$, $i(r-m, n) - r i(m, n)$, $i(m, rn) - r i (m,n)$
		\item ({\bf Tensori semplici}) Una cosa della forma $m \otimes n$ in $M \otimes_R N$ è detto tensore semplice. L'insieme dei tensori semplici genera $M \otimes_R N$ come $R$-modulo. Inoltre se $\{m_\alpha\}$ genera $M$ e $\{n_\beta\}$ genera $N$, allora $\{m_\alpha \otimes n_\beta\}$ genera $M \otimes_R N$
		\item ({\bf Formule di uguaglianza}) Valgono le seguenti cose, alcune ovvie alcune meno:
			\begin{itemize}
				\item $R \otimes_R M \cong M$
				\item $M \otimes_R N \cong N \otimes_R M$
				\item $(M \otimes_R N) \otimes_R P \cong M \otimes_R (N \otimes_R P)$
				\item $(M \oplus N) \otimes_R P \cong (M \otimes_R P) \oplus (M \otimes_R P)$ (vale anche per somme dirette infinite)
				\item $\frac{R}{I} \otimes_R M \cong \frac{M}{IM}$
				\item $M_\kp \otimes_{A_\kp} N_\kp = (M \otimes_A N)_\kp$
				\item $\frac{A}{I} \otimes_A \frac{A}{J} \cong \frac{A}{I+J}$
				\item $\Hom_A(\frac{A}{I}, \frac{A}{J}) \cong \frac{(J : I)}{J}$ [Non fatta in classe]
			\end{itemize}
		\item ({\bf Aggiunzione con $\Hom$}) $M, N, P$ tre $R$-moduli. Allora vale che $\Hom_R(M \otimes_R N, P) \cong \Hom_R(M, \Hom_R(N, P))$ dove l'isomorfismo è naturale (e vale in particolare anche a livello di $R$-moduli)
		\item ({\bf Esattezza a destra}) Essendo aggiunto sinistro il funtore $\textunderscore \otimes_R M$ (o anche $M \otimes_R \textunderscore$, che è canonicamente equivalente al primo) è esatto a destra, cioè: \\
		$M \rar N \rar P \rar 0$ è esatta $\sse \forall Q \qquad M \otimes Q \rar N \otimes Q \rar P \otimes Q \rar 0$ è esatta
		\item ({\bf Implicazioni varie})
			\begin{itemize}
				\item $M, N$ f.g. $\implies M \otimes_R N$ f.g.
				\item $M, N$ liberi $\implies M \otimes_R N$ libero
			\end{itemize}
	\end{itemize}
	
	\subsection*{Anello e Modulo delle frazioni}
	\begin{itemize}
		\item ({\bf Anello delle frazioni}) $A$ anello ed $S \subseteq A$ moltiplicativamente chiuso ($1 \in S, s,t \in S \implies st \in S$). Allora l'insieme $A \times S$ quozientato per la relazione di equivalenza $(a, s) \sim (b, t) \sse \exists u \neq 0 \in S \tc u (at - bs) = 0$ è un anello dotato di una mappa $A \rar \frac{A \times S}{\sim}$ tale per cui ogni elemento di $S$ va a finire in un invertibile. \\
		Gode inoltre della proprietà universale per la quale per ogni altro anello $B$ e morfismo di anelli $g: A \rar B$ tale che tutti gli elementi di $S$ vadano a finire in elementi invertibili di $B$, allora questo morfismo si spezza in modo unico attraverso il passaggio per $S^{-1}A := \frac{A \times S}{\sim}$
		\item ({\bf Nullità dell'anello delle frazioni}) $0 \in S \sse S^{-1}A = 0$
		\item ({\bf Ideali di $S^{-1}A$}) Valgono le seguenti affermazioni sugli ideali di $S^{-1}A$:
			\begin{itemize}
				\item Ogni ideale di $S^{-1}A$ è un ideale esteso
				\item Sia $I \subseteq A$ ideale. Allora $I^e = 1 \sse I \cap S \neq \emptyset$
				\item $I^{ec} = \cup_{s \in S} (I : s)$
				\item C'è una corrispondenza biunivoca tra gli ideali primi di $A$ che non intersecano $S$ ed i primi di $S^{-1}A$. Infatti:
					\begin{itemize}
						\item Se $Q$ è primo in $S^{-1}A$ allora $Q^c$ è primo in $A$ (e questo è sempre vero)
						\item $P$ primo in $A$, $P \cap S = \emptyset \quad \implies S^{-1}P$ primo
					\end{itemize}
				\item $P_1$, $P_2$ ideali primi. Allora si ha $S^{-1}P_1 = S^{-1}P_2 \implies P_1 = P_2$
				\item $Q \subseteq A$ ideale $P$-primario. Allora se $S \cap P \neq \emptyset$ si ha $S^{-1}Q = S^{-1}A$ \\
					Se $S \cap P = \emptyset$ allora $S^{-1}Q$ è $S^{-1}P$-primario ed inoltre $(S^{-1}Q)^c = Q$
			\end{itemize}
		\item ({\bf $S^{-1}$ e le altre operazioni}) Potremmo dire in linea di massima che $S^{-1}$ commuta con tutte le operazioni principali, purché siano finite:
			\begin{itemize}
				\item $S^{-1}(I + J) = S^{-1}I + S^{-1}J$
				\item $S^{-1}(I \cap J) = S^{-1}I \cap S^{-1}J$
				\item $S^{-1}(I \cdot J) = (S^{-1}I) \cdot (S^{-1}J)$
				\item $S^{-1}\sqrt{I} = \sqrt{S^{-1}I}$
			\end{itemize}
		\item ({\bf Modulo delle frazioni}) Sia $M$ un $A$-modulo ed $S \subseteq A$ un insieme moltiplicativamente chiuso. Allora definiamo $S^{-1}M := \frac{S \times M}{\sim}$ dove $(m, s) \sim (m', s') \sse \exists u \in S \quad u (s'm - sm') = 0$ ed indicheremo con $\frac{m}{s}$ la classe di equivalenza. \\
		$S^{-1}M$ ha una struttura di $S^{-1}A$-modulo. Inoltre si può facilmente verificare che $S^{-1}$ è un funtore dalla categoria degli $A$-moduli a quella degli $S^{-1}A$-moduli, dove dato $f: M \rar N$ morfismo si può definire $S^{-1}f: S^{-1}M \rar S^{-1}N$ come $(S^{-1}f) (\frac{m}{s}) = \frac{f(m)}{s}$
		\item ({\bf $S^{-1}$ è un funtore esatto}) Si ha che $S^{-1}$ è esatto, ovvero se $M \rar^f N \rar^g P$ è una sequenza esatta di $A$-moduli allora $S^{-1}M \rar^{S^{-1}f} S^{-1}N \rar^{S^{-1}g} S^{-1}P$ è una sequenza esatta di $S^{-1}A$-moduli. \\
		In particolare omomorfismi iniettivi o surgettivi rimangono rispettivamente iniettivi o surgettivi
		\item ({\bf $S^{-1}$ e le altre operazioni}) Siano $M, P \subseteq N$ sotto$A$-moduli, $S \subseteq A$ moltiplicativamente chiuso. Allora $S^{-1}$ commuta con somme finite, intersezioni finite e quozienti, ovvero vale che
			\begin{itemize}
				\item $S^{-1}(M + P) = S^{-1}M + S^{-1}P$
				\item $S^{-1}(M \cap P) = S^{-1}M \cap S^{-1}P$
				\item $S^{-1}(\frac{N}{M}) \cong \frac{S^{-1}N}{S^{-1}M}$ dove l'isomorfismo è come $S^{-1}A$-moduli
				\item Se $M$ è f.g. allora $\Ann(S^{-1}M) = S^{-1}\Ann(M)$
				\item Sapendo che $(N:P) = \Ann(\frac{N+P}{N})$ si può mostrare che se $P$ è f.g. allora $S^{-1}(N:P) = (S^{-1}N : S^{-1}P)$
			\end{itemize}
		Valgono inoltre le seguenti uguaglianze furbe:
			\begin{itemize}
				\item $S^{-1}A \otimes_A M = S^{-1}M$
				\item $S^{-1}(M \otimes_A N) \cong S^{-1}M \otimes_{S^{-1}A} S^{-1}N$
			\end{itemize}
		\item ({\bf Correlazione tra anello e modulo delle frazione e prodotto tensore}) Vale che $S^{-1}A \otimes_A M = S^{-1}M$. \\
			Inoltre abbiamo anche dimostrato che $S^{-1}A$ è un $A$-modulo piatto, ovvero $0 \rar M \rar^f N$ rimane iniettiva tensorizzando per il piatto, cioè $0 \rar S^{-1}A \otimes_A M \rar^{S^{-1}A \otimes_A f} S^{-1}A \otimes_A N$ per l'osservazione precedente.
		\item ({\bf Altri fatti})
			\begin{itemize}
				\item $f: A \rar B$ omomorfismo di anelli, $S \subseteq A$ motliplicativamente chiuso e $T = f(S)$. Allora $S^{-1}B \cong T^{-1}B$ come $S^{-1}A$-moduli
				\item $S \subseteq A$ molt. chiuso. Diciamo che $S$ è saturato se $xy \in S \implies x \in S, y \in S$. Si ha allora che:
					\begin{itemize}
						\item $S$ saturato $\sse S = A \setminus \cup_{\kp \cap S = \emptyset} \kp$
						\item Se $S$ è un sistema molt. chiuso allora $\exists ! S \subseteq \overline{S}$ con $\overline{S}$ saturato e minimale rispetto alla proprietà di contenere $S$.
						\item $\overline{S}^{-1}A \cong S^{-1}A$
					\end{itemize}
			\end{itemize}
	\end{itemize}
	
	\subsection*{Localizzazione e proprietà locali}
	\begin{itemize}
		\item ({\bf Definizione}) $A$ anello, $\kp \subseteq A$ ideale primo e consideriamo $S = A \setminus \kp$ che è moltiplicativamente chiuso. Allora $A_\kp := S^{-1}A$ si dice localizzazione a $\kp$. Si ha che $A_\kp$ è un anello locale, dove l'unico ideale massimale è $\kp_\kp = \{ \frac{a}{s} \mid a \in \kp, s \notin \kp \}$
		\item ({\bf Proprietà locali}) $P$ è una proprietà per $A$ anello oppure per $M$ modulo si dice che è locale se \\
			$P$ vale per $A$ (o per $M$) $\sse P$ vale per $A_\kp$ (o $M_\kp$) $\forall \kp$ primo
		\item ({\bf Essere nullo è una proprietà locale (e anche massimale)}) $M$ un $A$-modulo. TFAE:
			\begin{itemize}
				\item $M = 0$
				\item $M_\kp = 0 \quad \forall \kp$ primo
				\item $M_\km = 0 \quad \forall \km$ massimale
			\end{itemize}
		\item ({\bf Per un omomorfismo essere iniettivo (o surgettivo) è una proprietà locale (e anche massimale)}) Sia $f: M \rar N$. TFAE:
			\begin{itemize}
				\item $f$ iniettivo (surgettivo)
				\item $f_\kp: M_\kp \rar N_\kp$ iniettivo (surgettivo) $\forall \kp$ primo
				\item $f_\km: M_\km \rar N_\km$ iniettivo (surgettivo) $\forall \km$ massimale
			\end{itemize}
			Per l'iniettività basta mostrare che $\Ker f_\kp = (\Ker f)_\kp$ e usare che $M = 0$ è locale. Uguale per la surgettività con i $\coKer$
		\item ({\bf Essere ridotto è una proprietà locale}) Un anello infatti è ridotto se $\cN(A) = 0$ e abbiamo mostrato che $S^{-1}\cN(A) = \cN(S^{-1}A)$, ovvero $N(A) = 0 \sse N(A)_\kp = N(A_\kp) = 0 \quad \forall \kp$ primo
		\item ({\bf Dominio NON è una proprietà locale})
		\item ({\bf L'esattezza è una proprietà locale e massimale}) $M \rar N \rar P$ è esatta in $N$ se e solo se lo solo le sequenze localizzate ai primi o ai massimali [Questo ci viene detto da D.A. ma non è stato fatto a lezione]
	\end{itemize}
	
	\subsection*{Successioni Esatte di Moduli}
	\begin{itemize}
		\item La successione $M_1 \rar^f M \rar^g M_2 \rar 0$ è esatta $\sse$ la successione $0 \rar \Hom_A(M_2, N) \rar^{g^*} \Hom_A(M, N) \rar^{f^*} \Hom_A(M_1, N)$ è esatta $\forall N$ $A$-moduli.
		\item La successione $0 \rar M_1 \rar^f M \rar^g M_2$ è esatta $\sse$ la successione $0 \rar \Hom_A(N, M_1) \rar^{f^*} \Hom_A(N, M) \rar^{g^*} \Hom_A(N, M_2)$ è esatta $\forall N$ $A$-moduli.
		\item ({\bf Successioni che spezzano}) Data una successione esatta corta di $A$-moduli $0 \rar M \rar^\alpha N \rar^\beta P \rar 0$ si ha TFAE:
			\begin{itemize}
				\item $N \cong M \oplus P$
				\item $\exists r: N \rar M \tc r \circ \alpha = \Id_M$
				\item $\exists s: P \rar N \tc \beta \circ s = \Id_P$
			\end{itemize}
		\item ({\bf Proprietà estremi-intermedio}) Sia $0 \rar M \rar^\alpha N \rar^\beta P \rar 0$ una successione esatta di $A$-moduli. Allora valgono le seguenti:
			\begin{itemize}
				\item $M, P$ f.g $\implies N$ f.g (Il viceversa non vale)
			\end{itemize}
		\item ({\bf Moduli Proiettivi}) $P$ si dice proiettivo se vale una delle seguenti, tutte equivalenti:
			\begin{itemize}
				\item Data $\phi: M \srar N$ surgettivo si ha $\forall f: P \rar N$, $\exists g: P \rar M$ tale che $f = \phi \circ g$
				\item $\forall g: M \srar N$ surgettiva l'omomorfismo indotto $\Hom_A(P,M) \rar^{g^*} \Hom_A(P,N)$ è surgettivo
				\item Ogni successione esatta corta $0 \rar M \rar N \rar P \rar 0$ spezza
				\item $P$ è sommando diretto di un modulo libero (ovvero $\exists F$ libero $\tc F = P \oplus C$)
				\item $0 \rar K \rar M \rar N \rar 0$ esatta $\sse 0 \rar \Hom_A(P, K) \rar \Hom_A(P, M) \rar \Hom_A(P, N) \rar 0$ esatta \\
					Ovvero anche $\Hom_A(P, \textunderscore)$ è un funtore esatto
			\end{itemize}
			Hanno inoltre le seguenti proprietà rispetto ad alcune costruzioni:
			\begin{itemize}
				\item $P_1 \oplus P_2$ proiettivo $\sse P_1$ e $P_2$ sono proiettivi
				\item $P_1, P_2$ proiettivi $\implies P_1 \otimes_R P_2$ proiettivo (il viceversa non vale)
			\end{itemize}
		\item ({\bf Moduli Iniettivi}) $Q$ si dice modulo iniettivo se vale una delle seguenti, tutte equivalenti:
			\begin{itemize}
				\item Per ogni $f: N \hrar M$ iniettiva e $g: N \rar Q$ si ha $\exists G: M \rar Q$ tale che $g = G \circ f$
				\item Ogni successione esatta $0 \rar Q \rar M \rar N \rar 0$ spezza
				\item $\forall g: N \hrar M$ iniettiva l'omomorfismo indotto $\Hom_A(M, Q) \rar^{g_*} \Hom_A(N, Q)$ è iniettivo
				\item Per ogni $I \subseteq A$ ideale vale la caratterizzazione (1) con $N = I$ e $M = A$ \\
					Si può dire anche per per ogni $I$ ideale di $A$ si ha che ogni $f: I \rar Q$ si estende ad una funzione $\tilde{f}: A \rar Q$ [Su questa serbiamo qualche dubbio]
				\item $0 \rar K \rar M \rar N \rar 0$ esatta $\sse 0 \rar \Hom_A(N, Q) \rar \Hom_A(M, Q) \rar \Hom_A(K, Q) \rar 0$ esatta \\
					Ovvero anche $\Hom_A(\textunderscore, Q)$ è un funtore esatto
			\end{itemize}
% Non ho idea se siano vere oppure no
%			Hanno inoltre le seguenti proprietà rispetto ad alcune costruzioni:
%			\begin{itemize}
%				\item $Q_1 \oplus Q_2$ iniettivo $\sse Q_1, Q_2$ iniettivi
%				\item NON è invece detto che $P_1, P_2$ iniettivo $\implies P_1 \otimes P_2$ iniettivo FALSO
%			\end{itemize}
		\item ({\bf Moduli Piatti}) $N$ è un $R$-modulo piatto se il funtore $N \otimes_R \textunderscore$ è esatto \\
			Valgono le seguenti proprietà rispetto alle costruzioni:
			\begin{itemize}
				\item $N, M$ piatti $\sse N \oplus M$ piatto
				\item $N, M$ piatti $\implies N \otimes_R M$ piatto (il viceversa non vale)
				\item $S^{-1}R$ è un $R$-modulo piatto $\forall S \subseteq R$ moltiplicativamente chiusi
			\end{itemize}
			Per quozienti si può controllare la piattezza sapendo che le seguenti sono equivalenti:
			\begin{itemize}
				\item $a \in a^2A$
				\item $aA$ è sommando diretto di $A$
				\item $\frac{A}{aA}$ è $A$-piatto
			\end{itemize}
		\item ({\bf Implicazioni varie})
			\begin{itemize}
				\item Libero $\implies$ Proiettivo (Il viceversa vale se $A$ è PID oppure anche se $A$ è locale e $P$ f.g.)
				\item Proiettivo $\implies$ Piatto (Viene da Libero $\implies$ Piatto, piuttosto semplice da mostrare usando che $L = \oplus_\ni R^{(\ni)}$ se $L$ è libero ed utilizzando il fatto che un modulo proiettivo è un sommando diretto di un modulo libero). \\ Il viceversa non vale. Ad esempio $\bbQ$ come $\bbZ$-modulo
			\end{itemize}
%		\item ({\bf Lemma del Serpente}) %TODO
%		\item ({\bf Lemma dei cinque}) %TODO
	\end{itemize}
	
	\section*{Moduli Nötheriani ed Artiniani}
	\begin{itemize}
		\item ({\bf Definizione}) Se $(\Sigma, \le)$ è un insieme parzialmente ordinato allora sono equivalenti:
			\begin{itemize}
				\item Ogni catena ascendente è stazionaria
				\item Ogni sottoinsieme diverso dal vuoto ha un elemento massimale
			\end{itemize}
			Sia ora $A$ un anello, $M$ un $A$-modulo e $\Sigma = \{ N \subseteq M \text{ sottomodulo } \}$. Se $(\Sigma, \subseteq)$ soddisfa una delle due condizioni equivalenti di cui sopra, $M$ viene detto $A$-modulo Nötheriano [ACC] \\
			Se invece è $(\Sigma, \supseteq)$ a soddisfare una delle due condizioni, $M$ viene detto $A$-modulo Artiniano [DCC] \\
			Un anello $A$ si dice Artiniano (Nötheriano) se è Artiniano (Nötheriano) come $A$-modulo su sè stesso
		\item ({\bf Condizione equivalente alla Nötherianità}) $M$ è un $A$-modulo Nötheriano $\sse$ ogni sottomodulo è f.g.
		\item ({\bf Passaggio per sequenze esatte}) Sia $0 \rar M \rar N \rar P \rar 0$ una sequenza esatta corta. Allora vale che:
			\begin{itemize}
				\item $N$ Nötheriano $\sse M, P$ Nötheriani
				\item $N$ Artiniano $\sse M, P$ Artiniani
			\end{itemize}
			Come corollari si ottengono i seguenti:
			\begin{itemize}
				\item $M_1, \ldots, M_n$ Nötheriani $\sse \oplus_i M_i$ Nötheriano
				\item $A$ Nötheriano e $M$ $A$-modulo f.g. $\implies M$ Nötheriano
				\item $A$ Nötheriano e $I \subseteq A$ ideale $\implies \frac{A}{I}$ Nötheriano
				\item $f: A \srar B$ surgettiva. Allora $A$ Nötheriano $\implies B$ Nötheriano
				\item $A$ Nötheriano $\implies S^{-1}A$ Nötheriano (per la corrispondenza tra ideali)
				\item Vale inoltre che $A$ Nötheriano $\implies A[x]$ Nötheriano (Base di Hilbert)
			\end{itemize}
		\item ({\bf Lemmi per i Nötheriani}) Valgono le seguenti cose a caso:
			\begin{itemize}
				\item $A$ Nötheriano. Ogni ideale contiene allora una potenza del suo radicale, ovvero $\forall I \subseteq A \quad \exists n \tc (\sqrt{I})^n \subseteq I$
				\item $A$ Nötheriano, $\km$ ideale massimale. Allora TFAE:
					\begin{itemize}
						\item $Q$ è $\km$-primario
						\item $\sqrt{Q} = \km$
						\item $\exists n \quad \km^n \subseteq Q \subseteq \km$
					\end{itemize}
			\end{itemize}
		\item ({\bf Teoremi per gli Artiniani})
			\begin{itemize}
				\item $A$ è Artiniano $\sse A$ è Nötheriano e $\Dim A = 0$ [Non dimostrato]
				\item ({\bf Teorema di Struttura per anelli artiniani}) $A$ è Artiniano $\sse A = \oplus_i A_i$ con gli $A_i$ Artiniani e Locali. La decomposizione è unica a meno di isomorfismi [Non dimostrato]
				\item $A$ Artiniano $\implies A$ semilocale
			\end{itemize}
	\end{itemize}
	
	\section*{Decomposizione Primaria}
	\begin{itemize}
		\item ({\bf Decomposizione primaria di un ideale}) $I \subseteq A$ ideale si dice decomponibile se si può scrivere come intersezione di un numero finito di ideali primari $Q_1, \ldots, Q_n$ come $I = \cap_i Q_i$. (Definiamo inoltre primi associati ad una decomposizione $P_i := \sqrt{Q_i}$)
		\item ({\bf Minimalizzazione di una decomposizione}) Se $P_i = P_j$ in una decomposizione allora vale che $Q_i \cap Q_j$ è ancora primario e posso quindi sostituirlo al posto di $Q_i$ e $Q_j$ (Vale ancora che $\sqrt{Q_i \cap Q_j} = P_i = P_j$. \\
		Una decomposizione si dice minimale o irridondante se $P_i \neq P_j \quad \forall i \neq j$ e $\cap_{i\neq j} Q_j \not\subseteq Q_i$
		\item ({\bf Proposizione tecniche}) $Q$ primario e $P = \sqrt{Q}$, $a \in A$. Allora valgono le seguenti:
			\begin{itemize}
				\item Se $a \in Q$ si ha $(Q : a) = 1$
				\item Se $a \notin Q$ allora $(Q : a)$ è $P$-primario, ovvero $Q:a$ è primario e $\sqrt{Q:a} = P$
				\item Se $a \notin P$ allora $(Q : a) = Q$
			\end{itemize}
			Notare che se dovessi avere un ideale $J$ finitamente generato al posto di $a$, basta ricordare che $(Q : \sum_i J_i) = \cap_i (Q : J_i)$ per ricavarne le relative proposizioni
		\item ({\bf Unicità dei primi associati}) Sia $I = \cap_{i=1}^n Q_i$ con $\sqrt{Q_i} = P_i$ e supponiamo la scrittura minimale. Allora i $P_i$ sono indipendenti dalla decomposizione ed inoltre vale che $\{P_1, \ldots, P_n\} = \{ \sqrt{I:a} \text{ primi} \mid a \in A \}$ (Ovvero $\forall a \in A$ faccio $\sqrt{I:a}$. Se $\sqrt{I:a}$ è primo allora lo prendo.
		\item ({\bf Primi minimali}) Data una decomposizione minimale di $I$, considero i primi associati $P_i$. Tra questi posso considerare i primi minimali per inclusione (detti primi minimali). In particolare i primi minimali associati ad $I$ sono quelli tali che $\forall P$ primo tale che $I \subseteq P$ allora si ha $\exists i$ tale che $P_i \subseteq P$ dove $P_i$ è un primo minimale.
		\item ({\bf Nilradicale}) In particolare se $(0)$ è decomponibile allora $\cN(A) = \cap_{P_i \text{ minimali di } (0)} P_i$
		\item ({\bf Caratterizzazione dell'unione dei primi associati}) $I = \cap_i Q_i$ minimale con $P_i = \sqrt{Q_i}$ allora $\{ a \in A \mid (I:a) \neq I \} = \cup_i P_i$
		\item ({\bf Divisori di Zero}) Se $(0)$ è decomponibile allora si ha $\cD(A) = \cup_{0 \neq a \in A} \sqrt{0 : a}$ e se $(0) = \cap_i Q_i$ allora $\sqrt{0:a} = \cap_{a \notin Q} \sqrt{Q:a} = \cap_{a \notin Q_i} P_i \subseteq P_i$, ovvero $\cD(A) \subseteq \cup_i P_i$ e visto che $P_i = \sqrt{0:a}$ si ha $P_i \subseteq \cD(A)$
		\item ({\bf Decomposizione Primaria con $S^{-1}$}) $S \subseteq A$ e $I = \cap_i Q_i$ minimale. Siano inoltre $P_1, \ldots, P_m, P_{m+1}, \ldots, P_n$ i primi associati ordinati in modo che $S \cap P_i = \emptyset$ con $i \le m$ e che $S \cap P_j \neq \emptyset$ se $j \ge m+1$. \\
		Allora si ha che $S^{-1}I = \cap_i S^{-1}Q_i = \cap_{i \le m} S^{-1}Q_i$ e quindi $(S^{-1}I)^c = Q_1 \cap \ldots \cap Q_m$. "Facendo così abbiamo ucciso le componenti i cui primi intersecano $S$" 
		\item ({\bf Unicità dei primari minimali}) Per il lemma di sopra abbiamo l'unicità dei primari minimali. Infatti visto che $P_i$ è minimale si ha $S = A \cap P_i$ e allora $S \cap P_j \quad \forall j \neq i$ e quindi $Q_i$ non dipende dalla decomposizione perché anche i $P_i$ non dipendono dalla decomposizione.
		\item ({\bf Esistenza della Decomposizione Primaria}) Mostriamo che in un anello Nötheriano ogni ideale è decomponibile, nei seguenti due step:
			\begin{itemize}
				\item Dimostriamo prima che $I \subseteq A$ ideale, con $A$ Nötheriano, allora $I = \cap_i I_i$ dove gli $I_i$ sono ideali irriducibili (ovvero tali che $I_i = J \cap K \implies I = J$ oppure $I = K$)
				\item Ogni irriducibile in un Nötheriano è primario
			\end{itemize}
	\end{itemize}
	
	\clearpage
	\title {Prontuario di cose utili (da ascari)}
	\section*{Operazioni tra ideali}
	$\forall \ka, \kb, \kc, \kd$ ideali di $A$ valgono le seguenti:
	\begin{itemize}
		\item $\ka (\kb + \kd) = \ka \kb + \ka \kd$
		\item $\ka \kb \subseteq \ka \cap \kb$
		\item $(\ka + \kb)(\ka \cap \kb) \subseteq \ka \kb$
		\item $\ka \subseteq (\ka : \kb)$
		\item $(\ka : \kb)\kb \subseteq \ka$
		\item $((\ka : \kb) : \kc) = ((\ka : \kc) : \kb) = (\ka : \kb\kc)$
		\item $(\cap_i \ka_i : \kb) = \cap_i (\ka_i : \kb)$
		\item $(\ka : \sum_i \kb_i) = \cap_i (\ka : \kb_i)$
		\item $\ka \subseteq \sqrt{\ka}$
		\item $\sqrt{\sqrt{\ka}} = \sqrt{\ka}$
		\item $\sqrt{\ka \kb} = \sqrt{\ka \cap \kb} = \sqrt{\ka} \cap \sqrt{\kb}$
		\item $\sqrt{\ka + \kb} = \sqrt{\sqrt{\ka} + \sqrt{\kb}}$
		\item Due ideali $\ka$ e $\kb$ si dicono coprimi se $\ka + \kb = 1$.
		\item $\ka + \kb = 1$, $\ka + \kd = 1$ $\implies \ka + \kb\kd = 1$
		\item $\ka + \kb = 1 \implies \ka\kb = \ka \cap \kb$
	\end{itemize}
	
	\section*{Estensione e contrazione}
	Sia dato un morfismo di anelli $\phi: A \rar B$. Allora si hanno le due operazioni di estensione e contrazione. Indicheremo con $\ka$ gli ideali di $A$ e con $\kb$ ideali di $B$. Allora vale che:
	\begin{itemize}
		\item $\ka \subseteq \ka^{ec}$
		\item $\kb \supseteq \kb^{ce}$
		\item $\ka^{ece} = \ka^e$
		\item $\kb^{cec} = \kb^c$
		\item L'insieme degli ideali contratti e di quelli estesi sono in biggezione tramite le operazioni di estensione e contrazione
		\item $(\ka_1 + \ka_2)^e = \ka_1^e + \ka_2^e$
		\item $(\ka_1 \cap \ka_2)^e \subseteq \ka_1^e \cap \ka_2^e$
		\item $(\ka_1 \ka_2)^e = \ka_1^e \ka_2^e$
		\item $(\ka_1 : \ka_2)^e \subseteq (\ka_1^e : \ka_2^e)$
		\item $(\sqrt{\ka})^e \subseteq \sqrt{\ka^e}$
		\item $(\kb_1 + \kb_2)^{c} \supseteq \kb_1^c + \kb_2^c$
		\item $(\kb_1 \cap \kb_2)^{c} = \kb_1^c \cap \kb_2^c$
		\item $(\kb_1 \kb_2)^{c} \supseteq \kb_1^c \kb_2^c$
		\item $(\kb_1 : \kb_2)^{c} \subseteq (\kb_1^c : \kb_2^c)$
		\item $(\sqrt{\kb})^c = \sqrt{\kb^c}$
		\item Inoltre si ha che se $\kb$ è primo (primario) (radicale) allora $\kb^c$ è primo (primario) (radicale)
		\item Se $\ka$ è principale (finitamente generato) allora $\ka^e$ è principale (finitamente generato)
	\end{itemize}
	
	\section*{$S^{-1}$ e corrispondenze tra ideali}
	\begin{itemize}
		\item $\kb$ radicale (primario) (primo) $\sse$ $\kb^c$ radicale (primario) (primo)
		\item $\kb$ massimale $\sse \kb^c$ massimale tra quelli che non intersecano $S$
		\item $\ka$ primo (primario) (massimale) tale che $\ka \cap S = \emptyset \quad \implies$ $\ka^{ec} = \ka$ ed $\ka^e$ primo (primario) (massimale)
		\item $\kb$ principale (finitamente generato) $\implies \kb^c$ principale (finitamente generato) (Con $A$ dominio vale anche il viceversa)
	\end{itemize}
	Vale inoltre che:
	\begin{itemize}
		\item $(\ka_1 \cap \ka_2)^e = \ka_1^e + \ka_2^e$
		\item $(\sqrt{\ka})^e = \sqrt{\ka^e}$
		\item Se $\ka_2$ è f.g. allora $(\ka_1 : \ka_2)^e = (\ka_1^e : \ka_2^e)$
	\end{itemize}
	Inoltre se $A$ è dominio (UFD) (PID) (Nötheriano) allora $S^{-1}A$ è dominio (UFD) (PID) (Nötheriano)
\end{document}
