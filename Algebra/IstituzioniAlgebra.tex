\documentclass[a4paper,NoNotes,GeneralMath]{stdmdoc}
\usepackage{pgf}
\usepackage{tikz-cd}
\usepackage{xfrac}
\usepackage{mathtools}
\usetikzlibrary{cd,arrows,automata}
\newcommand{\MCD}{\text{MCD }}
\newcommand{\Lt}{\text{lt }}
\newcommand{\Ass}{\text{Ass }}
\newcommand{\cl}{\ell}
\newcommand{\Lc}{\text{lc }}
\newcommand{\pipe}{\,|\,}
\newcommand{\BdG}{\text{BdG }}
\newcommand{\Ris}{\text{Ris }}
\newcommand{\Ht}{\text{ht }}
\newcommand{\Coht}{\text{coht }}
\newcommand{\srar}{\twoheadrightarrow}
\newcommand{\hrar}{\hookrightarrow}
\newcommand{\Hom}{\text{Hom }}
\newcommand{\coKer}{\text{coKer }}
\newcommand{\gen}[1]{\langle {#1} \rangle}
\newcommand{\lbr}[1]{\ensuremath{{\left( {#1} \right)}}}
\newcommand{\intclos}[2]{\ensuremath{\overline{#1}^{#2}}}
\newcommand{\Frac}{\text{Frac }}
\newcommand{\Spec}{\text{Spec }}
\newcommand{\Max}{\text{Max }}
\newcommand{\Gal}{\text{Gal }}
\newcommand{\trdeg}{\text{trdeg }}
\newcommand{\Gr}{\text{Gr }}
\newcommand{\Bl}{\text{Bl }}
\newcommand{\ov}{\overline}

\begin{document}
\title{Istituzioni di Algebra}

\section*{Estensioni Intere}
\begin{itemize}
\item ({\bf Equivalenze di Intero}) Sia $A \subseteq B$ un'estensione di
  anelli e sia $b \in B$. Allora le seguenti sono equivalenti:
  \begin{enumerate}
  \item $b$ è intero su $A$
  \item $A[b]$ (come sottoanello) è un $A$-modulo finito
  \item $\exists C \subseteq B$ sottoanello tale che $A[b] \subseteq
    C$ e $C$ è un $A$-modulo finito
  \item $\exists M$, $A[b]$-modulo fedele (ovvero $\Ann(M) = 0$) che
    sia finito come $A$-modulo.
  \end{enumerate}
\item ({\bf Hamilton-Cayley}) Se $M$ è un $A$-modulo finito e
  $I \subseteq A$ ideale, $\phi: M \rar M$ mappa di $A$-moduli tale che
  $\phi(M) \subseteq IM$, allora $\exists a_1, \ldots, a_n \in I$ tali
  che $\phi^n + a_1 \phi^{n-1} + \ldots + a_n \Id = 0$ (in
  $\Hom_A(M, M)$)
\item Valgono quindi le seguenti cose:
  \begin{enumerate}
  \item Se $b_1, \ldots, b_n$ sono interi su $A$, allora
    $A[b_1, \ldots, b_n]$ è un $A$-modulo finito
  \item $\intclos{A}{B} = \{ b \in B \pipe b \text{ intero su } A \}$ è un
    sottoanello di $B$.
  \item ({\bf Transitività Integrale}) Se $B$ è intero su $A$ e $C$ è
    intero su $B$, allora $C$ è intero su $A$
  \item ({\bf Transitività Finita}) Se $B$ è finito su $A$ e $C$ è
    finito su $B$, allora $C$ è finito su $A$
  \item ({\bf Idempotenza della Chiusura Integrale}) Sia $A \subseteq
    B$ e $C = \intclos{A}{B}$. Allora $\intclos{C}{B} = C$
  \end{enumerate}
\item ({\bf Stabilità per Localizzazione e Quoziente}) Sia
  $A \subseteq B$ intera, $S$ parte moltiplicativa di $A$ e
  $I \subseteq B$ ideale. Allora si ha che:
  \begin{enumerate}
  \item $S^{-1}A \rar S^{-1}B$ è intera
  \item $\sfrac{A}{I^c} \rar \sfrac{B}{I}$ è intera
  \end{enumerate}
\item ({\bf Relazioni con estensioni di campi}) Supponiamo $A$ dominio e
  $K = \Frac(A)$ suo campo delle frazioni e consideriamo
  $A \subseteq K \subseteq L$ dove $\sfrac{L}{K}$ è algebrica. Definiamo
  $B = \intclos{A}{L}$.
  \begin{enumerate}
  \item Sia $x \in L$ e $\mu_x$ suo polinomio minimo su $K$. Se
    $\mu_x \in A[t]$ allora $x$ è intero su $A$.
  \item Se $A$ è normale vale anche il viceversa, ovvero si ha $x$ è
    intero su $A$ $\sse$ $\mu_x \in A[t]$.
  \end{enumerate}
\item ({\bf UFD $\implies$ normale}) Se $A$ è un UFD allora è normale.
\item ({\bf estensioni intere di campi}) Sia $A \subseteq B$
  un'estensione intera di domini. Allora $A$ è un campo $\sse$ $B$ è un
  campo. Ne segue che:
  \begin{enumerate}
  \item Sia $\kq \subseteq B$ un ideale. Allora $\kq$ è massimale $\sse$
    $\kq^c$ è massimale
  \item Se prendo $\kp$ primo di $A$ e $\kq_1, \kq_2 \in \Spec B$ tali
    che $\kq_1^c = \kq_2^c = \kq$ e $\kq_1 \subseteq \kq_2$ allora vale
    che $\kq_1 = \kq_2$.
  \end{enumerate}
\item ({\bf Lying Over}) Se $A \subseteq B$ è intera, allora $\forall
  \kp \in \Spec A \quad \exists \kq \in \Spec B$ tale che $\kq^c = \kp$,
  ovvero $f^*: \Spec B \rar \Spec A$ è surgettiva.
\item ({\bf Going Up}) Se $A \subseteq B$ è intera, allora ha la
  proprietà del going up, ovvero se
  $\forall \kp_1 \subseteq \kp_2 \subseteq A$ primi e
  $\forall \kq_1 \subseteq B$ primo tale che $\kq_1^c = \kp_1$ allora
  $\exists \kq_2 \supseteq \kq_1$ primo tale che $\kq_2^c = \kp_2$
\item ({\bf Going Down}) $A \subseteq B$ estensione intera di domini,
  con $A$ normale allora vale la proprietà del going down, ovvero
  $\forall \kp_1 \subseteq \kp_2 \subseteq A$ primi e
  $\forall \kq_2 \subseteq B$ primo tale che $\kq_2 \cap A = \kp_2$ si
  ha che $\exists \kq_1 \subseteq \kq_2$ primo tale che
  $\kq_1 \cap A = \kp_1$. \newline
  Da notare che serve sia la condizione di dominio che la normalità di
  $A$ per far funzionare tutto ciò.
\item Se $A$ è un dominio normale, $K = \Frac A$ ed $\sfrac{L}{K}$ è
  un'estensione algebrica di campi, $B = \intclos{A}{L}$,
  $I \subseteq A$ ideale, ed $x \in L$ allora $x$ è intero su $I$ $\sse$
  $\mu_{\sfrac{L}{K}, x} \in \sqrt{I}[t]$
\item ({\bf Simil-Galois}) Sia $A$ normale e $K = \Frac A \subseteq L$
  con $\sfrac{L}{K}$ estensione di Galois finita e $B = \intclos{A}{L}$
  e sia $G = \Gal \sfrac{L}{K}$. Allora si ha che:
  \begin{enumerate}
  \item $g(B) \subseteq B \quad \forall g \in G$
  \item Fissato un primo $\kp \in \Spec A$ si ha
    $\cF_\kp = \{ \kq \in \Spec B \pipe \kq \cap A = \kp \}$
    Allora $G$ agisce transitivamente su $\cF_\kp$
  \end{enumerate}
\item ({\bf Chiusura integrale di $A[x]$}) Sia $A \subseteq B$ e sia
  $C = \intclos{A}{B}$. Allora la chiusura integrale di $A[x]$ in $B[x]$
  è $C[x]$.
\item ({\bf Interezza e Nötherianità}) $A \subseteq B$ con $A$
  Nötheriano e $B$ finito su $A$. Allora $B$ è Nötheriano come
  anello. \newline
  Attenzione che $A \subseteq B$ con $A$ Nötheriano ed estensione intera
  NON implica che $B$ sia Nötheriano.
\item Sia $A$ dominio, $K = \Frac A$, $\sfrac{L}{K}$ un'estensione
  finita di campi e sia $B = \intclos{A}{L}$. Allora si ha:
  \begin{enumerate}
  \item È FALSO che se $A$ è Nötheriano allora $B$ lo sia.
  \item Se $A$ è normale ed $\sfrac{L}{K}$ è un'estensione separabile
    allora si ha che se $A$ è Nötheriano allora $B$ diventa un
    $A$-modulo finito e quindi è Nötheriano.
  \end{enumerate}
\item $A$ noetheriano e dominio, con $K = \Frac A \subseteq L$ campi e
  vorremmo poter dire qualcosa anche se $A$ non è normale. Ci sono due
  casi significativi nei quali si ha che in queste ipotesi
  $\intclos{A}{K}$ (la normalizzazione di $A$) è Nötheriana:
  \begin{enumerate}
  \item $\Dim A = 1$
  \item $A$ è una $K$-algebra finitamente generata
  \end{enumerate}
\item ({\bf Normalizzazione di Nöther}) $A$ $K$-algebra f.g. allora
  $\exists x_1, \ldots, x_n \in A$ algebricamente indipendenti tali che
  $A$ è finita (come modulo) su $K[x_1, \ldots, x_n]$
\item ({\bf Lemmi generici e fatti vari}) Le seguenti cose valgono:
  \begin{enumerate}
  \item $E$ campo e sia $A$ una $E$-algebra finitamente generata. Allora
    $\forall I \subseteq A$ si ha che
    $$\sqrt{I} = \bigcap_{\km \text{ massimali } \quad \km \supseteq I} \km$$
  \item $A$ e $B$ due $K$-algebre finitamente generate con $K$ campo
    algebricamente chiuso ed $A$ dominio. Se $f^*: \Max B \rar \Max A$ è
    suriettiva allora $f$ è iniettiva.
  \item Per un modulo sono proprietà locali (e massimali) essere
    piatto, essere normale, essere nullo.
  \item Per un modulo NON sono proprietà locali essere Noetheriano,
    essere dominio.
  \item Per una sequenza di moduli essere esatta in un punto è una
    proprietà locale (e massimale)
  \end{enumerate}
\end{itemize}

\section*{Teoria della Dimensione e Grado di Trascendenza}
\begin{itemize}
\item ({\bf Dimensione in estensioni intere}) Se $A \subseteq B$ è
  intera allora $\Dim A = \Dim B$ (è compreso il caso in cui entrambe le
  dimensione siano infinite)
\item ({\bf Dimensione delle $K$-algebre f.g.}) Sia $K$ un campo e
  $f \in K[x_1, \ldots, x_n]$ con $f \neq 0$. Allora si ha
  $\Dim K[x_1, \ldots, x_n]_f = n$
\item ({\bf Relazione con il grado di trascendenza}) Sia $A$ $K$-algebra
  f.g. e $A$ dominio. Allora $\Dim A = \trdeg_K A$
\item ({\bf Cardinalità di una base di trascendenza}) $A$ dominio e
  $K$-algebra f.g. Allora tutte le basi di trascendenza hanno la stessa
  cardinalità. \newline
  Inoltre, se $A$ è un dominio, $E$ una $K$-algebra e sia $L = \Frac A$
  \begin{enumerate}
  \item $x_i \in A$ è una base di trascendenza di $A$ $\sse$ lo è di $L$
  \item Se $y_1, \ldots, y_n$ è base di trascendenza di $L$ su $K$
    allora $\exists b \in A$ tali che $by_1, \ldots, by_n$ è una base di
    trascendenza di $A$.
  \end{enumerate}
\item ({\bf Particolarità delle $K$-algebre}) Per le $K$-algebre
  f.g. valgono le seguenti cose:
  \begin{enumerate}
  \item Sono anelli catenari.
  \item Se $A$ è dominio, preso un primo $\kp$ di altezza uno si ha
    $\Dim \sfrac{A}{\kp} = \Dim A - 1$
  \item $\kp \in \Spec A \implies $ $\Ht \kp, \Coht p < + \infty$
  \item Se $A$ è dominio si ha $\forall \kp \in \Spec A$ vale che
    $\Dim A = \Ht \kp + \Coht \kp$
  \end{enumerate}
\item ({\bf Artinianità e Nötherianità}) $A$ artiniano se e solo se $A$
  Nötheriano e di dimensione zero.
\item ({\bf Richiami di Decomposizione Primaria}) Sia $I \subseteq A$ un
  ideale. Una decomposizione primaria di $I$ è una scrittura
  $I = \cap_i Q_i$ dove i $Q_i$ sono un numero finito di ideali
  primari. Se $A$ è Nötheriano valgono:
  \begin{enumerate}
  \item Primi associati ad $I$: $\Ass I = \{ \kp \in \Spec A \pipe \exists x
    \in A \tc \kp = (I : x) \}$
  \item Gli zero divisori di $A$ sono l'unione dei primi associati a
    zero: $$\cD(A) = \cup_{\kp \in \Ass 0} \kp$$
  \item $\Ass_{S^{-1}A} S^{-1}I = \Ass_{A} I \cap \Spec S^{-1}A$
  \item $\Ass I$ è finito
  \item Se $\kp$ è minimale sopra $I$, allora $\kp$ è associato ad $I$.
  \end{enumerate}

\item ({\bf Esercizi e Lemmi vari}) Valgono le seguenti cose:
  \begin{enumerate}
  \item $\prod_{n = 0}^\infty \bbZ$ NON è uno $\bbZ$-modulo libero
  \item $\prod_{n = 0}^\infty \rar^f \bbZ$ tale che $f(e_i) = 0 \quad
    \forall i$ allora deve essere che $f = 0$
  \item $X = \Spec A$ e
    $X_f := \{ \kp \subseteq A \pipe f \notin \kp \} = X \setminus V(f)$ è
    un aperto di $X$. Questi sono un sistema fondamentale di aperti di
    $X$ e si ha $X_f \cong \Spec A_f$
  \item $K((t))$ NON è algebrico su $K(t)$
  \item $X = \Spec A$ è compatto, qualunque sia $A$.
  \item $A \subseteq B$ intera $\implies f^*$ chiusa
  \item $f^*$ chiusa $\implies$ vale il Going Up.
  \end{enumerate}
\end{itemize}

\section*{Dimensione e Anelli Graduati}
\begin{itemize}
\item ({\bf Serie di Jordan-Hölder}) $A$ anello c.u., $M$ un
  $A$-modulo. Una serie di Jordan-Hölder per $M$ è una successione
  crescente di sottomoduli
  $0 = M_1 \subsetneq \ldots \subsetneq M_n = M$ tali che
  $\sfrac{M_i}{M_{i-1}}$ è un $A$-modulo semplice (ovvero è diverso dal
  modulo nullo e non ha sottomoduli propri)
\item ({\bf Lunghezza di un Modulo}) Se $M$ ha una serie di JH, diciamo
  che la lunghezza di $M$ è finita e definiamo
  $\cl(M) = \min \{ n \pipe \exists \text{ serie di JH con } n+1 \text{
    termini } \}$
\item ({\bf Lunghezza delle serie di JH}) Tutte le serie di JH di uno
  stesso modulo hanno la stessa lunghezza ed inoltre i fattori
  $\sfrac{M_i}{M_{i-1}}$ sono uguali per ogni serie, a meno di
  permutazioni.
\item ({\bf Comportamento per sequenze esatte}) Sia data una sequenza
  esatta corta di $A$-moduli $0 \rar X \rar Y \rar Z \rar 0$. Allora
  vale che:
  \begin{enumerate}
  \item $\cl(X), \cl(Z) < +\infty \sse \cl(Y) < +\infty$
  \item $\cl(Y) = \cl(X) + \cl(Z)$
  \end{enumerate}
  Inoltre si ha che per un generico modulo $M$ vale $\cl(M) < \infty
  \sse M$ è artiniano e Nötheriano.
\item ({\bf Anelli graduati noetheriani}) $A$ anello graduato, allora
  $A$ è noetheriano se e solo se $A_0$ è noetheriano ed $A$ è f.g. come
  $A_0$-algebra.
\item ({\bf Funzione e Serie di Hilbert}) Preso $A$ graduato, $A_0$
  artiniato, $A$ noetheriano, $M$ graduato e f.g. (ovvero $M$
  noetheriano) definiamo:
  \begin{enumerate}
  \item $n \mapsto \cl_{A_0}(M_n)$ funzione di Hilbert
  \item $P_M(t) := \sum_n t^n \cl_{A_0}(M_n)$ serie di Hilbert
  \end{enumerate}
  Inoltre in queste ipotesi se si ha una sequenza esatta corta con
  morfismi graduati di grado zero:
  $0 \rar X \rar^\phi Y \rar^\psi Z \rar 0$ allora sono definiti i
  polinomi di hilbert e vale che $P_Y = P_X + P_Z$
\item ({\bf Teorema di Hilbert}) $A$ graduato, $M$ graduato, $A_0$
  artiniano e $A$ noetheriano, $M$ f.g. e si chiamino $a_1, \ldots, a_k$
  i generatori omogenei di $A$ come $A_0$-algebra e $d_i := \Deg
  a_i$. Allora $\exists f \in \bbZ[t, t^{-1}]$ polinomio di Laurent tale
  che
  $$P_M(t) = \frac{f(t)}{\prod_{i=1}^k (1 - t^{d_i})}$$
\item ({\bf funzione di Hilbert e grado}) Se $A$ è generato in grado $1$
  allora la funzione di Hilbert (nelle ipotesi precedenti) per $n$
  grandi coincide con i valori assundi da un polinomio di grado $d(M) -
  1$ (dove $d(M)$ è l'ordine di polo in $1$ della funzione razionale
  $P_M$)
\item ({\bf Analogo di Nakyama}) $A$ graduato, $M$ graduato f.g. e
  supponiamo che $A_+ M = M$ allora $M = 0$
% TODO: Mi sono fermato all'inizio del pdf del 19/10
\end{itemize}

\section*{Filtrazioni, Graduato Associato e Scoppiamento}
\begin{itemize}
\item ({\bf Filtrazione}) Sia $A$ un anello e supponiamo di avere
  $A = I_0 \supseteq I_1 \supseteq \ldots \supseteq I_n \supseteq
  \ldots$ (che viene detta una filtrazione $\cF$) con
  $I_k I_h \subseteq I_{h + k}$. Il graduato associato allora è
  $\tilde{A} = \Gr_\cF (A) = \oplus_{k=0}^\infty \sfrac{I_k}{I_{k+1}} =
  \oplus \tilde{A_k}$ e viene indotto un prodotto.

  Se $I$ è un ideale di $A$ possiamo considerare $I_h = I^h$, dove la
  potenza è intesa come moltiplicazione di ideali. Allora si ha
  $\Gr_I (A) = \oplus_{k \ge 0} \sfrac{I^k}{I^{k+1}}$ e lo scoppiamento
  di $A$ in $I$ è $\Bl_I (A) = A \oplus I \oplus I^2 \oplus \ldots$,
  dove si ha $B_k = I^k$ ($B_0 = A$). Allora per definizione il prodotto
  verifica $B_h B_k \subseteq B_{h+k}$
\item ({\bf Noetherianità dei graduati}) Se $A$ è Nötheriano allora
  $\Gr_I (A)$ e $\Bl_I (A)$ sono a loro volta noetheriani e sono
  finitamente generati in grado $1$ come $B_0$-algebra.
\item ({\bf Esempio}) Sia $A = \frac{\bbC[x_1, \ldots, x_n]}{(f)}$ con
  $f$ irriducibile e sia $f = \sum_i f_i$ con $f_i$ omogenei. Sia
  $k = \min \{i \mid f_i \neq 0 \}$. Allora considerando
  $\km = \frac{(x_1, \ldots, x_n)}{(f)}$ che è un massimale di $A$ si ha
  che $\Gr_\km (A) = \frac{\bbC[y_1, \ldots, y_n]}{(f_k)}$
\item ({\bf Filtrazioni di Moduli}) Sia $A$ un anello ed $I$ un ideale
  di $A$, $M$ un $A$-modulo. Allora una $I$-filtrazione $\cF$ di $M$ è
  $$ M = M_0 \supseteq M_1 \supseteq M_2 \supseteq \ldots $$ tale che si
  abbia $I^k M_h \subseteq M_{h + k}$.

  Ad essa posso allora associare un graduato $\Gr_\cF (M) = \oplus_i
  \sfrac{M_i}{M_{i+1}}$ ed uno scoppiamento $\Bl_\cF (M) = \oplus_{i \ge
    0} M_i$ e di $M_i$ si dice che è ``di grado $i$''

  Si nota inoltre che $\Gr_\cF M$ è un $\Gr_I A$-modulo e che $\Bl_\cF
  M$ è un $\Bl_I A$-modulo.
\item ({\bf $I$-stabilità}) Diciamo che una filtrazione $\cF$
  $M_0 \supseteq M_1 \supseteq \ldots$ è $I$-stabile se $\exists k$ tale
  che $M_{n + k} = I^n M_k$.
\item ({\bf Esistenza di una filtrazione $I$-stabile}) Una filtrazione
  $I$-stabile esiste sempre:
  $$ M \supseteq IM \supseteq I^2 M \supseteq \ldots $$
  Ed è inoltre la più piccola $I$-filtrazione.
\item ({\bf Lemma delle filtrazioni $I$-stabili}) Se $\cF$ e $\cF '$
  sono due $I$-filtrazioni $I$-stabili di $M$, date da:
  $$ \cF: M = M_0 \supseteq M_1 \supseteq \ldots  \hfill  \cF': M_0'
  \supseteq M_1' \supseteq \ldots $$ allora $\exists k > 0$ tale che
  $M_n \supseteq M_{n-k}'$ e $M_n' \supseteq M_{n-k}$ $\forall n >> 0$
  (abbastanza grandi)
\item ({\bf Delle $I$-filtrazioni sui Noetheriani}) Sia $A$ noetheriano,
  $M$ un $A$-modulo f.g., $\cF$ una $I$-filtrazione. Allora valgono i
  seguenti:
  \begin{enumerate}
  \item Se $\cF$ è $I$-stabile si ha $\Gr_\cF M$ è Noetheriano, anzi è
    f.g. come $\Gr_I A$-modulo
  \item $\cF$ è $I$-stabile se e solo se $\Bl_\cF M$ è f.g. come $\Bl_I
    A$-modulo.
  \end{enumerate}
\item ({\bf Lemma di Artin-Rees}) $A$ noetheriano, $I$ ideale di $A$,
  $M$ un $A$-modulo f.g., $\cF$ una $I$-filtrazione $I$-stabile
  $$ \cF: M_0 \supseteq M_1 \supseteq \ldots \supseteq M_n \supseteq
  \ldots $$ e sia $N \subseteq M$ sottomodulo.

  Allora $\cF \cap N$ è ancora una filtrazione $I$-stabile.
\item ({\bf ``Limite'' della filtrazione}) Sia $M$ un $A$-modulo f.g.,
  con $A$ noetheriano ed $I$ un ideale di $A$. Allora si ha
  $$ \cap_n I^n M = \{ x \in M \mid \exists y \in I \tc (1 - y)x = 0
  \} $$

  Come corollari otteniamo immediatamente che se $A$ è locale e
  noetheriano allora si ha $\cap_n \km^n = 0$.

  Se invece $A$ è un dominio noetheriano ed $I$ è un suo qualunque
  ideale allora $\cap_n I^n = 0$.
\item ({\bf Passaggio di dominio}) Sia $A$ noetheriano. Se $\Gr_\km A$ è
  un dominio allora $A$ è un dominio.
\item ({\bf Rapporto tra un po' di cose}) Sia $(A, \km)$ un anello
  locale e noetheriano, $I$ un ideale $\km$-primario e $M$ un $A$-modulo
  f.g. e $\cF$ una $I$-filtrazione stabile di $M$.

  Siano $\tilde{A} = \Gr_I A$ e $\tilde{M} = \Gr_\cF M$. Risulta allora
  definita una serie di Hilbert, ed un polinomio di Hilbert:
  $$ P(\tilde{M}, t) = \sum_n t^n \cl_{A_0} (\tilde{M_n}) =
  \frac{f(t)}{(1-t)^d} $$ e si ha $\phi_{\tilde{M}} (n) = \cl_{A_0}
  (\tilde{M_n})$ ha grado $d - 1$.

  Infine indichiamo con $S_I$ il minimo numero di generatori di $I$. In
  particolare $\tilde{A}$ come $\tilde{A_0}$-algebra è generata da al
  più $S_I$ generatori. Vale allora il seguente lemma:
  \begin{enumerate}
  \item $d \le S_I$
  \item $\cl_A (\sfrac{M}{M_n}) < \infty$
  \item Per $n >> 0$ vale che $\cl_A (\sfrac{M}{M_n})$ è un polinomio di
    grado $d$
  \item Il grado ed il coefficiente direttivo del polinomio NON
    dipendono dalla filtrazione scelta
  \item Inoltre anche il grado di $n \mapsto \cl_A (\sfrac{M}{I^n M})$
    non dipende da $I$.
  \end{enumerate}
\item ({\bf Equivalenza tra le dimensioni}) Sia $(A, \km)$ noetheriano
  locale e consideriamo $M = A$. Allora abbiamo che la dimensione di
  Krull di $A$, il minimo numero di generatori di un ideale
  $\km$-primario ed il grado del polinomio di Hilbert di $A$ come
  $A$-modulo coincidono.
\item ({\bf Corollario}) Per anelli $A$ noetheriani vale che:
  \begin{enumerate}
  \item Se $A$ è locale, $\Dim A < \infty$
  \item $\forall \kp$ primo si ha $\Ht \kp = \Dim A_\kp < \infty$
  \item $\{\kp\}_{\kp \subseteq A}$ soddisfa la condizione della catena
    discendente
  \item Se $A$ è locale si ha $\Dim A \le \Dim_\bbK \sfrac{\km}{\km^2}$
    con $\bbK = \sfrac{A}{\km}$
  \end{enumerate}
\item ({\bf Teorema dell'ideale principale di Krull}) Se $A$ è
  noetheriano ed $x \in A$ un non divisore di zero e $\kp$ è un primo
  minimale tra quelli contenenti $x$. Allora vale che $\Ht \kp = 1$.

  Una semplice generalizzazione dice che se $A$ è noetheriano e $x_1,
  \ldots, x_k \in A$ e si ha $\kp \supseteq (x_1, \ldots, x_k)$ minimale
  tra gli ideali che li contengono tutti allora vale che $\Ht \kp \le k$

  Ancora, se $A$ è locale e noetheriano e $x \notin \cD (A)$ vale che
  $\Dim \frac{A}{(x)} = \Dim A - 1$
\item ({\bf Anelli regolari}) Sia $(A, \km)$ un anello locale e
  noetheriano. $\bbK = \frac{A}{\km}$. Allora $A$ si dice regolare se
  $\Dim A = \Dim_\bbK \frac{\km}{\km^2}$, ovvero se $\km$ è generato
  come ideale da $\Dim A$ elementi.

  Più in generale un anello $A$ si dice regolare se è regolare $A_\km$
  $\forall \km$ massimale (è equivalente a chiederlo $\forall \kp$ con
  $\kp$ primo, ma è piuttosto difficile da mostrare)
\item ({\bf Teorema dello Jacobiano}) Sia $A$ una $\bbK$-algebra, $A =
  \lbr{\frac{\bbK[x_1, \ldots, x_n]}{I}}_\km$ con $I = (f_1, \ldots,
  f_k)$ e tale che $I \subseteq (x_1, \ldots, x_n) = \km$.

  Facciamo allora la matrice Jacobiana dei polinomi in zero
  $J = \lbr{\dpar{f_i}{x_j}\mid_0}_{\substack{i = 1, \ldots, k \\ j = 1,
    \ldots, n}}$ e supponiamo $d = \Dim A$ e $r = \Rk J$.

  Allora $A$ è regolare se e solo se $n = d + r$ e vale inoltre che
  $I_\km = (f_1, \ldots, f_r)$ a meno di riordinamenti (ovvero è
  generato da un qualunque insieme i cui gradienti siano linearmente
  indipendenti)
\item ({\bf Regolarità per locali noetheriani}) Sia $(A, \km)$ un anello
  locale noetheriano e $\bbK = \frac{A}{\km}$ e $\Dim A = d$. Allora $A$
  è regolare se e solo se $\Gr_\km A \cong \bbK[t_1, \ldots, t_d]$, dove
  l'isomorfismo è da intendersi come anelli graduati.

  Come corollario si ottiene che un anello locale noetheriano e regolare
  è un dominio.
\end{itemize}

\section*{Dimensioni basse}
\begin{itemize}
\item ({\bf Discrete Valuation Ring}) Sia $K$ un campo e si abbia
  $v: K \setminus \{ 0 \} \rar \bbQ$ una funzione ``di valutazione'',
  ovvero che soddisfi le seguenti proprietà:
  \begin{enumerate}
  \item $v(xy) = v(x) + v(y)$
  \item $v(x + y) \ge \min \{ v(x), v(y) \}$
  \end{enumerate}

  Un dominio $A$ si dice ``di valutazione discreta'' se $\exists v:
  \Frac(A) \setminus \{ 0 \} \rar \bbQ$ con $\Im v = q \bbZ$, con $q \in
  \bbQ^*$ e $A = A_v = \{ x \in K \mid v(x) \ge 0 \} \cup \{ 0 \}$
\item ({\bf Esempio}) Se $A$ è UFD, $p$ un elemento primo allora si ha
  $v_p (x) = \max {n \mid \quad p^n | x }$ ha le due proprietà di sopra
  e si può estendere a tutto $K = \Frac A$ nel seguente modo:
  $$ v_p (\frac{x}{y}) = v_p(x) - v_p(y) $$.

  In questo caso $A$ è un DVR con la valutazione data da $v_p$.
\item ({\bf Proprietà dei DVR}) Se $A$ è DVR allora si ha:
  \begin{enumerate}
  \item $A$ è Noetheriano
  \item $A$ è PID
  \item $A$ è locale con $\km = \{ x \mid v(x) > 0 \}$
  \item $\km = (t)$
  \item $\Dim A = 1$
  \item $A$ è regolare
  \item $A$ è UFD ed anche Normale
  \end{enumerate}
\item ({\bf Lemma di equivalenze}) Sia $A$ dominio noetheriano normale,
  $\Dim A = 1$. Allora le seguenti sono equivalenti:
  \begin{enumerate}
  \item $A$ è regolare
  \item $A$ è normale
  \item $A$ è DVR
  \item $A$ è UFD
  \end{enumerate}
\item ({\bf Dominio di Dedekind}) Sia $A$ dominio noetheriano di
  dimensione uno. $A$ si dice dominio di Dedekind se è normale, ovvero
  se la localizzazione in tutti gli ideali massimali è normale, ovvero
  $A_\km$ è DVR $\forall \km$ massimale.
\item Se $A$ è un dominio Noetheriano normale, e $\kp \subseteq A$ è un
  ideale primo di altezza uno ($\Ht \kp = 1$) allora si ha che $A_\kp$ è
  regolare.
\item ({\bf Regolare in codimensione uno}) $A$ dominio noetheriano si
  dice regolare in codimensione uno se $\forall \kp \subseteq A$ primo
  di altezza uno si ha che $A_\kp$ è regolare.
\item ({\bf Cosa non dimostrata}) Sia $A$ una $\bbK$-algebra dominio e
  $A = \frac{\bbK[x_1, \ldots, x_n]}{(f_1, \ldots, f_k)}$ e sia $\kp =
  (f_1, \ldots, f_k)$, $d = \Dim A$, $\bbK$ algebricamente chiuso e $r =
  n - d$. Allora vale che:
  \begin{enumerate}
  \item Tutti i determinanti dei minori $(r + 1) \times (r + 1)$ della
    matrice Jacobiana $Jac = \lbr{\dpar{f_i}{x_j}}$ sono elementi di
    $\kp$
  \item Sia $J$ l'ideale generato dai determinanti dei minori
    $r \times r$. Allora, preso $\ov{\kq}$ primo di $A$ con $\ov{\kq} =
    \sfrac{\kq}{\kp}$ si ha che $A_{\ov\kq}$ NON è regolare se e solo se
    $\kq \supseteq J$

    Se pongo $\ov{J} = \frac{J + \kp}{\kp} \subseteq A$ allora $\ov\kq$
    NON è regolare se e solo se $\ov\kq \supseteq \ov{J}$, ovvero
    $\ov\kq \in V(\ov{J}) \subseteq \Spec A$, o ancora $v(\ov{\kq})
    \subseteq v(\ov{J}) \subseteq \bbK^n$
  \end{enumerate}
\item ({\bf Regolarità in codimensione per le $K$-algebre}) Se $A$ è una
  $\bbK$-algebra allora essa è regolare in codimensione uno se si ha
  $\Dim \sfrac{A}{\ov{J}} \le \Dim A - 2$ (Con le notazioni precedenti)
\item ({\bf Criterio di Serre}) Sia $A$ dominio noetheriano. Allora $A$
  è normale se e solo se vale uno dei seguenti:
  \begin{enumerate}
  \item $\forall x \neq 0$ e $\forall \kp \in \Ass (x)$ si ha $\kp_\kp$
    è principale ovvero $A_\kp$ è regolare
  \item $\forall x \neq 0$ e $\forall \kp \in \Ass(x)$ si ha
    $\Ht \kp = 1$. Inoltre vale che $\forall \kp$ tale che $\Ht \kp = 1$
    si ha $A_\kp$ è regolare.
  \end{enumerate}
\item ({\bf Caratterizzazione di UFD}) $A$ dominio noetheriano. Allora
  $A$ è UFD se e soltanto se vale una delle seguenti:
  \begin{enumerate}
  \item $\forall x \neq 0 \quad \forall \kp \in \Ass (x)$ si ha $\kp$ è
    principale
  \item $\forall x \neq 0 \quad \forall \kp \supseteq (x)$ minimale tra
    quelli che contengono $(x)$ vale che $\kp$ è principale.
  \end{enumerate}
\item $A$ dominio noetheriano e sia $\kq$ un ideale $\kp$-primario e $S
  \subseteq A$ una parte moltiplicativa tale che $S \cap \kp =
  \emptyset$. Allora valgono che:
  \begin{enumerate}
  \item $S^{-1}\kq$ è $S^{-1}\kp$-primario
  \item $S^{-1}A \cap S^{-1}\kq = \kq$
  \item $(S^{-1}\kp)^n = S^{-1} \kp^n$
  \item $S^{-1} \kp^n \cap A = \kp^n$
  \end{enumerate}
\item $N \subseteq M$. Allora $x \in N \sse x \in N_\kp \quad \forall
  \kp$ primo.
\item $A$ dominio noetheriano e $I \subseteq A$ ideale allora si ha
  $x \in I \sse x \in I_\kp = I A_\kp \quad \forall \kp$ primo associato
  ad $I$.
\item $A$ dominio noetheriano e $K = \Frac(A)$. Allora dato un $x \in K$
  si ha che $x \in A \sse x \in A_\kp \quad \forall \kp \in \cA$, dove
  $\cA = \{ \kp \subseteq A \text{ primi } \mid \exists y \neq 0 \tc \kp
  \in \Ass (y) \}$
\item ({\bf Teorema di Hartogs}) Sia $A$ dominio noetheriano
  normale. Allora si ha $A = \cap_{\kp \tc \Ht \kp = 1} A_\kp$
\end{itemize}

\section*{Fibrati vettoriali e moduli localmente liberi}
In questa sezione assumeremo l'ipotesi di $A$ dominio noetheriano

\begin{itemize}
\item ({\bf Modulo localmente libero}) $M$ $A$-modulo f.g. si dice
  localmente libero se $\forall \kp \subseteq A$ primo $M_\kp$ è libero
  ($M_\kp \cong A_\kp^r$)
\item ({\bf Caratterizzazione}) Se $M$ è un $A$-modulo f.g. allora sono
  equivalenti:
  \begin{enumerate}
  \item $M$ è localmente libero di rango $r$
  \item $\exists f_1, \ldots, f_k \in A$ tali che $(f_1, \ldots, f_k) =
    A$ e $M_{f_i} \cong A_{f_i}^r$
  \item $M$ è proiettivo e $M \otimes_A K(A) \cong K(A)^r$
  \end{enumerate}

  Osserviamo che se $M$ è localmente libero e si ha
  $M_\kp \cong A_\kp^r$ allora
  $M \otimes_A K = S^{-1}M = M \otimes_{A_\kp} K \cong A_\kp^r
  \otimes_{A_\kp} K \cong K^r$ e quindi se in un primo ha rango $r$
  allora è uguale per tutti poiché
  $r = \Dim_K K^r \cong \Dim_K (M \otimes_A K)$ e viene detto rango.
\item ({\bf Fibrato lineare}) Un fibrato lineare è un modulo localmente
  libero di rango uno.
\item ({\bf Lemma di isomorfismo}) $A$ noetheriano, $S$ parte
  moltiplicativa e $M$ f.g. allora si ha
  $$ S^{-1} \Hom_A (M, N) \rar \Hom_{S^{-1}A} (S^{-1}M, S^{-1}N) $$
  definita da
  $\frac{\phi}{s} \mapsto \frac{\phi}{s} (\frac{m}{t}) =
  \frac{\phi(m)}{st}$ è un isomorfismo.
\item $M$ un $A$-modulo, $M^* = \Hom_A (M, A)$ ed osservo che
  $\mu: M^* \otimes_A M \rar A$ definita da $\phi \otimes m \mapsto
  \phi(m)$ è una mappa. Allora si trova che $M$ è f.g. e localmente
  libero di rango uno $\sse$ $\mu$ è un isomorfismo

  Come corollario, se $M$ è localmente libero di rango uno allora $M^*$
  è f.g. ed è anch'esso un modulo localmente libero di rango uno.
\item $M, N$ moduli localmente liberi di rango uno. Allora si ha:
  \begin{enumerate}
  \item $M \otimes N$ è localmente libero di rango uno e vale che
    $(M \otimes N)^* \cong M^* \otimes N^*$
  \item $M \otimes N \cong A \implies N \cong M^*$
  \item $\Hom_A (M, N) \cong M^* \otimes_A N$
  \end{enumerate}
\item ({\bf Ideali frazionari}) $A$ dominio noetheriano e $K = \Frac
  A$. Un ideale frazionario è un $A$-sottomodulo f.g. di $K$.

  Se $I$ è frazionario allora
  $I^{-1} = \{ x \in K \mid xI \subseteq A \}$ e dico che $I$ è
  invertibile se $I^{-1} I = A$
\item ({\bf Proprietà base}) Per gli ideali frazionari valgono le
  seguenti proprietà:
  \begin{enumerate}
  \item Se $I$ e $J$ sono invertibili allora $IJ$ è invertibile
  \item Se $I$ è invertibile allora $I^{-1}$ è invertibile
  \end{enumerate}
\item ({\bf Divisori di Cartier}) Possiamo introdurre il gruppo delle
  classi dei divisori di Cartier:
  $$ \Cacl (A) = \sfrac{\{\text{ ideali invertibili }\}, \cdot}{~} $$
  dove la relazione è $I ~ J$ se e solo se $I = aJ$ con $a \in K^*$
\item ({\bf Lemma}) $A$ dominio noetheriano. Allora si ha
  \begin{enumerate}
  \item $I$ e $J$ ideali frazionari che siano anche moduli localmente
    liberi di rango uno $I \otimes_A J \cong IJ$ e
    $\Hom_A(I, J) \cong I^{-1}J$, inoltre $I^{-1} \cong I^*$
  \item Se $I$ è un ideale frazionario, allora $I$ è un modulo
    localmente libero di rango uno se e solo se $I$ è invertibile
  \item $\forall M$ modulo localmente libero di rango uno $\exists I$
    ideale invertibile tale che $I \cong M$
  \end{enumerate}
\item ({\bf Teorema interessante}) $A$ dominio noetheriano allora $\Cacl
  (A) \cong \Pic (A)$ attraverso la mappa $[I] \mapsto \{\text{ classe
    di isomorfismo di } I \text{ come } A \text{ modulo }\}$
\item ({\bf Divisori di Weil}) Definiamo
  $\Div A = \oplus_{\Ht \kp = 1} \bbZ \kp$ su $A$ dominio noetheriano e
  $\forall \kp$ tale che $\Ht \kp = 1$ assumiamo $A_\kp$ regolare (ad
  esempio $A$ normale va bene). In queste ipotesi $A_\kp$ è DVR e
  $v_\kp: K^* \rar \bbZ$ in modo che $A_\kp = K_{v_\kp}$.

  Allora dato $f \in K^*$ posso definire
  $$ \div f = \sum_\substack{\kp \text{ primi } \\ \Ht \kp = 1} v_\kp(f)
  \kp $$ Si verifica che la somma è sempre finita e quindi si può
  definire $\Cl A = \sfrac{\Div A}{\{ \div f \mid f \in K^*\}}$ ovvero
  tutti i divisori quozientati per quelli principali.
\item ({\bf Cose vere}) $A$ normale. Allora $A$ è UFD se e solo se
  $\Cl A = 1$

  Con qualche ipotesi si ha $\Cl (A) = \Pic (A)$
\end{itemize}

\section*{Moduli semplici e semisemplici}
In questa sezione $A$ è un anello unitario (non necessariamente
commutativo)

\begin{itemize}
\item ({\bf Moduli semplici}) Sia $M$ un $A$-modulo, $M \neq 0$, allora
  $M$ si dice semisemplice se non ha sottomoduli non banali.
\item ({\bf Lemma di Schur}) Se $S$ e $T$ sono moduli semplici e
  $\phi: S \rar T$ è un morfismo, allora $\phi = 0$ oppure $\phi$ è un
  isomorfismo

  In particolare $\End_A (M) = D$ se $M$ è semplice allora $D$ è un
  corpo.
\item ({\bf Moduli semisemplici}) Le seguenti sono equivalenti:
  \begin{enumerate}
  \item $M$ è somma di moduli semplici (anche NON diretta, anche
    infinita)
  \item $M = \oplus_j M_j$ con gli $M_j$ semplici
  \item $\forall N \subseteq M$ si ha $\exists P \subseteq M$ tale che
    $N \oplus P = M$
  \end{enumerate}
  In questi casi $M$ si dice semisemplice.
\item ({\bf Proprietà dei semisemplici}) Se $M$ è semisemplice e $N
  \subseteq M$ allora anche $N$ è semisemplice.

  Inoltre se $M$ è semisemplice, $N \subseteq M$ si ha $\sfrac{M}{N}$ è
  semisemplice.

  Se invece $N \subseteq M$ è s.s. e $\sfrac{M}{N}$ è s.s. allora NON è
  detto che $M$ lo sia.
\item ({\bf Tutti gli $A$-moduli s.s.?}) Le seguenti sono equivalenti:
  \begin{enumerate}
  \item Tutti gli $A$-moduli sono s.s.
  \item $\prescript{}{A}{A}$ è s.s. (ovvero $A$ come $A$-modulo sinistro
    è s.s.)
  \item Tutti gli $A$-moduli sono proiettivi
  \end{enumerate}
\item ({\bf Esempi}) Sia $D$ un corpo.

  Allora $A = \kM_{n\times n}(D)$ è s.s.

  $D^n$ è un $A$-modulo semplice.
\item ({\bf Teorema di Wedderburn}) Sia $A$ anello s.s. Allora è
  prodotto diretto di anelli di matrici su corpi, ovvero vale che
  $$ A \cong \prod_i \kM_{a_i \times a_i}(D_i') $$
  con $D_i' = \End_A (S_i, S_i)^{\text{op}}$ dove $S_i$ sono i moduli
  semplici tali che $A = \oplus_i S_i$ (che sono in numero finito)
\end{itemize}

\section*{Categorie Abeliane}
\begin{itemize}
\item ({\bf Categoria}) Una categoria $\cC$ è una classe di oggetti
  $\Ob \cC$ e $\forall x, y \in \Ob \cC$ si ha un insieme
  $\Hom_\cC (x, y) = \cC(x, y)$ in modo che si possano comporre ed
  esista l'identità di ogni oggetto.
\item ({\bf Categoria additiva}) È una categoria $\cA$ in cui valgono:
  \begin{enumerate}
  \item[A1] $\Hom(x, y)$ è un gruppo abeliano e $\circ$ è bilineare
  \item[A2] $\exists o \in \cA$ tale che $\Hom (o, o) = 0$ e quindi
    anche $\Hom (o, x) = 0 \quad \forall x \in \cA$
  \item[A3] Esistono il prodotto ed il coprodotto finito.

    Questo assioma si può in realtà rimpiazzare con la sola esistenza
    del prodotto. In questo caso si può dimostrare che il coprodotto
    (assumendo A1 e A2) è isomorfo al prodotto.
  \end{enumerate}
\item ({\bf })
\end{itemize}
\end{document}

