\usepackage{bm, bbm}
\usepackage{xifthen, xparse}
\usepackage{amsmath, amssymb, amsthm}
\usepackage{ragged2e}
\usepackage{color}
\usepackage{array}
\usepackage{tcolorbox}
\usepackage{thmtools, thm-restate}
\usepackage{enumitem}
\usepackage{titlesec}

% Definiamo un environment per i sistemi di equazioni
\newenvironment{eqsystem}[1]
{
  \begin{displaymath}
    \left\{
      \begin{array}{#1}
}{
      \end{array}
    \right.
  \end{displaymath}
}

% Comandi vari e particolarmente utili
\newcommand{\dpar}[2]{\ensuremath{\frac{\partial {#1}}{\partial {#2}}}}
\newcommand{\ddpar}[2]{\ensuremath{\frac{\partial^2 {#1}}{\partial {#2}^2}}}
\newcommand{\de}{\ensuremath{\text{d}}}
\newcommand{\matr}[4]{{\ensuremath{\lrt{\begin{array}{cc} {#1} & {#2} \\ {#3} & {#4} \\ \end{array}}}}}

% Abbreviazioni per i comandi più usati classici
\newcommand{\sbsq}{\ensuremath{\subseteq}}
\newcommand{\spsq}{\ensuremath{\supseteq}}
\newcommand{\sbs}{\ensuremath{\subset}}
\newcommand{\sps}{\ensuremath{\supset}}
\newcommand{\lar}{\ensuremath{\leftarrow}}
\newcommand{\rar}{\ensuremath{\rightarrow}}
\newcommand{\rhu}{\ensuremath{\rightharpoonup}}
\newcommand{\Lar}{\ensuremath{\Leftarrow}}
\newcommand{\Rar}{\ensuremath{\Rightarrow}}
\newcommand{\disc}{\ensuremath{\text{disc }}}
\newcommand{\sse}{\ensuremath{\Leftrightarrow}}
\newcommand{\Gal}{\ensuremath{\text{Gal}}}
\newcommand{\ord}{\ensuremath{\text{ord}}}
\newcommand{\Span}{\ensuremath{\text{Span}}}
\newcommand{\argmin}{\ensuremath{\text{argmin}}}
\newcommand{\Seq}{\ensuremath{\text{Seq}}}
\newcommand{\tc}{\ensuremath{\text{t.c.}}}
\newcommand{\SCI}{\ensuremath{\text{SCI}}}

\newcommand{\halfpi}{\ensuremath{\frac{\pi}{2}}}

\newcommand{\lrg}[1]{\ensuremath{\left\{{#1}\right\}}}
\newcommand{\lrq}[1]{\ensuremath{\left[{#1}\right]}}
\newcommand{\lrt}[1]{\ensuremath{\left({#1}\right)}}
\newcommand{\lra}[1]{\ensuremath{\langle{#1}\rangle}}
\let\scal\lra
\newcommand{\abs}[1]{\ensuremath{\left|{#1}\right|}}
\newcommand{\norm}[1]{\ensuremath{\left|\left|{#1}\right|\right|}}
\newcommand{\ceil}[1]{\ensuremath{\left\lceil{#1}\right\rceil}}
\newcommand{\floor}[1]{\ensuremath{\left\lfloor{#1}\right\rfloor}}

\newcommand{\zu}{\lrg{0,1}}
\newcommand{\grad}{\ensuremath{\nabla}}
\let\bar\overline

%mathbb mathcal mathfrak e mathbm per le lettere dell'alfabeto e anche mathbb per quelle greche
\def\mydeflett#1{\expandafter\def\csname bb#1\endcsname{\mathbb{#1}}
		\expandafter\def\csname c#1\endcsname{\mathcal{#1}}
		\expandafter\def\csname k#1\endcsname{\mathfrak{#1}}
		\expandafter\def\csname bl#1\endcsname{\mathbf{#1}}}
\def\mydefalllett#1{\ifx#1\mydefalllett\else\mydeflett#1\expandafter\mydefalllett\fi}
\mydefalllett ABCDEFGHIJKLMNOPQRSTUVWXYZ\mydefalllett

\def\mydeffrakmath#1{\expandafter\def\csname k#1\endcsname{\mathfrak{#1}}}
\def\mydefallfrak#1{\ifx#1\mydefallfrak\else\mydeffrakmath#1\expandafter\mydefallfrak\fi}
\mydefallfrak abcdefghijklmnopqrstuvwxyz\mydefallfrak

\def\mydefgreek#1{\expandafter\def\csname bl#1\endcsname{\text{\boldmath$\mathbf{\csname #1\endcsname}$}}}
\def\mydefallgreek#1{\ifx\mydefallgreek#1\else\mydefgreek{#1}%
   \lowercase{\mydefgreek{#1}}\expandafter\mydefallgreek\fi}
\mydefallgreek {Gamma}{Delta}{Theta}{Lambda}{Xi}{Pi}{Sigma}{Upsilon}{Phi}{Varphi}{Psi}{Omega}{alpha}{beta}{gamma}{delta}{epsilon}{varepsilon}{zeta}{eta}{theta}{iota}{kappa}{lambda}{mu}{nu}{xi}{omicron}{pi}{rho}{sigma}{tau}{upsilon}{phi}{varphi}{chi}{psi}{omega}\mydefallgreek

\def\defcomplexityclass#1{\expandafter\def\csname #1\endcsname{\text{\boldmath$\mathbf{#1}$}}}
\def\defallcc#1{\ifx\defallcc#1\else\defcomplexityclass{#1}%
  {\defcomplexityclass{#1}}\expandafter\defallcc\fi}
\defallcc {BPP}{EXP}{DTIME}{NTIME}{BPTIME}{SUBEXP}{P}{NP}{ZPP}{RP}{AM}{MA}{IP}{PSPACE}{L}{NL}{E}{NE}{coNE}{NL}{coNL}{coNP}{coRP}{DSPACE}{coNTIME}\defallcc

\newcounter{TheoremCounter}
\newcounter{DefinitionCounter}
\newcounter{ExampleCounter}
\newcounter{EquationCounter}
\newcounter{NotationCounter}
\newcounter{RemarkCounter}

\newtheoremstyle{nostrostile} % <name>
{\baselineskip} % <Space above>
{\baselineskip} % <Space below>
{} % <Body font>
{} % <Indent amount>
{\bf\scshape} % <Theorem head font>
{:} % <Punctation after theorem head>
{1em} % <Space after theorem head>
{} % <Theorem head spec>

\declaretheoremstyle[
spaceabove=3pt, spacebelow=3pt,
headfont=\normalfont\bf\scshape,
notefont=\mdseries, notebraces={(}{)},
bodyfont=\normalfont
]{genericlittle}

\theoremstyle{nostrostile}
\newtheorem{definition}[DefinitionCounter]{Definizione}
\newtheorem{definizione}[DefinitionCounter]{Definizione}
\newtheorem{criterio}[TheoremCounter]{Criterio}
\newtheorem{corollario}[TheoremCounter]{Corollario}
\newtheorem{example}[ExampleCounter]{Esempio}
\newtheorem{esempio}[ExampleCounter]{Esempio}
\newtheorem{theorem}[TheoremCounter]{Teorema}
\newtheorem{teorema}[TheoremCounter]{Teorema}
\newtheorem{lemma}[TheoremCounter]{Lemma}
\newtheorem{proposizione}[TheoremCounter]{Proposizione}
\newtheorem{proposition}[TheoremCounter]{Proposizione}
\newtheorem{notation}[NotationCounter]{Notazione}
\newtheorem{remark}[RemarkCounter]{Osservazione}
\newtheorem{fatto}[RemarkCounter]{Fatto}
\let\proof\undefined
\declaretheorem[style=genericlittle,numbered=no,prefoothook=\qed,title=Dimostrazione]{proof}

\interfootnotelinepenalty=10000

\newcolumntype{L}{>{$}l<{$}} % math-mode version of "l" column type
\newcolumntype{C}{>{$}c<{$}}
\newcolumntype{R}{>{$}r<{$}}

\newcommand{\TODO}{\textcolor{red}{\bf{ANCORA DA SCRIVERE / FINIRE}}}
\newcommand{\Optional}{\textcolor{blue}{[Opzionale]}}

\newenvironment{question}{%
  \begin{tcolorbox}[colback=red!5!white,colframe=red!75!black,title=Domanda]
}{%
  \end{tcolorbox}
}

