\usepackage{xifthen}
\usepackage[linktoc=all]{hyperref}
\usepackage{color}
\usepackage{xparse}
\usepackage{etoolbox}
\usepackage{fancyhdr}

% Definiamo i font da utilizzare per le varie sezioni
\renewcommand*{\familydefault}{\rmdefault}
\renewcommand*{\rmdefault}{ppl}
\renewcommand*{\sfdefault}{cmss}
\renewcommand*{\ttdefault}{lmtt}

% Oggetti utili per dare spaziatura al corpo del testo e per aggiungere note a margine
\newcommand{\Nextblock}{{\vskip 1.5ex}\noindent}
\newcommand{\paragrafo}[1]{{\vskip 3ex}{\normalfont\large\bf\noindent{#1}}{\hskip 3ex}}
\newcommand{\Nota}[1]{\marginpar{\footnotesize{\vskip 4ex}#1}}
\newcommand{\Passo}[1]{\paragrafo{Passo {#1}}}
\newcommand{\Freccia}[1]{\paragrafo{Freccia {#1}}}

\NewDocumentCommand{\Altro}{g}{
  \IfNoValueTF{#1}
	{\paragrafo{}}
	{\paragrafo{#1}}}

\newcommand{\frdx}{ \framebox[\width]{ $\Rightarrow$ } }
\newcommand{\frsx}{ \framebox[\width]{ $\Leftarrow$ } }
\newcommand{\opp}{\text{ oppure }}
\def\checkmark{\tikz\fill[scale=0.4](0,.35) -- (.25,0) -- (1,.7) -- (.25,.15) -- cycle;} 
\newcommand{\crossmark}{$\times$}


%mathbb mathcal mathfrak e mathbm per le lettere dell'alfabeto e anche mathbb per quelle greche
\def\mydeflett#1{\expandafter\def\csname bb#1\endcsname{\mathbb{#1}}
		\expandafter\def\csname c#1\endcsname{\mathcal{#1}}
		\expandafter\def\csname k#1\endcsname{\mathfrak{#1}}
		\expandafter\def\csname bl#1\endcsname{\mathbf{#1}}}
\def\mydefalllett#1{\ifx#1\mydefalllett\else\mydeflett#1\expandafter\mydefalllett\fi}
\mydefalllett ABCDEFGHIJKLMNOPQRSTUVWXYZ\mydefalllett

\def\mydeffrakmath#1{\expandafter\def\csname k#1\endcsname{\mathfrak{#1}}}
\def\mydefallfrak#1{\ifx#1\mydefallfrak\else\mydeffrakmath#1\expandafter\mydefallfrak\fi}
\mydefallfrak abcdefghijklmnopqrstuvwxyz\mydefallfrak

\def\mydefgreek#1{\expandafter\def\csname bl#1\endcsname{\text{\boldmath$\mathbf{\csname #1\endcsname}$}}}
\def\mydefallgreek#1{\ifx\mydefallgreek#1\else\mydefgreek{#1}%
   \lowercase{\mydefgreek{#1}}\expandafter\mydefallgreek\fi}
\mydefallgreek {Gamma}{Delta}{Theta}{Lambda}{Xi}{Pi}{Sigma}{Upsilon}{Phi}{Varphi}{Psi}{Omega}{alpha}{beta}{gamma}{delta}{epsilon}{varepsilon}{zeta}{eta}{theta}{iota}{kappa}{lambda}{mu}{nu}{xi}{omicron}{pi}{rho}{sigma}{tau}{upsilon}{phi}{varphi}{chi}{psi}{omega}\mydefallgreek

\NewDocumentCommand{\de}{gg}{
	\IfNoValueTF{#1}
		{\text{ d}}
		{\IfNoValueTF{#2}	{\text{ d}#1}
			{\frac{\text{d}#1}{\text{d}#2}}
	}
}

\NewDocumentCommand{\dpar}{gg}{
	\IfNoValueTF{#1}
		{\partial}
		{\IfNoValueTF{#2}	{\partial_{#1}}
			{\frac{\partial {#1}}{\partial {#2}}}
	}
}

\newcommand{\sse}{\Leftrightarrow}
\newcommand{\Rar}{\Rightarrow}
\newcommand{\rar}{\rightarrow}
\newcommand{\ol}[1]{\overline{#1}}
\newcommand{\ot}[1]{\widetilde{#1}}
\newcommand{\oc}[1]{\widehat{#1}}
\newcommand{\tc}{\mbox{ t.c. }}

\newcommand{\norma}[1]{\mid\mid #1 \mid\mid}
\newcommand{\abs}[1]{\left|{#1}\right|}
\newcommand{\scal}[2]{\langle #1 \mid #2 \rangle}
\newcommand{\floor}[1]{\lfloor #1 \rfloor}

\newcommand{\Ker}{\mbox{Ker } }
\newcommand{\Deg}{\mbox{deg }}
\newcommand{\Det}{\mbox{det }}
\newcommand{\Dim}{\mbox{dim }}
\newcommand{\End}{\mbox{End }}
\newcommand{\Rad}{\mbox{Rad }}
\newcommand{\Ann}{\mbox{Ann }}
\newcommand{\Sp}{\mbox{Sp }}
\newcommand{\Rk}{\mbox{rk }}
\newcommand{\Tr}{\mbox{tr }}
\newcommand{\GL}{\mbox{GL}}
\newcommand{\Isom}{\mbox{Isom}}
\newcommand{\Fix}{\mbox{Fix }}
\newcommand{\Giac}{\mbox{Giac }}
\newcommand{\Ort}{\mbox{O}}
\newcommand{\Aff}{\mbox{Aff }}
\newcommand{\Supp}{\mbox{Supp }}
\newcommand{\Span}{\mbox{Span }}
\newcommand{\Symm}{\mbox{Sym }}
\newcommand{\Asymm}{\mbox{Asym }}
\newcommand{\Img}{\mbox{Im }}
\newcommand{\Id}{\mbox{id}}
\newcommand{\PS}{\mbox{PS }}
\newcommand{\Mtr}{\mathfrak{m}}
\newcommand{\fucknullset}{\{0\}}

% Definiamo lo stile degli ambienti theorem che andremo ad utilizzare
\newtheoremstyle{nostrostile} % <name>
{\baselineskip} % <Space above>
{\baselineskip} % <Space below>
{} % <Body font>
{} % <Indent amount>
{\bf\scshape} % <Theorem head font>
{:} % <Punctation after theorem head>
{1em} % <Space after theorem head>
{} % <Theorem head spec>

\theoremstyle{nostrostile}

\newcommand{\lrt}[1]{\ensuremath{\left({#1}\right)}}
\newcommand{\lrq}[1]{\ensuremath{\left[{#1}\right]}}
\newcommand{\lrg}[1]{\ensuremath{\left\{{#1}\right\}}}

