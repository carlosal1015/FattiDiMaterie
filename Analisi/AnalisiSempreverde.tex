\documentclass[a4paper,NoNotes,GeneralMath]{stdmdoc}

\begin{document}
	\title{Lemmi sempreverdi di Analisi}
	
	\section*{Scambio di limite con integrale e/o successione}
	\subsection*{Derivabilità del limite di una successione}
	Sia $(f_n)_{n \in \bbN}$ una successione di funzioni derivabili sull'intervallo $I = [a, b]$. Si supponga che:
	\begin{enumerate}
		\item Le derivate $f_n'$ convergano uniformemente su $I$ ad una funzione $g$
		\item $\exists x_0 \in I$ tale che $\lim_{n \rar \infty} f_n(x_0) = l \in \bbR$
	\end{enumerate}
	Allora le funzioni $f_n$ convergono uniformemente su $I$ alla funzione $f$ che soddisfa le condizioni:
	$$ \left\{ \begin{array}{cc} f'(x) = g(x) & \forall x \in I \\ f(x_0) = l & \\ \end{array} \right. $$
	
	\subsection*{Convergenza totale}
	Sia $(V, \norm{})$ uno spazio vettoriale normato. Sono allora equivalenti le seguenti due condizioni:
	\begin{enumerate}
		\item Rispetto alla distanza $d(v, v') = \norm{v - v'}$ indotta dalla norma $\norm{\cdot}$, $(V, d)$ è completo
		\item Data comunque una successione $(v_n)_{n \in \bbN}$ di elementi di $V$ tali che $\sum_0^\infty \norm{v_n} < +\infty$, le serie $\sum_0^\infty v_n$ converge ad un elemento di $V$
	\end{enumerate}
	
	\subsection*{Convergenza normale e criterio di Weierstrass}
	Sia $(f_n)_{n \in \bbN}$ una successione di funzioni a valori reali definite su un insieme $E$, e si supponga che:
	\begin{enumerate}
		\item $\forall n \in \bbN$ esiste una costante $M_n > 0$ tale che $\abs{f_n(x)} \le M_n \quad \forall x \in E$
		\item $sum_{n = 0}^\infty M_n < +\infty$
	\end{enumerate}
	Allora la serie $\sum_0^\infty f_n$ converge uniformemente su $E$. In particolare, se $(E, d)$ è uno spazio metrico e le funzioni $f_n$ sono continue, anche la somma della serie $\sum_0^\infty f_n$ è continua.
	
	\subsection*{Derivazione sotto il segno d'integrale}
	Sia $L : [a, b] \times (c, d) \rar \bbR$ $\cC^0$ e tale che $\de{L(t, s)}{s}$ è continua in $[a, b] \times (c, d)$. Allora vale
	$$ \dde{s} \int_a^b L(t, s) \de{t} = \int_a^b \dde{s} L(t,s) \de{t} \qquad \forall s \in (c, d) $$
	
	\subsection*{Scambio limite integrale}
	Sia $(f_n)_{n \in \bbN}$ una successione di funzioni che converge puntualmente ad $f$ su un insieme $E$, e che converge uniformemente in ogni compatto contenuto in $E$ ed esiste una funzione $g$ ad integrale finito tale che $\abs{f_n(x)} \le g(x) \quad \forall x \forall n$. Allora si ha $\lim_{n \rar +\infty} \int_E f_n(x) = \int_E \lim_{n \rar +\infty} f_n(x)$
\end{document}
