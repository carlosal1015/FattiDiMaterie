\documentclass[a4paper,NoNotes,GeneralMath]{stdmdoc}

\newcommand{\intpie}{\int_{-\pi}^\pi }
\newcommand{\fracpie}{\frac{1}{\pi}}
\newcommand{\fractopie}{\frac{1}{2\pi}}
\newcommand{\CT}[1]{\cC^{#1}(\bbT)}
\newcommand{\LT}[1]{\cL^{#1}(\bbT)}
\newcommand{\cl}{\ell}
\newcommand{\Lincont}{\text{Lincont}}


\begin{document}
	\title{Analisi 3}
	
	\section*{Serie e Trasformata di Fourier}
        \subsection*{Definizioni e Remarks}
        \begin{itemize}
          \item ({\bf $f \in \CT{1}$ a tratti}) Diciamo che $f \in \CT{1}$ a tratti quando $f \in \CT{0}$ e la derivata esiste dovunque tranne che in un numero finito di punti, nei quali esiste $f'_\pm (x_0) \in \bbR$ derivate destre e sinistre.
        \end{itemize}

        \subsection*{Teoremi utili}
        \begin{itemize}
          \item ({\bf Densità in norma $\cL^1$ delle $\cC^\infty_0$ nelle $\cL^1$}) Sia $f \in \cL^1(a, b)$. Allora $\forall \varepsilon > 0, \exists f_\varepsilon \in \cC^\infty_0(a,b)$ tale che $$ \norm{f - f_\varepsilon}_{\cL^1} = \int_a^b \abs{f - f_\varepsilon} \de x  < \varepsilon $$
          \item ({\bf Lemma di Riemann-Lebesgue}) $f \in \cL^1(a, b)$. Allora $\int_a^b f(x) \sin(n x) \de x \rar 0$ quando $\abs{n} \rar \infty$ (e per il coseno si ha un enunciato analogo) [Si usi il teorema di densità precedente con qualche stima]
        \end{itemize}
        
	\subsection*{Serie di Fourier}
	\begin{itemize}
		\item ({\bf Relazioni di ortogonalità}) Valgono le seguenti formule:
			\begin{enumerate}
				\item Se $m + n > 0$ allora $\intpie \cos(mx) \cos(nx) \de{x} = \pi \delta_{mn}$
				\item Se $m + n > 0$ allora $\intpie \sin(mx) \sin(nx) \de{x} = \pi \delta_{mn}$
				\item $\forall m, n \in \bbN$ si ha $\intpie \cos(mx) \sin(nx) \de{x} = 0$
			\end{enumerate}
			Ovvero seni e coseni (interi) sono ortogonali sull'intervallo $[-\pi, \pi]$
		\item ({\bf Serie di Fourier}) Data una funzione $f$ definiamo Serie di Fourier la serie formale seguente
			$$ \frac{a_0}{2} + \sum_{k = 1}^\infty a_k \cos(kx) + \sum_{k = 1}^\infty b_k \sin(kx) = \sum_{k \in \bbZ} c_k e^{ikx} $$
			dove si ha $a_k = \fracpie \intpie \cos(kx) f(x) \de{x}$, $b_k = \fracpie \intpie \sin(kx) f(x) \de{x}$, $c_n = \fractopie \intpie f(x) e^{-ikx} \de{x}$. Ovvero gli $a_k, b_k, c_n$ sono legati dalle seguenti relazioni: \\
			$c_k = \frac{a_k - i b_k}{2}$ e $c_{-k} = \frac{a_k + i b_k}{2}$ $\forall k > 0$
	\end{itemize}
	
	\subsection*{Nucleo di Dirichlet}
	\begin{itemize}
		\item ({\bf Nucleo di Dirichlet}) $D_n(z) = \sum_{k = -n}^n e^{ikz} = \frac{\sin((n + \frac{1}{2}) z)}{\sin(\frac{z}{2})}$ [Raccogliere $e^{-inz}$ e sommare la geometrica]
		\item ({\bf Parità ed integrale}) $D_n(z)$ è pari e si ha $\int_{-\pi}^\pi D_n(z) = 2 \pi$ [Scambiare somma con integrale]		
	\end{itemize}
	
	\begin{itemize}
		\item ({\bf Riemann-Lebesgue per i coefficienti}) Se $f \in \LT{2}$ allora si ha $\hat{f_n} \rar 0$ per $\abs{n} \rar \infty$ [Considerare la norma $\cL^2$ delle code]
		\item ({\bf Più regolarità più decrescenza}) Se $f \in \CT{k}$ allora $\abs{n}^k \hat{f_n} \in \ell^2(\bbZ)$. In particolare $\hat{f_n} = o(\abs{n}^{-k})$ quando $\abs{n} \rar \infty$ [Integrare per parti la precedente]
		\item ({\bf $f \cC^1$ convergenza assoluta}) Se $f \in \CT{1}$ allora la Serie di Fourier converge assolutamente [Usare GM-QM sulla precedente]
		\item ({\bf $f \cC^1$ convergenza delle parziali}) Se $f \in \CT{1}$ allora $S_n(f, x) \rar f(x)$ per $n \rar \infty$ e $\forall x \in \bbT$ [Scrivere $S_n$ come convoluzione tra $f$ e $D_n$ e moltiplicare per uno. Poi stimare la differenza con RL]
		\item ({\bf $f \cC^1$ a tratti convergenza delle parziali}) Se $f \in CT{0}$ e la derivata esiste ovunque tranne al più in un numero finito di punti, nei quali $\exists f_{\pm}'(x_0) \in \bbR$ derivate sinistre e destre, allora si ha $S_n(f, x) \rar f(x)$ quando $n \rar \infty$ [Usare la parità di $D_n$ e spezzare in due pezzi per la stima con le derivate da un lato]
                \item ({\bf $f \cC^1$ a tratti, comportamento nei punti di salto}) Siano $x_1, \ldots, x_n$ i punti di salto. In questi definiamo $\tilde{f}(x) = \left\{ \begin{array}{cc} f(x) & \text{se } x \in [-\pi, \pi] \setminus \{ x_i \}_{i=1,\ldots,n} \\ \frac{f(x^+) + f(x^-)}{2} & \text{se } x \in \{x_i\}_{i=1,\ldots, n} \\ \end{array} \right.$, dove $f(x^\pm) = \lim_{h \rar 0^\pm} f(x + h)$, allora $S_n(f, x) \rar \tilde{f}(x)$ quando $n \rar \infty$ [Basta stimare come già precedentemente fatto solo nei punti di salto]
                \item ({\bf Dini per la convergenza delle parziali}) Sia $f \in \CT{0}$ tale che $\exists \delta$ per cui si abbia $$ \int_{-\delta}^\delta \frac{\abs{f(x + z) - f(x)}}{\abs{z}} \de z < + \infty $$ Allora si ha $S_n(f, x) \rar f(x)$ (o eventualmente a $\tilde{f}(x)$ come sopra) [Usare Riemann-Lebesgue sulla funzione $g(z)$]
	\end{itemize}

        \subsection*{Stime}
        \begin{itemize}
        \item ({\bf Stime sui coefficienti di Fourier}) Siano $f, g \in \LT{1}$. Allora si ha:
          \begin{enumerate}
          \item $\abs{\hat{f}_k - \hat{g}_k} \le \fractopie \intpie \abs{f(x) - g(x)} \de x$
          \item $\abs{S_n(f, x) - S_n(g, x)} \le \frac{2n+1}{2\pi} \intpie \abs{f(x) - g(x)} \de x$
          \end{enumerate}
        \item ({\bf serie dei coefficienti in $\cl^1(\bbZ)$}) Se $\{c_n\} \in \cl^1(\bbZ)$, allora la serie $\sum_{n \in \bbZ} c_n e^{inx}$ converge ad un certo $g(x) \in \CT{0}$ tale che $c_n = \hat{g}_n \quad \forall n$.
          \item ({\bf Sommabilità secondo Cesaro}) $\{a_n\} \in \bbR$ successione. Se $a_n \rar L$ allora $b_n = \frac{1}{n} \sum_{k=0}^{n-1} a_k \rar L$ (e $a_n$ si dice sommabile secondo Cesaro)
        \end{itemize}

        \subsection*{Nucleo di Fejer}
        \begin{itemize}
        \item ({\bf Nucleo di Fejer}) $\phi_n(z) = \frac{1}{n} \sum_{k=0}^{n-1} D_k(z) = \frac{1}{2\pi n} \left( \frac{\sin(\frac{n}{2} z)}{\sin(\frac{z}{2})} \right)^2 $
        \item ({\bf Serie di Fourier alla Cesaro}) $\sigma_n(f, x) = \frac{1}{n} \sum_{k=0}^{n-1} S_k(f, x)$
        \item ({\bf Proprietà del Nucleo di Fejer}) Si hanno le seguenti proprietà:
          \begin{enumerate}
          \item ({\bf Normalizzazione}) $\intpie \phi_n(z) \de z = 1$
          \item ({\bf Parità}) $\phi_n(z) = \phi_n(-z)$
          \item ({\bf Positività}) $\phi_n(z) \ge 0$
          \item ({\bf Rapida decrescenza}) $\forall \varepsilon, \delta > 0 \quad \exists N \tc \forall z \notin [-\delta, \delta] \quad \forall n > N \qquad \phi_n(z) < \varepsilon$. [Stime]
          \item ({\bf Velocità di decrescenza}) $\phi_n(0) = O(n)$
          \end{enumerate}
        \item ({\bf Convergenza uniforme della Serie ``di Fejer'' per funzioni continue}) $f \in \CT{0}$. Allora $\sigma_n(f, x) \rar f(x)$ uniformemente quando $n \rar \infty$, cioè $\sup_{x \in [-\pi, \pi]} \abs{\sigma_n(f, x) - f(x)} \rar 0$ [Spezzare l'integrale nella parte interna ed esterna, e stimare in norma $\cL^1$. Notare che la maggiorazione non dipende da $x$]
        \end{itemize}

        \subsection*{Successione di Dirac}
        \begin{itemize}
        \item ({\bf Successione di Dirac}) Una successione di funzioni $\{ Q_n: [-\pi, \pi] \rar \bbR \}$ si dice successione di Dirac se soddisfa le proprietà:
          \begin{enumerate}
          \item ({\bf Normalizzazione}) $\intpie Q_n(x) \de x = 1$
          \item ({\bf Parità}) $Q_n(-x) = Q_n(x)$
          \item ({\bf Positività}) $Q_n(x) \ge 0$
          \item ({\bf Quasi-nullità integrale esterna}) $\int_{\delta \le \abs{x}} Q_n(x) \de x \rar 0$ se $n \rar \infty \quad \forall \delta > 0$
          \end{enumerate}
        \item ({\bf Convergenza uniforme delle convolute}) Se $Q_n$ è una successione di Dirac, allora si ha $$ f \star Q_n (x) = \intpie f(x - z)Q_n(z) \de z \rar f(x) \quad \text{per } n \rar \infty $$
        \item ({\bf Coefficienti di Fourier della convoluzione}) $ \hat{(f \star g)}_n = \hat{f}_n \cdot \hat{g}_n $
        \item ({\bf Commutatività della convoluzione}) $f \star g = g \star f$
        \item ({\bf Disuguaglianza di convoluzione in norma $\cL^1$}) Si ha $\norm{f \star g}_{\cL^1} \le \norm{f}_{\cL^1} \cdot \norm{g}_{\cL^1}$
        \item ({\bf Disuguaglianza di convoluzione}) $f \in \cL^1, g \in \cL^p, 1 \le p \le \infty$. Allora si ha $\norm{f \star g}_{\cL^p} \le \norm{f}_{\cL^1} \norm{g}_{\cL^p}$. [Per $p = \infty$ è evidente. Per gli altri utilizare la disuguaglianza di Holder]
        \item ({\bf Coefficienti di Fourier della moltiplicazione}) $\hat{fg}_n = \hat{f}_k \star \hat{g}_k (n)$
        \item ({\bf Derivabilità della Convoluta}) $f \in \cC_0^k(\bbR^n), g \in \cL^1_{\text{loc}}(\bbR^n)$. Allora $f \star g \in \cC^k(\bbR^n)$ e si ha $D^\alpha (f \star g) (x) = (D^\alpha f \star g) (x)$
        \item ({\bf Formula per le $\sigma_n$}) $\sigma_n(f, x) = \sum_{k = -n+1}^{n-1} \left( 1 - \frac{\abs{k}}{n} \right) \hat{f}_k e^{ikx}$
        \item ({\bf ``Completezza'' dei coefficienti di fourier}) Sia $f \in \CT{0} \tc \hat{f}_k = 0 \forall k$. Allora $f \cong 0$
        \item ({\bf Convergenza assoluta in norma della successione di Dirac}) $f \in \LT{1}$. Allora $\intpie \abs{\sigma_n(f, x) - f(x)} \de x \rar 0$ per $n \rar \infty$
        \item $f \in \LT{1}$ tale che $\hat{f}_k = 0 \forall k$ allora si ha $f \cong 0$ quasi ovunque
        \item ({\bf Identità di Parseval}) Sia $f(x) = \sum_{n \in \bbZ} \hat{f}_n e^{inx}$ con $\{\hat{f}_n\} \in \cl^2(\bbZ)$. Allora vale l'uguaglianza $\norm{f}^2_{\cL^2} = \sum_{n \in \bbZ} \abs{\hat{f}_n}^2$.
          \item ({\bf Massimalità della Circonferenza}) $C$ curva chiusa, semplice e $\cC^1$. Allora l'area della zona interna a $C$ è massima quando $C$ è la circonferenza. [Scrivere ``$x(t), y(t)$'' in serie di fourier e calcolarne l'area per poi usare parseval]
        \end{itemize}

        \subsection*{Controesempi}
        \begin{itemize}
        \item ({\bf Du Bois-Raymond}) Si può costruire una funzione $f \in \CT{0}$ tale che $S_n(f, x)$ NON converge a $f$ per qualche $x$. Ovvero, non tutte le serie di fourier di funzioni continue convergono alla funzione di partenza.
        \end{itemize}


        \section*{Fisica}
        Cerchiamo in questa sezione di utilizzare i precedenti teoremi sulla serie di Fourier per risolvere problemi classici di Fisica. ATTENZIONE: nelle sezioni seguenti proviamo a sviluppare in serie di Fourier funzioni in due variabili senza dire precisamente quello che stiamo facendo (probabilmente perché non ne abbiamo idea). Solitamente si può intendere, visto che sviluppiamo nei confronti di una variabile sola, che vogliamo a parametro fissato che la funzione si possa sviluppare in serie di Fourier (e, forse, eventualmente, che le funzioni che vengano fuori abbiano qualche tipo di continuità)

        \subsection*{Equazione della corda vibrante}
        L'equazione è $u_{tt} = c^2 u_{xx}$. Vogliamo capire per quali condizioni iniziali $u(0, x) = f(x)$ la soluzione esiste ed è unica.
        \begin{enumerate}
        \item ({\bf Alcune soluzioni}) Ci accorgiamo che le funzioni $u_n(t, x) = \sin(nx) \cos(cnt)$ risolvono l'equazione, dunque ogni serie ottenuta sommandole con coefficienti risolve (per linearità dell'equazione).
        \item ({\bf Separazione delle variabili}) Proviamo a risolvere l'equazione separando le variabili. Scriviamo dunque $u(t, x) = T(t) X(x)$ e sostituendo nell'equazione principale si ha $\frac{T''(t)}{T(t)} = c^2 \frac{X''(x)}{X(x)}$. Le due funzioni sono uguali, ma in variabili diverse. Ciò significa che sono costanti. Ovvero ci siamo ricondotti allo studio del sistema $$\left\{ \begin{array}{c} X''(x) + \frac{\lambda}{c^2} X(x) = 0 \\ X(-\pi) = X(\pi) = 0 \\ \end{array} \right.$$ visto che vogliamo la corda fissata alle estremità. (D'ora in poi mettiamo $c = 1$)
        \item ({\bf $\lambda$ è positivo}) Notiamo che $\lambda$ è positivo: infatti si ha che $X''(x) X(x) + \lambda (X(x))^2 = 0 \implies \intpie X''(x)X(x) \de x + \lambda \intpie X(x)^2 \de x = 0$ e quindi, integrando per parti, si ottiene che $[X'(x)X(x)]_{-\pi}^{\pi} - \intpie (X'(x))^2 \de x + \lambda \intpie X(x)^2 \de x = 0$ e visto che $X(-\pi) = X(\pi) = 0$ (la corda sta ferma alle estremità) si ha che il primo termine è nullo e quindi $\lambda > 0$ poiché è quoziente di due numeri positivi.
        \item ({\bf ``Risoluzione'' dell'equazione}) L'equazione caratteristica è quindi $\mu^2 + \lambda = 0$, che ha come radici $\mu_{1,2} = \pm i \sqrt{\lambda}$ e si ha la soluzione $X(x) = c_1 \sin(\sqrt\lambda x) + c_2 \cos(\sqrt\lambda x)$. Imponendo le condizioni iniziali si ottiene $X(x) = c_1 \sin(kx)$ dove si ha che $\lambda = k^2$ con $k \in \bbZ$, visto che $\sin(\sqrt\lambda \pi) = 0 \implies \sqrt\lambda \pi = k \pi$ con $k \in \bbZ$ e quindi $c_2 = 0$. Ovvero l'equazione è risolvibile soltanto nel caso in cui $\lambda$ sia un quadrato.
        \item ({\bf Scriviamo la soluzione come serie di Fourier (caso generale)}) Nel caso generale (in cui non si possono separare le variabili) possiamo supporre che la soluzione sia esprimibile con {\it la sua} serie di Fourier: $u(t, x) = \sum_{n \in \bbZ} c_n(t) e^{inx} $. In questo caso ci riconduciamo dunque a risolvere il seguente sistema:
          $$\left\{ \begin{array}{c} u_{tt} = c^2 u_{xx} \\ u(-\pi, x) = u(\pi, x) \\ u_t(-\pi, x) = u_t(\pi, x) \\ u(0, x) = u_0(x) = \sum_{n \in \bbZ} c_n e^{inx} \\ u_t(0, x) = u_1(x) = \sum_{n \in \bbZ} d_n e^{inx} \\ \end{array} \right.$$
        \item ({\bf Sostituzione della serie di Fourier}) Sostituendo la serie si ottiene l'equazione $\sum_{n \in \bbZ} (c_n''(t) + c^2 n^2 c_n(t)) e^{inx} = 0$, da cui si ricava che ogni coefficiente è zero, visto che $e^{inx}$ è base (assumendo le $c_n''(t)$ con opportuna regolarità. Infatti bisogna chiedersi le $e^{inx}$ sono base {\it di cosa?}). Con questo magheggio otteniamo il sistema differenziale ordinario seguente:
          $$ \left\{ \begin{array}{c} c_n''(t) + c^2 n^2 c_n(t) = 0 \\ c_n(0) = c_n \\ c_n'(0) = d_n \end{array} \right. $$
        \item ({\bf Soluzione del sistema ordinario}) La soluzione del sistema così trovato è $c_n(t) = \frac{1}{2} (c_n + \frac{d_n}{icn}) e^{icnt} + \frac{1}{2} (c_n - \frac{d_n}{icn}) e^{-icnt}$ (per $n \neq 0$), dal profondo significato fisico, mentre per $n = 0$ si ha $c_0(t) = c_0 + d_0 t$.
        \item ({\bf Quanto è giusto ciò che abbiamo fatto?}) Per non spararle grosse dobbiamo chiederci quando converge la somma seguente (altrimenti non avremmo potuto in primo luogo scrivere la soluzione come {\it propria} serie di Fourier): $u^N(t, x) = c_0 + d_0 t + \sum_{0 \neq \abs{n} \le N} (\frac{1}{2} (c_n + \frac{d_n}{icn}) e^{icnt} + \frac{1}{2} (c_n - \frac{d_n}{icn}) e^{-icnt}) e^{inx}$. Notaci la teoria della serie di Fourier, se avessimo che sono soddisfatte $\sum_{n \in \bbZ} \abs{n}^2 \abs{c_n} < +\infty$ e $\sum_{n \in \bbZ} \abs{n} \abs{d_n} < +\infty$ allora la serie convergerebbe assolutamente ad una funzione in $\cC^2(\bbR \times \bbT)$ [Non è troppo chiaro, visto che abbiamo trattato solo funzioni in una variabile precedentemente], che sarebbe la soluzione voluta. Guarda caso, se $u_0 \in \CT{3}$ e $u_1 \in \CT{2}$ abbiamo esattamente queste convergenze, che ci permettono di scrivere la soluzione dell'equazione della corda come scritto precedentemente (le $u^N(t, x)$)
        \item ({\bf Unicità della soluzione}) Notiamo inoltre che nelle condizioni precedenti (ovvero nel caso si possa sviluppare in serie di Fourier) si ha anche l'unicità della soluzione. Supponiamo infatti che ve ne siano due, dette $u$ e $v$. Si ponga $U = u - v$. Allora $U$ è una funzione periodica sul toro che soddisfa il seguente sistema di equazioni: $$ \left\{ \begin{array}{c} U_{tt} = c^2 U_{xx} \\ U(0,x) = 0 \\ U_t(0, x) = 0 \\ U_x(0, x) = 0 \\ \end{array}{c} \right. $$
          Ovvero possiamo notare che $(U_{tt} - c^2 U_{xx}) U_t = 0$ e, integrando per parti si ottiene $\de{}{t} \left( \intpie \abs{U_t}^2 \de x + c^2 \intpie \abs{U_x}^2 \de x \right) = 0$ e quindi la somma fra parentesi è costante e si ha $U_t \equiv 0 \equiv U_x$ cioè $U$ costante. Visto che inizialmente è nulla, lo è sempre.
        \end{enumerate}

        Per i più arditi esiste anche la versione in due dimensioni. Ovvero consideriamo una membrana circolare $\Omega = B(0, 1) \subseteq \bbR^2$ ``fissata al bordo ma libera di vibrare''. Cioè cerchiamo una funzione $u(t, x, y): \bbR \times \Omega \rar \bbR$ che in ogni momento $t$ indichi l'altezza del punto $(x, y)$. Passando in coordinate polari si ottiene [in notazione fisica] $u = u(t, \rho, \theta): \bbR \times [0, 1] \times \bbT$ con $u(t, 1, \theta) = 0 \quad \forall t, \theta$ (ovvero siamo riusciti ad esprimere il vincolo in maniera decente. L'equazione da risolvere sarebbe $u_{tt} = c^2 \Delta u$. Inutile dire che si possono adottare le stesse strategie viste sopra. Ovvero in primis supporre la separazione delle variabili (e far finta di saperlo risolvere togliendosi via i casi scomodi [ci sarà da porre $p = 0$]) e poi risolvere l'equazione ottenuta per ottenerne un risultato fisico.

        \subsection*{Equazione del calore}
        Non contenti vogliamo anche risolvere l'equazione che descrive lo spostamento di calore all'interno di un segmento la cui temperatura iniziale ($t = 0$) è descritta dalla funzione $u_0(x)$. Il problema in termini matematici si formula così: vogliamo trovare le funzioni $u(t, x): [0, T] \times \bbR \rar \bbR$ che risolvono il sistema $$\left\{ \begin{array}{c} u_t = \nu u_{xx} \\ u(t, -\pi) = u(t, \pi) \\ u_x(t, -\pi) = u_x(t, \pi) \\ u(0, x) = u_0(x) \\ \end{array} \right. $$
        %% TODO: Da fare. Mi viene davvero male a guardarla. Non ci si capisce nulla

        \section*{Cabaret e Varietà}
        Un po' di cose che più o meno appartengono alla parte di fisica ma non troppo propriamente.
        \begin{itemize}
        \item ({\bf Funzione di Green}) Definiamo funzione di Green la seguente cosa: $G(x, t) = \fractopie \sum_{n \in \bbZ} e^{-\nu n^2 t} e^{inx}$ per $t > 0$ e $x \in [-\pi, \pi]$ (si può vedere che la serie converge assolutamente). Supponiamo inoltre che la condizione iniziale dell'equazione del calore sia $u_0(x) = \sum_{n \in \bbZ} u_n e^{inx}$ allora notiamo che $u(t, x) = (u_0 \star G)(x)$, dove $\star$ è la convoluzione.
        \item ({\bf Soluzione generale del calore}) Se volessimo allora risolvere la stessa equazione del calore ma con condizioni iniziali diverse, basterà saper esprimere $u_0$ in termini di funzioni decenti (tipo come somma di seni o simili) ed utilizzando l'unicità della soluzione e la relazione sopra trovata si può CALCOLARE la soluzione completa ad ogni tempo.
        \item ({\bf Disuguaglianza di Poincaré}) Sia $f \in \cC^1(a,b) \tc \int_a^b f(x) \de x = 0$. Allora $\exists c_P$ costante tale che $\int_a^b \abs{f}^2 \de x \le c_P \int_a^b \abs{f'}^2 \de x$. [Usare media integrale ed il teorema fondamentale]
        \item ({\bf Decadimento esponenziale del calore}) Se supponiamo che si abbia $\intpie u_0(x) \de x = 0$ nell'equazione del calore [dal significato fisico ignoto] allora osserviamo che il valor medio di $u(t,x)$ è nullo in ogni momento [Usare l'equazione del calore e derivare l'integrale]. Allora moltiplicando l'equazione del calore per $u$ ed integrando per parti si ottiene che $\de{}{t} \left( \norm{u(t)}^2_{\cL^2} e^{\frac{2}{c_P}t} \right) \le 0$ ovvero che si ha $\norm{u(t)}^2_{\cL^2} \le C e^{-\alpha t}$ per certi $C, \alpha$. Ovvero se il valor medio di $u$ è nullo all'istante iniziale, allora la soluzione dell'equazione del calore decade esponenzialmente.
        \item ({\bf Variante della disuguaglianza di Poincaré}) $f \in \cC^1_0(a, b) \tc f(a) = f(b) = 0$. Allora $\exists c_P$ costante tale che $\int_a^b \abs{f}^2 \de x \le c_P \int_a^b \abs{f'}^2 \de x$. [Anche qui usare il teorema fondamentale].
        \item ({\bf Disuguaglianza di Poincaré multidimensionale}) Sia $\Omega \subseteq \bbR^2, \Omega \subseteq L_a = \{ \abs{x_d} < \frac{a}{2} \}$ (ovvero limitato) e sia $f \in \cC^1_0(\Omega)$ Allora si ha: $ \int_\Omega \abs{f}^2 \le (\frac{a}{2})^2 \int_\Omega \abs{\nabla f}^2 \de x $ [Usare le derivate parziali con il teorema fondamentale e procedere ``alla fubini-tonelli'']
        \item ({\bf Divergenza di scalari (e Pd)}) Siano $u, v \in \cC^2(\Omega), \Omega \subseteq \bbR^d$ limitato e con bordo regolare (?). Allora $$ \int_\Omega \nabla u \cdot v \de x = \int_{\dpar \Omega} uv \cdot n \de S - \int_{\Omega} u \nabla \cdot v \de x $$, dove $n$ è il vettore normale alla superficie. Effettuando la sostituzione $v \mapsto \nabla v$ si ottiene un'altra cosa a cui si da un nome. A questa cosa ci si può buttare dentro l'equazione del calore in più variabili e si ottiene ancora una volta un decadimento esponenziale della norma $\cL^2$ della soluzione.
        \end{itemize}
        
        \section*{Spazi di Hilbert}
        \subsection*{Definizioni e Prime Proprietà}
        \begin{itemize}
        \item ({\bf Definizione}) Sia $X$ spazio vettoriale su $\bbR$ (oppure su $\bbC$) si dice spazio euclideo se esiste un prodotto interno $\langle \cdot \rangle : X \times X \rar \bbR$ simmetrico (rispettivamente hermitiano) che induce una norma $\norm{x} = \sqrt{\scal{x}{x}}$
        \item ({\bf Disuguaglianza di Cauchy-Schwarz}) $\forall x, y \in X \quad \abs{\scal{x}{y}} \le \norm{x} \norm{y}$
        \item ({\bf Identità del Parallelogramma}) $\forall x, y \in X$ si ha $\norm{\frac{x+y}{2}}^2 + \norm{\frac{x-y}{2}}^2 = \frac{1}{2} \left( \norm{x}^2 + \norm{y}^2 \right)$
        \item ({\bf $\cl^p(\bbZ)$ euclideo per $p = 2$}) $\cl^p(\bbZ)$ euclideo $\sse p = 2$. Infatti, per $p \neq 2$ i vettori della base canonica non soddifano l'identità del parallelogramma
        \item ({\bf Completezza di $\cl^2(\bbZ)$}) [è completo]
        \item ({\bf Cubo di Hilbert}) Definiamo cubo di Hilbert $Q = \{ \{c_k\} \in \cl^2(\bbZ) \mid \abs{c_k} \le \frac{1}{k} \}$
        \item ({\bf Compattezza del cubo di Hilbert}) Data una successione $\{ c^n \}_{n \in \bbN} \in Q$, esiste una sottosuccessione $\{ c^{n_m} \} \subseteq \{ c^n \}$ che converge ad un $c \in \cl^2(\bbZ)$
        \item ({\bf Di compattezza relativa}) Sia $\{ c^j \} \in \cl^2(\bbZ)$ tale che $c_n^j \le a_n \quad \forall j$, con $\norm{a}_{\cL^2} < +\infty$. Allora $\exists \{ c^{j^*} \} \subseteq \{ c^j \}$ che converge ad un elemento $c \in \cl^2(\bbZ)$
        \item ({\bf Base Hilbertiana}) Una successione $\{e_n\}$ di vettori indipendenti di uno spazio di Hilbert $H$ si dice base hilbertiana se $\forall x \in H, \exists \varepsilon > 0, \text{ e } \exists \sum_{n=1}^N c_n e_n$ tale che: $\norm{\sum_{n=1}^N c_n e_n - x} < \varepsilon$
        \item ({\bf Sistema completo}) $B = \{ x_\alpha \}$ si dice sistema completo per $X$ se la chiusura delle combinazioni lineari di elementi di $B$ è tutto $X$.
        \item ({\bf Base Ortogonale/Ortonormale}) $\{x_\alpha\}$ sistema competo, $\scal{x_\alpha}{x_\beta} = 0 \quad \forall \alpha \neq \beta$. Allora $\{x_\alpha\}$ si dice base ortogonale. Inoltre, se vale anche che $\scal{x_\alpha}{x_\beta} = \delta_{\alpha\beta}$, allora il sistema viene detto base ortonormale.
        \item ({\bf Separabilità}) $X$ si dice separabile se esiste un sottoinsieme denso (al più) numerabile.
        \item ({\bf Cardinalità di una base ortonormale}) Sia $X$ separabile ed $\{x_\alpha\}$ ortonormale. Allora $\{x_\alpha\}$ è al più numerabile.
        \item ({\bf Ortonormalizzazione di Gram-Schmidt}) $\{f_j\}_{j \in \bbN} \subseteq X$ linearmente indipendenti. Allora $\exists \{\phi_j\}_{j \in \bbN}$ ortonormale con $\Span(\varphi_1, \ldots, \varphi_n) = \Span(f_1, \ldots, f_n) \quad \forall n \in \bbN$. Da ciò deriva in particolare che se $X$ è separabile, allora esiste una base ortonormale numerabile
        \item ({\bf Cose che non si dicono}) D'ORA IN POI ASSUMIAMO $X$ SEPARABILE
        \item ({\bf Teorema di proiezione}) Sia $\{ \phi_n \}_{n \in \bbN} \subset X$ ortonormale (non necessariamente base), $f \in X$. Allora si ha:
          \begin{enumerate}
          \item $\min_{\alpha \in \bbR^n} \norm{f - \sum_{j=1}^n \alpha_j \phi_j} = \norm{f - \sum_{j=1}^n c_j \phi_j}$ dove i $c_j = \scal{f}{\phi_j}$ sono i coefficienti di Fourier
          \item $\min_{\alpha \in \bbR^n} \norm{f - \sum_{j=1}^n \alpha_j \phi_j}^2 = \norm{f}^2 - \sum_{j=1}^n c_j^2$
          \end{enumerate}
          [Calcolare $\norm{f - \sum_{j=1}^n \alpha_j \phi-j}^2$ con i prodotti scalari e raccogliere in modo furbo]
          Passando poi al limite in $n$ si ottiene $\sum_{j=1}^\infty c_j^2 \le \norm{f}^2$, a cui viene dato il nome di disuguaglianza di Bessel.
        \item ({\bf Sistema Ortonormale Chiuso}) Un sistema ortonormale $\{ \phi_n \}$ si dice chiuso se $\forall f \in X$ vale l'identità di Parseval, ovvero $\sum_{j=1}^\infty c_j^2 = \norm{f}^2$
        \item ({\bf Perché ci interessa?}) Teorema: $\{ \phi_n \}$ è chiuso $\sse$ ogni $f \in X$ si può scrivere tramite la sua serie di Fourier, ovvero $f = \sum_{j=1}^\infty c_j \phi_j$ [Usare i minimi trovati prima sapendo che ora abbiamo l'uguaglianza]
        \item ({\bf Hilbert equivalente ad essere chiuso}) $X$ è di Hilbert $\sse$ ogni sistema ortonormale è chiuso.
        \item ({\bf Definizione di $\cH^n$, funzioni in $\cL^2$ ``derivabili'' $n$ volte}) Consideriamo il seguente sottospazio di $\cl^2$: $h^n = \{ \{c_k\} \in \cl^2 \mid \sum_{k = 1}^\infty k^{2n}c_k^2 < \infty \}$, con la norma $\norm{c}_{h^n} = (\sum_{k=1}^\infty (c_k^2 + k^2 c_k^2 + \ldots + k^{2n} c_k^2)^\frac{1}{2}$. Definiamo inoltre, per $s \in \bbR$, $\cH^s(-\pi, \pi) = \{ f \in \cL^2(-\pi, \pi) \mid \sum_{k=1}^\infty k^{2s} c^2_k < +\infty \}$. Ora, presa $f \in \cH^s(-\pi, \pi)$ con serie di Fourier $f(x) = \sum_{k=1}^\infty c_k e^{ikx}$, possiamo associarle la sua derivata debole $s$-esima definita da $\de{{}^s}{x^s} f(x) = \sum_{k=1}^\infty k^s c_k e^{ikx}$ che risulta convergente in $\cL^2$ per definizione.
        \item ({\bf Teorema di rappresentazione di Riesz}) Sia $F \in \Lincont(\cl^p, \bbR)$ lineare e continua con $1 < p < \infty$. Allora $\exists y \in (\cl^p)^* = \cl^q$, con $\frac{1}{p} + \frac{1}{q} = 1$, tale che $F(x) = \sum_{i=1}^\infty x_i y_i \quad \forall x \in \cl^p$ (e dunque $F(x) = \scal{x}{y}$ se $p = 2$) [Scrivere $x$ in ``base'' e applicare $F$. Prendere le troncate e vedere se convergono] \\
          Inoltre si ha anche la seguente versione: $F \in \Lincont(\cL^p(\Omega), \bbR)$ con $1 < p < \infty$. Allora $\exists g \in \cL^q(\Omega)$ con $\frac{1}{p} + \frac{1}{q} = 1$, tale che $F(f) = \int_\Omega f(x) g(x) \de x \quad \forall f \in \cL^p(\Omega)$
        \item $X$ Hilbert, $F \in \Lincont(X, \bbR)$ limitato e continuo non nullo. Allora si ha $\codim \Ker F = 1$ [Prendere un vettore non nullo e decomporre in somma diretta]
        \item ({\bf Proiezione su un convesso}) $X$ spazio di Hilbert, $K \subseteq X$ convesso chiuso. Allora $\forall x \in X \quad \exists ! P(x) \in K \tc \norm{x - P(x)} = \min_{y \in K} \norm{x - y}$. [Prendere la successione degli $y_n$ che tende all'inf, mostrare che è di Cauchy con l'id del parallelogramma. L'unicità si ottiene nello stesso modo] \\
          Ciò vale in particolare se $K$ è un sottospazio lineare chiuso.
        \item $\forall w \in K$ con $K$ convesso chiuso, si ha $\scal{x - P(x), w - P(x)} \le 0$ [Dimostrare prima che data $f: \bbR^d \supseteq K \rar \bbR$ è $\cC^1$ e $\phi(t) = f(x_0 + t (x - x_0))$ con $t \in [0, 1]$ allora si ha che $\phi '(0) \ge 0$] \\
          Si può inoltre vedere che questa condizione è caratterizzante, ovvero se il punto $P(x)$ rispetta quella proprietà allora è il punto che minimizza la distanza con $x$. \\
          Come corollario se ne deriva che se $X$ è Hilbert, e $M \subseteq X$ è un sottospazio chiuso, allora si ha $X = M \oplus M^\bot$
          
        \end{itemize}
        
\end{document}
