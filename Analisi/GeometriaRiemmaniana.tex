\documentclass[a4paper,NoNotes,GeneralMath]{stdmdoc}

\newcommand{\Dde}[1]{\ensuremath{\text{D} {#1}}{\de t}}
\newcommand{\todo}{{\scshape Da capire}}
\newcommand{\lra}[1]{\ensuremath{\langle {#1} \rangle}}
\newcommand{\Exp}{\ensuremath{\text{Exp}}}

\begin{document}
\title{Geometria Riemmaniana}

\section*{Flusso di coscienza}
\begin{itemize}
\item ({\bf Metrica Riemmaniana}) Una metrica riemmaniana su $M$ è un campo tensoriale $g$ di tipo $(0, 2)$ simmetrico e definito positivo
\item ({\bf Distanza indotta}) Ogni metrica $g$ induce una distanza $d_g (x, y) = \inf \{ l_g (c) | c: [a, b] \rar M \tc c(a) = x, c(b) = y\}$, dove $l_g(c)$ è la ``lunghezza'' della curva: $l_g(c) = \int_a^b g(\dde{c}{t}, \dde{c}{t})^\frac{1}{2} \de t$.
\item ({\bf Ogni varietà ha una metrica}) Ogni varietà differenziale (ed N2) $M$ ha una metrica $g$: L'idea è che localmente (in aperti con una carta sopra) possiamo fare il pullback della metrica euclidea. Per incollarle bene, prendiamo una partizione dell'unità $\varphi_\alpha$ e definiamo $g = \sum_\alpha \varphi_\alpha g_\alpha$.
\item ({\bf Forma di volume indotta}) \todo
\item ({\bf Connessione affine}) Una connessione affine $\nabla$ su $M$ è una funzione $\nabla: \chi(M) \times \chi(M) \rar \chi(M)$, scritta come $(X, Y) \mapsto \nabla_X Y$ con le seguenti proprietà:
  \begin{enumerate}
  \item ({\bf $\cC^\infty$-lineare nella prima componente})
  \item ({\bf Finitamente additiva nella seconda componente})
  \item ({\bf ``Leibnitz'' sulle $\cC^\infty$ nella seconda componente}) $\nabla_X (fY) = f \nabla_X Y + X(f) Y$
  \end{enumerate}
\item ({\bf Derivata di campi lungo curve}) Data una varietà con una connessione $\nabla$, ed una curva $c: [0, 1] \rar M$ $\cC^1$, esiste un'unica funzione $\Dde{}{t} : \chi_{M, c} \rar \chi_{M, c}$ dove $\chi_{M, c}$ sono ``i campi vettoriali lungo $c$'' (che non so che voglia dire) con le proprietà:
  \begin{enumerate}
  \item ({\bf Finita additività}) $\Dde{(V + W)} = \Dde{V} + \Dde{W}$
  \item ({\bf ``Leibnitz'' sulle $\cC^\infty$}) $\Dde{(fV)} = \dde{f}{t} V + f \Dde{V}$
  \item ({\bf Coincide con la connessione}) Se $V$ è indotto da $Y \in \chi(M)$, allora $\Dde{V} = \nabla_{\dde{c}{t}} Y$ (dove non si capisce che campo sia $\dde{c}{t}$ e dove - se mai - sia definito)
  \end{enumerate}

  In particolare, utilizzando le proprietà si mostra l'unicità ottenendo anche la formula in coordinate $\Dde{V} = \dde{V^j}{t} \dpar{}{x^j} + \dde{x^i}{t} V^j \dpar{}{x^i} \dpar{}{x^j}$. A questo punto si definisce $\Dde{V}$ attraverso la formula appena trovata, si verifica che abbia le proprietà di cui sopra (tutte locali) e si mostra l'invarianza della forma della scrittura di $\Dde{V}$ per cambi di coordinate.
\item ({\bf Simboli di Christoffel}) Facendo il conto in coordinate di $\nabla_X Y$ usando le proprietà delle connessioni, saltano fuori dei coefficienti a cui non abbiamo ancora dato un nome e quindi $\nabla_{\dpar{}{x^i}} \dpar{}{x^j} = \Gamma_{ij}^k \dpar{}{x^k}$. In questo modo la formula diventa ``semplicissima'': $\nabla_X Y = (X^i Y^j \Gamma_{ij}^k + X(Y^k)) \dpar{}{x^k}$. I $\Gamma_{ij}^k$ si dicono simboli di Christoffel ma attenti perché non sono tensori! (Sono solo una manciata di numeri)
\item ({\bf Campi vettoriali paralleli}) Un campo vettoriale $V$ ``lungo la curva $c$'' si dice parallelo se $\Dde{V} = 0 \quad \forall t$.
\item ({\bf Campi paralleli $\simeq \bbR^n$}) Dato $t_0 \in I$ e $V_0 \in T_{c(t_0)}M$ si ha $\exists! V$ ``lungo $c$'' parallelo e tale che $V(c_0) = V_0$. In pratica i campi paralleli sono univocamente determinati dal loro valore in un punto della curva.
  Per mostrarlo basta farlo in una carta (notare che se sull'intersezione coincidono in un punto...) e quindi passiamo in coordinate ed usiamo il fatto che $\Dde{V} = 0$ sono un po' di ODE, per cui si ha esistenza ed unicità per Cauchy-Lipschitz.
\item ({\bf Connessioni compatibili}) Una connessione $\nabla$ si dice compatibile con la metrica $g$ se $\forall c$ curve $\cC^1$ e per ogni coppia $V, V'$ di campi vettoriali paralleli ``lungo $c$'' si ha $g(V, V') \equiv \text{costante}$.
\item $\nabla$ è compatibile con $g$ se e solo se $\dde{}{t} g(V, W) = \lra{\Dde{V}, W} + \lra{V, \Dde{W}}$. Per mostrarlo prendere una base ortonormale del tangente in un punto, estenderla a base ortonormale lungo $c$ usando campi paralleli e poi fare il conto in queste coordinate.
\item $\nabla$ è compatibile con $g$ se e solo se
  \begin{equation}
    \label{eq:cobordanza}
    X g(Y, Z) = g(\nabla_X Y, Z) + g(Y, \nabla_X Z)
  \end{equation}
\item ({\bf Connessioni simmetriche}) $\nabla$ si dice simmetrica se $\nabla_X Y - \nabla_Y X = [X, Y]$ (il commutatore). Scrivendo questa cose per i campi coordinati si ottiene che se $\nabla$ è simmetrica $\Gamma_{ij}^k = \Gamma_{jk}^i$ (c'è chi dice che ``i campi trasportati non routano'')
\item ({\bf Teorema di Levi-Civita}) Data una varietà riemmaniana $\exists! \nabla$ connessione simmetrica e compatibile con la metrica. Per mostrarlo usare la formula \ref{eq:cobordanza}, ciclarla e sommare a segni alterni, poi usare la simmetria e se siete fortunati trovate
  \begin{equation}
    \label{eq:levi_civita}
    2 g(Z, \nabla_Y X) = X g(Y, Z) + Y g(Z, X) - Z g(X, Y) - g([X, Z], Y) - g([Y, Z], X) - g([X, Y], Z)
  \end{equation}
  che ci dà l'unicità. Definendo la $\nabla$ sulla base di questa abbiamo anche l'esistenza.

  Dalla stessa formula si possono anche ottenere i simboli di Christoffel:
  $$ \Gamma_{ij}^m = \frac{1}{2} g^{km} \left\{ \dpar{}{x^i} g_{jk} + \dpar{}{x^j} g_{ki} - \dpar{}{x^k} g^{km} \right\} $$

  {\it D'ora in poi gli analisti intenderanno che ogni volta che compare o che viene menzionato $\nabla$, si fa riferimento a quello specialissimo del signor Civita}.
\item ({\bf Geodetiche}) Una curva $\gamma: I \rar M$ di classe $\cC^2$ è detta geodetica se soddisfa $\Dde{\dde{\gamma}{t}} = 0$ lungo $gamma$: ``La velocità è parallela''. Andando a scrivere in coordinate questa cosa si ottengono le equazioni ordinarie $\frac{\de^2 x^k}{\de t^2} + \Gamma_{ij}^k \dde{x^i}{t} \dde{x^j}{t} = 0$ per $k = 1, \ldots, n$ e quindi segue da Cauchy-Lipschitz la loro esistenza locale.
\item ({\bf Esistenza locale delle Geodetiche}) Dato $p \in M$, $\exists U$ intorno di $p$, $\delta, \varepsilon > 0$ tali che, definendo $\cU = \{ (q, v) \mid q \in U, v \in T_q M \tc g(v, v) < \varepsilon^2 \}$ si ha $\exists \gamma: (- \delta, \delta) \times \cU \rar M$ tale che $t \mapsto \gamma(t, q, v)$ è una geodetica (ed è l'unica) con $\gamma(0, q, v) = 0$ e $\dot{\gamma}(0, q, v) = v$.

  Notiamo che se riparametrizziamo una geodetica con $t \mapsto t\lambda$ (e $\lambda > 0$), essa rimane una geodetica.
\item ({\bf Mappa esponenziale}) La mappa esponenziale $\Exp : \Omega \subseteq TM \rar M$ è la mappa che manda $v \in T_p M$ in $\Exp_p(v) = \gamma_v(1)$, dove $\gamma_v$ è la geodetica che parte da $p$ ed è diretta lungo $v$: $\gamma_v(0) = p$, $\dot{\gamma}_v(0) = v$. Per il teorema precedente, questa mappa, per ogni $p \in M$ è ben definita in un intorno di $0 \in T_p M$ dentro $TM$ (che può cambiare col punto).

  Inoltre considerata $\Exp_p: T_pM \rar M$ risulta che $\de \Exp_p (0) = \Id_{T_p M}$ (chissà in che senso???), e quindi $\Exp_p$ è un omeomorfismo da un intorno di $0 \in T_pM$ ad un intorno di $p \in M$
\item ({\bf Coordinate Geodetiche}) Data una base ortonormale di $T_pM$ definiamo $\Psi(x^1, \ldots, x^n) := \Exp_p (x^i e_i)$. Allora sull'intorno spannato da $\Exp_p$ abbiamo una carta $\varphi = \Psi^{-1}$ sulla varietà. Queste coordinate sono speciali, poiché risulta che $g_{ij} = \delta_{ij} + O(\abs{x}^2)$ (O-grande su una varietà??? Forse significa $f = O(g)$ sse $\limsup_{q \rar p} \frac{\abs{f(q)}}{\abs{g(q)}}$). Da ciò segue che $\Gamma_{ij}^k (p) = 0$.
\item ({\bf Geodetiche e distanza}) Dato $p \in M$, $\exists \varepsilon < 0$ tale che le geodetiche $\Exp_p (tv)$ con $g(v, v) < \varepsilon^2$ minimizzano la distanza tra $p$ ed $\Exp_p(v)$. (Ma è falso per tempi grandi, esempio della sfera).
\item ({\bf Hopf-Rinon}) Sia $(M, g)$ una varietà Riemmaniana e $p \in M$. Allora sono equivalenti:
  \begin{itemize}
  \item $\Exp_p$ è definita su tutto $T_pM$
  \item $(M, d_g)$ è uno spazio metrico completo
  \item $\forall q \in M$ $\Exp_q$ è definita su tutto $T_qM$
  \end{itemize}
  Inoltre se valgono queste proprietà $\forall p, q \in M$ $\exists \gamma_{p, q}$ geodetica tale che $l(\gamma_{p,q}) = d_g(p, q)$, $\gamma(0) = p, \gamma(1) = q$.
\item ({\bf Curvatura}) La curvatura è una corrispondenza che associa a $X, Y \in \chi(M)$, una mappa $R(X, Y): \chi(M) \rar \chi(M)$ data da $R(X, Y) Z = \nabla_Y \nabla_X Z - \nabla_X \nabla_Y Z + \nabla_{[X, Y]} Z$.

  Notiamo che in $M = \bbR^n$ la forma è nulla. Inoltre $R$ è $\cC^\infty$-bilineare in $X, Y, Z$, ovvero si può identificare con un campo tensoriale di tipo $(1, 3)$. Vale inoltre l'identità di Bianchi $R(X, Y) Z + R(Y, Z) X + R(Z, X) Y = 0$.
\item ({\bf Il Riemann}) Dato $T \in \chi(M)$, possiamo porre $R(X, Y, Z, T) = g(R(X, Y) Z, T)$. Troviamo allora le seguenti simpatiche relazioni:
  \begin{enumerate}
  \item $R(X, Y, Z, T) + R(Y, Z, X, T) + R(Z, X, Y, T) = 0$
  \item $R(X, Y, Z, T) = - R(Y, X, Z, T)$
  \item $R(X, Y, Z, T) = - R(X, Y, T, Z)$
  \item $R(X, Y, Z, T) = R(Z, T, X, Y)$
  \end{enumerate}

  A questo punto in coordinate possiamo divertirci a definire $R(\dpar{}{x^i}, \dpar{}{x^j}) \dpar{}{x^k} = R_{ijk}^l \dpar{}{x^l}$. Ricordando la definizione dei simboli di Christoffel si ottiene $R_{ijk}^s = \Gamma_{ik}^l \Gamma_{jl}^s - \Gamma_{jk}^l \Gamma_{il}^s + \dpar{}{x^j} \Gamma_{ik}^s - \dpar{}{x^i} \Gamma_{jk}^s$.

  Potete pure definire $R_{ikjs} = R(\dpar{}{x^i}, \dpar{}{x^j}, \dpar{}{x^k}, \dpar{}{x^s}) = R_{ijk}^l g_{ls}$ e riscrivere le uguaglianze di cui sopra.
\item ({\bf Curvatura sezionale}) Dato un piano $2$-dimensionale $\sigma$ in $T_pM$, la curvatura sezionale di $\sigma$ viene definita da $K(X, Y) = \frac{R(X, Y, X, Y)}{g(X, X) g(Y, Y) - g(X, Y)^2}$, dove $X, Y$ è una base di $\sigma$. (La quantità è indipendente dalla scelta della base e coincide con la curvatura sezionale definita ``geometricamente'', come curvatura gaussiana dell'immagine di $\sigma$ attraverso la mappa esponenziale).
\item ({\bf Curvatura e curvatura sezionale}) La curvatura sezionale determina univocamente la curvatura $R$. Ovvero supponiamo di avere due funzioni $R, R': V \times V \times V \rar V$ con $\dim V \ge 2$ e tali che valgono le uguaglianze di cui sopra. Se $\forall \sigma$ $2$-piani in $V$ vale $K(\sigma) = K'(\sigma)$ allora $R = R'$.
\item ({\bf Tensore di Ricci}) Il tensore di Ricci è di tipo $(0, 2)$ definito da $R_{ik} = R_{ijks} g^{sj} = \sum_j R_{ijk}^j$. In particolare prendendo una base ortonormale di $T_pM$ $e_1, \ldots, e_n$ si ha $Ric(X, Y) := \sum_i R(X, e_i, Y, e_i)$.
\item ({\bf Curvatura scalare}) La curvatura scalare (indovinate) $R$ è definita come $R_{ij}g^{ji}$, ovvero come $\sum_i Ric(e_i, e_i)$.
\item ({\bf Derivata covariante di tensori}) Come prima avevamo il $\nabla$, ora lo vogliamo estendere ai tensori, ovvero $\nabla: \chi(M) \times \cT^r_s (M) \rar \cT^r_s (M)$ che soddisfa:
  \begin{enumerate}
  \item $\nabla_X (S \otimes T) = \nabla_X S \otimes T + S \otimes \nabla_X T$
  \item $\nabla_X (T_1 + T_2) = \nabla_X T_1 + \nabla_X T_2$
  \item $\nabla_{fX + gY} T = f \nabla_X T + g \nabla_Y T$
  \item $\nabla_X f = X(f)$
  \item $\nabla$ commuta con le contrazioni.
  \end{enumerate}
\item ({\bf Energia di una curva}) Data una curva $\gamma: [a, b] \rar M$, possiamo considerarne l'energia $E(\gamma) = \int_a^b g(\dot\gamma, \dot\gamma) \de t$. È quasi la stessa cosa della lunghezza, ma estremizzando l'energia si ottiene anche la velocità costante.
\item ({\bf Bonet-Myers}) Supponiamo che $M$ sia completa e che $\exists r > 0$ tale che $Ric(p) \ge \frac{1}{r^2} g(p) \quad \forall p \in M$. Allora $M$ è compatta e $\text{diam}_g (M) \le \pi r$. Inoltre $\pi_1(M)$ è finito.
\item ({\bf Hadamard}) $M$ completa, semplicemente connessa, $K_p (\sigma) \le 0 \quad \forall p \forall \sigma$ $2$-piano in $T_pM$. Allora $M$ è diffeomorfa ad $\bbR^n$ e $\forall p \quad \Exp_p : T_pM \rar M$ è un diffeomorfismo.
\item ({\bf Rauch-Klingenberg-Berger-Brondle-Schoen}) $M$ semplicemente connessa, curvatura sezionale positiva. Supponiamo che $\forall p$ $\frac{\inf_\sigma K_p (\sigma)}{\sup_\sigma K_p (\sigma)} > \frac{1}{4}$. Allora $M$ è diffeomorfa alla sfera standard. (Vi sono controesempi con $=\frac{1}{4}$, $\bbP^n\bbC$ Fubini-Study)
\item ({\bf Calcolo di Ricci}) Sistema di notazioni per le derivate covarianti di tensori solo per fare i calcoli. {\it Non aggiunge nulla a livello di teoria, se non ulteriori confusioni tra le notazioni}. Forse prima o poi lo aggiungo.
\item ({\bf Moving Frames}) Dato $U \subseteq M$, un moving frame è una $n$-upla di campi vettoriali in $U$ tali che $\forall p \in U \quad (X_1(p), \ldots, X_n(p))$ è una base di $T_pM$.

  Di un moving frame ne prendiamo le forme duali $\theta^i$, tali che $\theta^i(X_j) = \delta^i_j$. Allora $X_q = \theta^l(X_q) X_l$.
\item Dati $(X_i)_i$ e $(\theta^i)_i$ come sopra, $\exists!$ uno forme $\omega^i_j$ tali che
  \begin{itemize}
  \item $\omega^i_j = - \omega^j_i$
  \item $\de \theta^i = \theta^k \wedge \omega^i_k$
  \end{itemize}

  Se il moving frame è ortonormale, le $1$-forme $\omega^i_j$ sono dette forme di connessione.
  Possiamo poi definire anche le $2$-forme $\Omega^i_j$ attraverso $\Omega^i_j = \de \omega^i_j + \omega^i_k \wedge \omega^k_j$. Queste sono dette forme di curvatura del moving frame, e l'equazione attraverso cui si definiscono viene detta equazione di struttura.
\item ({\bf Christoffel Falsi}) Definiamo i simboli di Christoffel falsi attraverso $\nabla_{X_i} X_j = \widetilde\Gamma_{ij}^k X_k$. Inoltre definiamo $R(X_i, X_j) X_k = \widetilde R_{kij}^l X_l$.
\item ({\bf Formule per il moving frame}) Sia $(X_i)_i$ un moving frame ortonormale, con forme $\theta^l, \omega^i_j, \Omega^i_j$. Allora valgono
  \begin{itemize}
  \item $\de \theta^i = -\omega^i_k \wedge \theta^k$
  \item $\de \omega^i_j = - \omega^i_k \wedge \omega^k_j + \Omega^i_j$
  \end{itemize}
  dove $\omega^i_j = \widetilde\Gamma_{kj}^i \theta^k$ e $\Omega^i_j = \frac{1}{2} \widetilde R_{jkl}^i \theta^k \wedge \theta^l$.

  Osserviamo che allora vale $\nabla_{X_k} X_j = \omega^i_j (X_k) X_i$ e $R(X_k, X_l) X_j = \Omega^i_j (X_k, X_l) X_i$.
\item ({\bf Curvatura nulla}) Sia $M$ una varietà $n$-dimensionale tale che il tensore di curvatura è zero. Allora $M$ è isometrica localmente ad $\bbR^n$.
\item ({\bf $\cL(\Delta)$}) Data una distribuzione $\Delta$ $k$-dimensionale $\cL(\Delta)$ indica il sottoanello di $\Omega(M)$ generato dalle forme $\omega$ tali che $\omega(X_1, \ldots, X_l) = 0 \quad \forall X_1, \ldots, X_l \in \Delta$.
\item ({\bf Integrabilità e $\cL(\Delta)$}) $\Delta$ è integrabile se e solo se $\de (\cL(\Delta)) \subseteq \cL(\Delta)$.
\end{itemize}
\end{document}

