\documentclass[a4paper,NoNotes,GeneralMath]{stdmdoc}

\newcommand{\card}[1]{\text{card }({#1})}
\newcommand{\Set}{\text{Set}}
\newcommand{\ON}{\text{ON}}
\newcommand{\cof}{\text{cof}}

\begin{document}
	\title{ETI}
	
	Questo file, a differenza degli altri, vuole essere un luogo dove raccolgo tutti i trucchetti vari di Teoria Degli Insiemi. Ciò viene reso necessario dal fatto che al corso si definiscono solo delle cose, e gli esercizi c'entrano poco con tutto quanto.
	
	\section*{Assiomi}
	\begin{itemize}
		\item ({\bf Estensionalità}) Due classi sono uguali se hanno gli stessi elementi
		\item ({\bf Di Astrazione}) Data una proprietà ben definita $P$, esiste una classe i cui elementi sono gli oggetti che verificano $P$.
		\item ({\bf Di Comprensione}) Una sottoclasse di un insieme è un insieme
		\item ({\bf Insieme Vuoto}) La classe vuota è un insieme
		\item ({\bf Coppia}) Dati due insiemi $a, b$ la coppia $\{ a, b \}$ è un insieme
		\item ({\bf Unione}) Se $X$ è un insieme, allora $\cup X = \{ z \mid z \in y \in X \}$ è un insieme
		\item ({\bf Dell'Infinito}) $\exists X$ insieme tale che $\emptyset \in X$ e $a \in X \implies {a} \cup a \in X$
		\item ({\bf Potenza}) Se $X$ è un insieme, allora $\cP(X) = \{ Y \mid Y \subseteq X \}$ è un insieme
		\item ({\bf Di Rimpiazzamento}) Se $F: X \rar Y$ è una funzione tra classi ed il suo dominio $X$ è un insieme, allora la sua immagine $\Img F$ è un insieme.
		\item ({\bf Scelta}) Ne diamo un po' di formulazioni equivalenti:
			\begin{enumerate}
				\item Dato un insieme $X$ i cui elementi sono insiemi non vuoti a due a due disgiunti, esiste un insieme $S$ che interseca ciascuno degli elementi di $X$ in un singolo elemento.
				\item Data una famiglia $(X_i : i \in I)$ di insiemi non vuoti $X_i$, esiste una funzione $f$ che associa a ciascun $i \in I$ un elemento $f(i) \in X_i$.
				\item Data una famiglia $\cF$ di insiemi non vuoti, esiste una funzione $g$ che associa a ciascun $X \in \cF$ un elemento $g(X) \in X$. In particolare, fissato un insieme non vuoto $A$, possiamo considerare la famiglia $\cF = \cP(A) \setminus \{ \emptyset \}$ di tutti i sottoinsiemi non vuoti di $X$ ottenendo una funzione $g : \cP(A) \setminus \{ \emptyset \} \rar A$ che associa a ciascun sottoinsieme non vuoto $X \subseteq A$ un elemento $g(A) \in A$
				\item Siano $X, Y$ due insiemi e sia $R \subseteq X \times Y$ una relazione tra $X$ ed $Y$. Supponiamo che $(\forall x \in X)(\exists y \in Y) \quad R(x, y)$. Allora esiste $f: X \rar Y$ tale che $(\forall x \in X) \quad R(x, f(x))$
				\item Per ogni famiglia $(X_i : i \in I)$ non vuota di insiemi non vuoti, il prodotto cartesiano $\prod_{i \in I} X_i$ è non vuoto.
				\item Data una funzione surgettiva $f: X \rar Y$ tra due insiemi, esiste una funzione iniettiva $g: Y \rar X$ tale che $f(g(y)) = y \quad \forall y \in Y$
			\end{enumerate}
	\end{itemize}
	
	\section*{Teoremi Importanti}
	\subsection{Scrittura in base di ordinali}
	Dato un ordinale $\gamma \neq 0$ possiamo rappresentare ogni ordinale $\alpha \neq 0$ in modo unico nella forma $\alpha = \gamma^{\alpha_1} t_1 + \ldots + \gamma^{\alpha_k} t_k$ con $k \in \omega$, $t_1, \ldots, t_k < \gamma$ e $\alpha_1 > \ldots > \alpha_k$.
	
	\subsection{Ordinali fissi}
	Sia $f: \ON \rar \ON$ una funzione crescente e continua, ovvero tale che $f(\lambda) = \sup_{\alpha < \lambda} f(\alpha)$ per ogni ordinale limite $\lambda$. Allora esistono ordinali $x$ arbitrariamente grandi tali che $f(x) = x$.
	
	\subsection{Teorema di König}
	Per $i \in I$ sia $\alpha_i$ un cardinale. Definiamo la somma $\sum_{i \in I} \alpha_i$ come la cardinalità di $\cup_{i \in I} A_i$ dove gli $A_i$ sono insiemi disgiunti tali che $\card{A_i} = \alpha_i$. \\
	{\bf König}: Per ogni $i \in I$ siano $\alpha_i$ e $\beta_i$ cardinali tali che $\alpha_i < \beta_i$. Allora $\sum_{i \in I} \alpha_i < \prod_{i \in I} \beta_i$. \\
	{\it Da notare che è praticamente l'unico teorema sui cardinali che prende disuguaglianze strette e ci dà una disuguaglianza stretta. Può quindi essere molto utile nei ragionamenti per assurdo}
	
	\section*{Definizioni Vuote}
	\begin{itemize}
		\item ({\bf Rango di un insieme}) Assumendo BF definiamo il concetto di rango di un insieme per ricorsione sulla relazione ben fondata $\in$: $$ \rho(X) = \sup \{ \rho(y) + 1 \mid y \in X \} $$. Notiamo che il rango è una funzione $\rho: \Set \rar \ON$
		\item ({\bf ${}^{+}$}) Per ogni cardinale $\alpha$ esiste un cardinale $\alpha^{+}$ con la proprietà che: $\alpha^+$ è più grande di $\alpha$ e non esiste nessun cardinale tra $\alpha$ ed $\alpha^+$.
		\item ({\bf Aleph}) $\aleph_0 = \card{\bbN}$, $\aleph_{\alpha + 1} = \aleph_\alpha^{+}$, $\aleph_\lambda = \sup_{\beta < \alpha} \aleph_\beta$ se $\lambda$ è ordinale limite.
		\item ({\bf Beth}) $\beth_0 = \aleph_0$, $\beth_{\alpha + 1} = 2^{\beth_\alpha}$, $\beth_\lambda = \sup_{\alpha < \lambda} \beth_\alpha$ per $\lambda$ limite.
		\item ({\bf Funzione di Hartogs}) Dato un insieme $X$ sia $H(X)$ la classe degli ordinali $\alpha$ di cardinalità $\le \card{X}$
	\end{itemize}
	
	\section*{Cardinali, Aleph, Beth}
	\begin{itemize}
		\item ({\bf Sup di Cardinali}) Se $X$ è un insieme di ordinali iniziali (cardinali) allora $\sup X$ è un ordinale iniziale (cardinale)
		\item ({\bf Crescenza degli Aleph}) $\alpha < \beta \implies \aleph_\alpha < \aleph_\beta$
		\item ({\bf Biggezione Ordinali-Cardinali}) La funzione $\alpha \mapsto \aleph_\alpha$ è una biggezione dalla classe $\ON$ degli ordinali verso la classe dei cardinali infiniti
		\item ({\bf Operazioni tra cardinali}) Dati due cardinali infiniti $\alpha, \beta$ vale che $$ \alpha + \beta = \alpha \cdot \beta = \max \{ \alpha, \beta \} $$ dove le operazioni sono tra cardinali.
	\end{itemize}
	
	\section*{Funzione di Hartogs}
	\begin{itemize}
		\item $H(X)$ è un ordinale.
		\item $\card{H(X)} \not\le \card{X}$
	\end{itemize}
	
	\section*{Gerarchia di Von Neumann}
	Viene definita per ricorsione transfinita la seguente famiglia di (? insiemi) indicizzata da ordinali:
	\begin{itemize}
		\item $V_0 = \emptyset$
		\item $V_{\alpha + 1} = \cP(V_\alpha)$
		\item $V_\lambda = \cup_{\alpha < \lambda} V_\alpha$ per $\lambda$ ordinale limite.
	\end{itemize}
	Valgono i seguenti fatti:
	\begin{itemize}
		\item Ogni $V_\alpha$ è transitivo
		\item $\beta < \alpha \implies V_\beta \subseteq V_\alpha$
		\item $x \in V_\alpha \sse \rho(x) < \alpha$
		\item BF equivale all'affermazione che $\forall X \quad \exists \alpha \quad x \in V_\alpha$, ovvero che $V = \cup_{\alpha \in \ON} V_\alpha$ ($V$ è l'universo degli insiemi)
		\item $x \subseteq y \in V_\alpha \implies x \in V_\alpha$
		\item (Assumendo BF) Una classe $X \subseteq V$ è un insieme $\sse \exists \alpha \in \ON \tc X \in V_\alpha$
		\item $\forall \alpha$ si ha $\card{V_{\omega + \alpha}} = \beth_\alpha \ge \aleph_\alpha$
	\end{itemize}
	
	\section*{Cofinalità}
	Una funzione $f: A \rar B$ tra due insiemi ordinati si dice cofinale o illimitata se l'immagine di $f$ non ha maggioranti stretti in $B$. La cofinalità di $B$ è il minimo ordinale $\alpha$ tale che esiste una funzione cofinale $f: \alpha \rar B$
	\begin{itemize}
		\item Se $\beta$ è un ordinale successore si ha $\cof{\beta} = 1$
		\item $\beta \ge \cof{\alpha} \sse \exists f: \beta \rar \alpha$ cofinale.
		\item Per ogni ordinale $\alpha$ vale $\cof{\alpha} \le \card{\alpha} \le \alpha$
		\item $\cof{\beta} = \beta \implies \beta$ è un cardinale (ordinale iniziale)
		\item Ogni cardinale successore $\kappa^{+}$ (ovvero il minimo cardinale maggiore di $\kappa$) è tale che $\cof{\kappa^{+}} = \kappa^{+}$
		\item Vale $\cof{\kappa} = \kappa \sse $ per ogni famiglia $(A_i : i \in I)$ di insiemi $A_i$ tali che $\card{A_i} < \kappa$ e $\card{I} < \kappa$ si ha $\card{\cup_{i \in I} A_i} < \kappa$
		\item Per ogni ordinale $\alpha$ si ha $\cof{2^{\aleph_\alpha}} > \aleph_\alpha$
		\item Se un ordinale limite $\alpha$ NON è un cardinale si ha $\cof{\alpha} < \alpha$
		\item Per ogni ordinale limite $\cof{\cof{\alpha}} = \cof{\alpha}$
	\end{itemize}
	
	\section*{Aritmetica Cardinale}
	Nel seguito diamo qualche risultato sull'esponenziazione di cardinali
	\begin{itemize}
		\item $2 \le \kappa \le \lambda$ e $\lambda$ infinito $\implies \kappa^\lambda = 2^\lambda$
		\item Inoltre si ha $2^\lambda \ge \kappa \implies \kappa^\lambda = 2^\lambda$
		\item $\lambda \ge \cof{\kappa} \implies \kappa < \kappa^\lambda$
		\item Definiamo ora detto $\lambda$ cardinale e $\card{A} \ge \lambda$ l'insieme $[A]^\lambda = \{ X \subseteq A : \card{X} = \lambda \}$.
		\item $\card{A} = \kappa \ge \lambda$ implica che $[A]^\lambda$ ha cardinalità $\kappa^\lambda$
		\item $\lambda$ cardinale infinito e $\kappa_i > 0 \quad \forall i < \lambda$, allora $$ \sum_{i < \lambda} \kappa_i = \lambda \cdot \sup_{i < \lambda} \kappa_i $$ 
	\end{itemize}
	
	\section*{Trucchi per gli esercizi}
	\subsection*{Calcoli con le forme normali di Cantor}
	Diamo ora delle regole di calcolo per fare conti con prodotti di cose in forma normale di Cantor
	\begin{itemize}
		\item Se $\alpha > \beta$ si ha $\omega^\beta b + \omega^\alpha a = \omega^\alpha a$
		\item Se $0 < \alpha = \omega^{\alpha_1} c_1 + \ldots + \omega^{\alpha_k} c_k$ in CNF e $0 < \beta$ allora si ha
			$$ \alpha \omega^\beta = \omega^{\alpha_1 + \beta} $$ ed anche, per ogni $n \in \bbN$, $n \neq 0$
			$$ \alpha n = \omega^{\alpha_1} c_1 n + \omega^{\alpha_2} c_2 + \ldots + \omega^{\alpha_k} c_k $$
	\end{itemize}
	
	\subsection*{Punti fissi di funzioni}
	\begin{itemize}
		\item Data $f: \ON \rar \ON$ strettamente crescente e continua ai limiti, esistono punti fissi arbitrariamente grandi
		\item (Inserire l'enunciato per i punti fissi comuni di una certa famiglia di funzioni da un insieme a se stesso)
	\end{itemize}
	
	\section*{Assiomi Utilizzati}
	Viene di seguito riportata una tabella con i principali teoremi di Insiemi, e gli assiomi necessari per dimostrarli.
	
	
\end{document}
