\documentclass[a4paper,NoNotes,GeneralMath]{stdmdoc}
\usepackage{tikz}

\newcommand{\card}[1]{\text{card }({#1})}
\newcommand{\Set}{\text{Set}}
\newcommand{\ON}{\text{ON}}
\newcommand{\cof}[1]{\text{cof}({#1})}
\newcommand{\Fun}{\text{Fun}}

\begin{document}
	\title{ETI}
	
	Questo file, a differenza degli altri, vuole essere un luogo dove raccolgo tutti i trucchetti vari di Teoria Degli Insiemi. Ciò viene reso necessario dal fatto che al corso si definiscono solo delle cose, e gli esercizi c'entrano poco con tutto quanto.
	
	\section*{Assiomi}
	\begin{itemize}
		\item ({\bf Estensionalità}) Due classi sono uguali se hanno gli stessi elementi
		\item ({\bf Di Astrazione}) Data una proprietà ben definita $P$, esiste una classe i cui elementi sono gli oggetti che verificano $P$.
		\item ({\bf Di Comprensione}) Una sottoclasse di un insieme è un insieme
		\item ({\bf Insieme Vuoto}) La classe vuota è un insieme
		\item ({\bf Coppia}) Dati due insiemi $a, b$ la coppia $\{ a, b \}$ è un insieme
		\item ({\bf Unione}) Se $X$ è un insieme, allora $\cup X = \{ z \mid z \in y \in X \}$ è un insieme
		\item ({\bf Dell'Infinito}) $\exists X$ insieme tale che $\emptyset \in X$ e $a \in X \implies {a} \cup a \in X$
		\item ({\bf Potenza}) Se $X$ è un insieme, allora $\cP(X) = \{ Y \mid Y \subseteq X \}$ è un insieme
		\item ({\bf Di Rimpiazzamento}) Se $F: X \rar Y$ è una funzione tra classi ed il suo dominio $X$ è un insieme, allora la sua immagine $\Img F$ è un insieme.
		\item ({\bf Scelta}) Ne diamo un po' di formulazioni equivalenti:
			\begin{enumerate}
				\item Dato un insieme $X$ i cui elementi sono insiemi non vuoti a due a due disgiunti, esiste un insieme $S$ che interseca ciascuno degli elementi di $X$ in un singolo elemento.
				\item Data una famiglia $(X_i : i \in I)$ di insiemi non vuoti $X_i$, esiste una funzione $f$ che associa a ciascun $i \in I$ un elemento $f(i) \in X_i$.
				\item Data una famiglia $\cF$ di insiemi non vuoti, esiste una funzione $g$ che associa a ciascun $X \in \cF$ un elemento $g(X) \in X$. In particolare, fissato un insieme non vuoto $A$, possiamo considerare la famiglia $\cF = \cP(A) \setminus \{ \emptyset \}$ di tutti i sottoinsiemi non vuoti di $X$ ottenendo una funzione $g : \cP(A) \setminus \{ \emptyset \} \rar A$ che associa a ciascun sottoinsieme non vuoto $X \subseteq A$ un elemento $g(A) \in A$
				\item Siano $X, Y$ due insiemi e sia $R \subseteq X \times Y$ una relazione tra $X$ ed $Y$. Supponiamo che $(\forall x \in X)(\exists y \in Y) \quad R(x, y)$. Allora esiste $f: X \rar Y$ tale che $(\forall x \in X) \quad R(x, f(x))$
				\item Per ogni famiglia $(X_i : i \in I)$ non vuota di insiemi non vuoti, il prodotto cartesiano $\prod_{i \in I} X_i$ è non vuoto.
				\item Data una funzione surgettiva $f: X \rar Y$ tra due insiemi, esiste una funzione iniettiva $g: Y \rar X$ tale che $f(g(y)) = y \quad \forall y \in Y$
			\end{enumerate}
	\end{itemize}
	
	\section*{Teoremi Importanti}
	\subsection{Scrittura in base di ordinali}
	Dato un ordinale $\gamma \neq 0$ possiamo rappresentare ogni ordinale $\alpha \neq 0$ in modo unico nella forma $\alpha = \gamma^{\alpha_1} t_1 + \ldots + \gamma^{\alpha_k} t_k$ con $k \in \omega$, $t_1, \ldots, t_k < \gamma$ e $\alpha_1 > \ldots > \alpha_k$.
	
	\subsection{Ordinali fissi}
	Sia $f: \ON \rar \ON$ una funzione crescente e continua, ovvero tale che $f(\lambda) = \sup_{\alpha < \lambda} f(\alpha)$ per ogni ordinale limite $\lambda$. Allora esistono ordinali $x$ arbitrariamente grandi tali che $f(x) = x$.
	
	\subsection{Teorema di König}
	Per $i \in I$ sia $\alpha_i$ un cardinale. Definiamo la somma $\sum_{i \in I} \alpha_i$ come la cardinalità di $\cup_{i \in I} A_i$ dove gli $A_i$ sono insiemi disgiunti tali che $\card{A_i} = \alpha_i$. \\
	{\bf König}: Per ogni $i \in I$ siano $\alpha_i$ e $\beta_i$ cardinali tali che $\alpha_i < \beta_i$. Allora $\sum_{i \in I} \alpha_i < \prod_{i \in I} \beta_i$. \\
	{\it Da notare che è praticamente l'unico teorema sui cardinali che prende disuguaglianze strette e ci dà una disuguaglianza stretta. Può quindi essere molto utile nei ragionamenti per assurdo}
	
	\section*{Definizioni Vuote}
	\begin{itemize}
		\item ({\bf Rango di un insieme}) Assumendo BF definiamo il concetto di rango di un insieme per ricorsione sulla relazione ben fondata $\in$: $$ \rho(X) = \sup \{ \rho(y) + 1 \mid y \in X \} $$. Notiamo che il rango è una funzione $\rho: \Set \rar \ON$
		\item ({\bf ${}^{+}$}) Per ogni cardinale $\alpha$ esiste un cardinale $\alpha^{+}$ con la proprietà che: $\alpha^+$ è più grande di $\alpha$ e non esiste nessun cardinale tra $\alpha$ ed $\alpha^+$.
		\item ({\bf Aleph}) $\aleph_0 = \card{\bbN}$, $\aleph_{\alpha + 1} = \aleph_\alpha^{+}$, $\aleph_\lambda = \sup_{\beta < \alpha} \aleph_\beta$ se $\lambda$ è ordinale limite.
		\item ({\bf Beth}) $\beth_0 = \aleph_0$, $\beth_{\alpha + 1} = 2^{\beth_\alpha}$, $\beth_\lambda = \sup_{\alpha < \lambda} \beth_\alpha$ per $\lambda$ limite.
		\item ({\bf Funzione di Hartogs}) Dato un insieme $X$ sia $H(X)$ la classe degli ordinali $\alpha$ di cardinalità $\le \card{X}$
	\end{itemize}
	
	\section*{Cardinali, Aleph, Beth}
	\begin{itemize}
		\item ({\bf Sup di Cardinali}) Se $X$ è un insieme di ordinali iniziali (cardinali) allora $\sup X$ è un ordinale iniziale (cardinale)
		\item ({\bf Crescenza degli Aleph}) $\alpha < \beta \implies \aleph_\alpha < \aleph_\beta$
		\item ({\bf Biggezione Ordinali-Cardinali}) La funzione $\alpha \mapsto \aleph_\alpha$ è una biggezione dalla classe $\ON$ degli ordinali verso la classe dei cardinali infiniti
		\item ({\bf Operazioni tra cardinali}) Dati due cardinali infiniti $\alpha, \beta$ vale che $$ \alpha + \beta = \alpha \cdot \beta = \max \{ \alpha, \beta \} $$ dove le operazioni sono tra cardinali.
	\end{itemize}
	
	\section*{Funzione di Hartogs}
	\begin{itemize}
		\item $H(X)$ è un ordinale.
		\item $\card{H(X)} \not\le \card{X}$
	\end{itemize}
	
	\section*{Gerarchia di Von Neumann}
	Viene definita per ricorsione transfinita la seguente famiglia di (? insiemi) indicizzata da ordinali:
	\begin{itemize}
		\item $V_0 = \emptyset$
		\item $V_{\alpha + 1} = \cP(V_\alpha)$
		\item $V_\lambda = \cup_{\alpha < \lambda} V_\alpha$ per $\lambda$ ordinale limite.
	\end{itemize}
	Valgono i seguenti fatti:
	\begin{itemize}
		\item Ogni $V_\alpha$ è transitivo
		\item $\beta < \alpha \implies V_\beta \subseteq V_\alpha$
		\item $x \in V_\alpha \sse \rho(x) < \alpha$
		\item BF equivale all'affermazione che $\forall X \quad \exists \alpha \quad x \in V_\alpha$, ovvero che $V = \cup_{\alpha \in \ON} V_\alpha$ ($V$ è l'universo degli insiemi)
		\item $x \subseteq y \in V_\alpha \implies x \in V_\alpha$
		\item (Assumendo BF) Una classe $X \subseteq V$ è un insieme $\sse \exists \alpha \in \ON \tc X \in V_\alpha$
		\item $\forall \alpha$ si ha $\card{V_{\omega + \alpha}} = \beth_\alpha \ge \aleph_\alpha$
	\end{itemize}
	
	\section*{Cofinalità}
	Una funzione $f: A \rar B$ tra due insiemi ordinati si dice cofinale o illimitata se l'immagine di $f$ non ha maggioranti stretti in $B$. La cofinalità di $B$ è il minimo ordinale $\alpha$ tale che esiste una funzione cofinale $f: \alpha \rar B$
	\begin{itemize}
		\item Se $\beta$ è un ordinale successore si ha $\cof{\beta} = 1$
		\item (La somma è intesa ordinale) $\cof{\alpha + \beta} = \cof{\beta}$ (Basta accorgersi che è sufficiente mandare gli ordinali nella parte che contiene solo $\beta$ affinché siano cofinali)
		\item (Il prodotto è ordinale) Se $\beta$ è limite si ha $\cof{\alpha \cdot \beta} = \cof{\beta}$ (Come sopra, basta mandare gli ordinali in cose del tipo $(0, \gamma) \in \alpha\beta$). Se $\beta$ non fosse limite si può spezzare $\alpha \cdot \beta$ ed utilizzare la formula di sopra.
		\item (L'esponenziazione è ordinale) Se $\beta$ è limite allora si ha $\cof{\alpha^\beta} = \cof{\beta}$ (Mandando gli ordinali in cose del tipo $f_\gamma: \beta \rar \alpha$ definita da $f_\gamma(\varepsilon) = 1$ se $\varepsilon = \gamma$ oppure $ = 0$ se $\varepsilon \neq \gamma$). Se $\beta$ non fosse limite, si può spezzare $\alpha^\beta$ in cose più semplici ed utilizzare la formula di sopra.
		\item $\beta \ge \cof{\alpha} \sse \exists f: \beta \rar \alpha$ cofinale.
		\item Per ogni ordinale $\alpha$ vale $\cof{\alpha} \le \card{\alpha} \le \alpha$
		\item $\cof{\beta} = \beta \implies \beta$ è un cardinale (ordinale iniziale)
		\item Ogni cardinale successore $\kappa^{+}$ (ovvero il minimo cardinale maggiore di $\kappa$) è tale che $\cof{\kappa^{+}} = \kappa^{+}$
		\item Vale $\cof{\kappa} = \kappa \sse $ per ogni famiglia $(A_i : i \in I)$ di insiemi $A_i$ tali che $\card{A_i} < \kappa$ e $\card{I} < \kappa$ si ha $\card{\cup_{i \in I} A_i} < \kappa$
		\item Per ogni ordinale $\alpha$ si ha $\cof{2^{\aleph_\alpha}} > \aleph_\alpha$ (Esponenziazione cardinale, non ordinale)
		\item Se un ordinale limite $\alpha$ NON è un cardinale si ha $\cof{\alpha} < \alpha$
		\item Per ogni ordinale limite $\cof{\cof{\alpha}} = \cof{\alpha}$
		\item Se $\lambda$ è limite vale che $\cof{\aleph_\lambda} = \cof{\lambda}$
		\item Se $\nu$ è successore allora vale che $\cof{\aleph_\nu} = \aleph_\nu$
	\end{itemize}
	
	\section*{Aritmetica Cardinale}
	Nel seguito diamo qualche risultato sull'esponenziazione di cardinali
	\begin{itemize}
		\item $2 \le \kappa \le \lambda$ e $\lambda$ infinito $\implies \kappa^\lambda = 2^\lambda$
		\item Inoltre si ha $2^\lambda \ge \kappa \implies \kappa^\lambda = 2^\lambda$
		\item $\lambda \ge \cof{\kappa} \implies \kappa < \kappa^\lambda$
		\item Definiamo ora detto $\lambda$ cardinale e $\card{A} \ge \lambda$ l'insieme $[A]^\lambda = \{ X \subseteq A : \card{X} = \lambda \}$.
		\item $\card{A} = \kappa \ge \lambda$ implica che $[A]^\lambda$ ha cardinalità $\kappa^\lambda$
		\item $\lambda$ cardinale infinito e $\kappa_i > 0 \quad \forall i < \lambda$, allora $$ \sum_{i < \lambda} \kappa_i = \lambda \cdot \sup_{i < \lambda} \kappa_i $$ 
	\end{itemize}
	
	\section*{Trucchi per gli esercizi}
	\subsection*{Calcoli con le forme normali di Cantor}
	Diamo ora delle regole di calcolo per fare conti con prodotti di cose in forma normale di Cantor
	\begin{itemize}
		\item Se $\alpha > \beta$ si ha $\omega^\beta b + \omega^\alpha a = \omega^\alpha a$ \\
			Dim: Visto che $\alpha > \beta$ si ha $\exists \gamma \neq 0 \quad \alpha = \beta + \gamma$. Allora $\omega^\beta b + \omega^\alpha a = \omega^\beta (b + \omega^\gamma a)$ Mostrando che $b + \omega^\gamma a = \omega^\gamma a \quad \forall \gamma \neq 0$ si avrebbe la tesi. Siccome $b \in \omega$ si può definire in maniera piuttosto semplice la biggezione di ordinamenti in questione: mandiamo un elemento $n \in b$ nella $\gamma$-upla $(n, 0, 0, \ldots)$ e data una $\gamma$-upla $(U_i)_{i \in \gamma}$ la si può mandare in $(U_0 + b, U_i)$.
		\item Se $0 < \alpha = \omega^{\alpha_1} c_1 + \ldots + \omega^{\alpha_k} c_k$ in CNF e $0 < \beta$ allora si ha
			$$ \alpha \omega^\beta = \omega^{\alpha_1 + \beta} $$ ed anche, per ogni $n \in \bbN$, $n \neq 0$
			$$ \alpha n = \omega^{\alpha_1} c_1 n + \omega^{\alpha_2} c_2 + \ldots + \omega^{\alpha_k} c_k $$
	\end{itemize}

	\subsection*{Fatti da Tenere a Mente}
	\begin{itemize}
		\item $A \subseteq \bbR$ bene ordinato $\implies \card{A} \le \aleph_0$ (Infatti dato un insieme bene ordinato dentro $\bbR$ riesco a trovarne un'immersione che preserva l'ordine di $A$ in $\bbQ$ e quindi si ha la tesi) (Basta mandare $\alpha$ in un razionale tra $f(\alpha)$ e $f(\alpha + 1)$)
		\item Dato $\alpha > 0$ sono proprietà equivalenti:
			\begin{enumerate}
				\item $\forall \beta < \alpha \quad \beta + \alpha = \alpha$
				\item $\forall \beta, \gamma < \alpha \quad \beta + \gamma < \alpha$
				\item $\alpha = \omega ^ \delta$ per un qualche $\delta$
			\end{enumerate}
		\item Dato $\alpha > 0$ sono proprietà equivalenti:
			\begin{enumerate}
				\item $\forall \beta < \alpha \quad \beta \alpha = \alpha$
				\item $\forall \beta, \gamma < \alpha \quad \beta \gamma < \alpha$
				\item $\exists \delta \quad \alpha = \omega^{\omega^\delta}$
			\end{enumerate}
		\item Dato $\alpha > 0$ sono proprietà equivalenti:
			\begin{enumerate}
				\item $\forall \beta < \alpha \quad \beta^\alpha = \alpha$
				\item $\forall \beta, \gamma < \alpha \quad \beta^\gamma < \alpha$
			\end{enumerate}
		\item $\lambda$ è limite $\sse \lambda = \cup_{\gamma < \lambda} \gamma$ $\sse$ è della forma $\lambda = \omega \gamma$ per un qualche $\gamma$
		\item Se $\beta$ è un successore allora $\exists \lambda$ limite oppure $\lambda = 0$ ed $\exists k \in \omega$ tali che $\beta = \lambda + k$ (Si dimostra poiché non esistono catene discendenti infinite di ordinali, e sapendo che $\beta$ è successore si può scrivere $\beta = \alpha + 1$ e per induzione su $\alpha$)
		\item Gli ordinali $\alpha$ con la proprietà che $\forall X \subseteq \alpha$ si ha $X$ ha il tipo d'ordine di $\alpha$ oppure $\alpha \setminus X$ ha il tipo d'ordine di $\alpha$ dovrebbero essere soltanto $\alpha = 0$ e tutti gli ordinali limite. [Ancora da controllare]
		\item Se $f: \nu \rar \kappa$ è iniettiva e illimitata, allora $\kappa = \sum_{\alpha < \nu} \card{f(\alpha)}$ dove $\nu$ e $\kappa$ sono cardinali infiniti
	\end{itemize}
	
	\subsection*{Punti fissi di funzioni}
	\begin{itemize}
		\item Data $f: \ON \rar \ON$ strettamente crescente e continua ai limiti, esistono punti fissi arbitrariamente grandi
		\item Supponiamo che $\kappa$ sia [Qualche ipotesi ancora da determinare per bene] allora si ha che, data una qualunque famiglia $\cF$ [massima cardinalità da esplicitare] di funzioni $f_i: \kappa \rar \kappa$ esistono punti fissi comuni a tutte le $f_i$ arbitrariamente grandi. \\
			Diamo un'idea della dimostrazione: Sia $\lambda$ un ordinale iniziale tale che $\lambda = \card{\cF}$. Allora definiamo per ricorsione ordinale le seguenti funzioni: $\Phi_0 = f_0$, $\Phi_{\alpha + 1} = f_{\alpha + 1} \circ \Fix(\Phi_\alpha)$, $\Phi_\lambda (\delta) = \sup_{\gamma < \lambda} \Phi_{\gamma}(\delta) $ se $\lambda$ è limite, dove $\Fix$ è la funzione che enumera i punti fissi. \\
			Controllando bene quando esistono punti fissi e quando si possono unire tutti per gli ordinali limite si ottengono le ipotesi, che prima o poi scriverò. \\
			Ci chiediamo inoltre quanti sono i punti fissi della funzione... E claimiamo che siano come $\kappa$
	\end{itemize}
	
	\subsection*{Gerarchia di Von Neumann}
	\begin{itemize}
		\item $X \subseteq V_\alpha \sse X \in V_{\alpha + 1}$
		\item Sia $A \subseteq V_\lambda$, con $A$ finito e $\lambda$ limite. Allora $A \in V_\lambda$ (Infatti $X \in A \implies X \in V_\lambda \implies \exists \alpha < \lambda \quad X \in V_\alpha$. Sfruttando il fatti che gli $X$ sono in numero finito allora si ha che, chiamato $\gamma$ il massimo degli $\alpha$ così ottenuti si ha $\gamma < \lambda$ e $\forall X \in A \quad X \in V_\gamma$, ovvero $A \in V_{\gamma + 1} \subseteq V_\lambda$, tesi)
		\item Se $\alpha$ NON è limite, allora si ha $\exists A \subseteq V_{\alpha}$ finito e non vuoto tale che $V_\alpha \setminus A$ è ancora transitivo. Infatti $\alpha = \lambda + k$ con $\lambda$ limite (oppure zero) e $k \in \omega$. Allora si prenda $A = \{V_\lambda, V_{\lambda + 1}, V_{\lambda + 2}, \ldots, V_{\lambda + (k - 1)} \}$ e si verifichi la transitività
		\item $\Fun(\omega, \omega) \subseteq V_\omega$ (e anche $\omega \times \omega \subseteq V_\omega$)
	\end{itemize}
	
	\subsection*{Aleph}
	\begin{itemize}
		\item $X \subseteq \aleph_{\alpha + 1}$ è illimitato se e solo se $\card{X} = \aleph_{\alpha + 1}$
	\end{itemize}
	
	\subsection*{Cardinalità Note}
	\begin{itemize}
		\item $\card{\bbN} = \aleph_0$ (e sono la più piccola cardinalità infinita)
		\item $\card{\bbN^\bbN} = \card{\bbR} = \card{\bbR^\bbN} = \kc$ (cardinalità del continuo)
		\item $\forall \alpha, \beta \ge \omega$ ordinali vale che $\card{\alpha^\beta} = \max{\card{\alpha}, \card{\beta}}$ dove l'esponenziazione è ordinale
		\item $X = \{ A \subseteq \omega_k \mid \card{A} = \aleph_n \}$ con $n \le k \in \bbN$. Allora possiamo calcolare la cardinalità di $X$, assumendo GCH: $[\omega_k]^{\aleph_n} = \omega_k ^ {\aleph_n}$ con esponenziazione cardinale. $ \card{X} = \aleph_k ^ {\aleph_n} = \aleph_k \cdot \aleph_0^{\aleph_n} = \aleph_k \cdot 2^{\aleph_n} = \aleph_k \cdot \aleph_{n + 1} = \aleph_{\max{k, n+1}}$
		\item $X = \{ f: \omega_k \rar \omega_n \mid f \text{ è illimitata } \}$. Allora, sempre assumendo GCH si ha: Supponiamo $k > n$. Allora $\card{X} \le (\aleph_n)^{\aleph_k} = \aleph_{k + 1}$. Per la disuguaglianza opposta si può considerare la seguente funzione: per ogni $A \in \cP (\omega_k \setminus \omega_n)$ sia $f_A: \omega_k \rar \omega_n$ la funzione così definita:
			$$ f_A(\alpha) = \left\{ \begin{array}{cc} \alpha & \text{ se } \alpha \in \omega_n \\ 
				\chi_A(\alpha) & \text{ se } \alpha \in \omega_k \setminus \omega_n \\ \end{array} \right.$$
			dove $\chi_A$ è la funzione caratteristica di $A$. Allora si ha che $f_A$ è suriettiva visto che $f\mid_{\omega_n}$ è l'identità, quindi $f_A \in X$. Ponendo $\Phi(A) = f_A$ allora si ottiene una funzione iniettiva $\Phi: \cP (\omega_k \setminus \omega_n) \rar X$. Poiché $\card{\omega_k \setminus \omega_n} = \aleph_k$ segue che $\aleph_{k + 1} = \card{\cP(\omega_k \setminus \omega_n)} \le \card{X}$. Allora concludiamo che $\card{X} = \aleph_{k + 1}$
		\item $X = \{ f: \omega_k \rar \omega_n \mid f \text{ è strettamente crescente} \}$ con $k < n$. Assumiamo GCH e si ha che, visto che $X$ è sottoinsieme di tutte le funzioni, $\card{X} \le (\aleph_n)^{\aleph_k} = \aleph_{\max(n, k + 1)} = \aleph_n$. Notiamo ora che $\forall \alpha \in \omega_n$ e $\forall \beta \in \omega_k$ si ha che $\alpha + \beta \in \omega_n$ (per verificarlo basta notare ad esempio che la cardinalità dell'ordinale $\alpha + \beta$ è $\max{\card{\alpha}, \card{\beta}} < \omega_n$. \\
			Dunque per ogni $\alpha \in \omega_n$ possiamo definire la funzione $f_\alpha: \omega_k \rar \omega_n$ ponendo $f_\alpha (\beta) = \alpha + \beta$. È semplice verificare che $f_\alpha$ è strettamente crescente. Ponendo $\Phi(\alpha) = f_\alpha$ si ottiene una funzione iniettiva $\Phi: \omega_n \rar X$ ed otteniamo così anche la disuguaglianza inversa $\aleph_n \le \card{X}$
	\end{itemize}
	
	\subsection*{Aritmetica Cardinale}
	\begin{itemize}
		\item Se $\kappa$ è infinito e $\cof{\kappa} \le \lambda$ allora $\kappa^\lambda > \kappa$
		\item Per $\lambda$ infinito abbiamo $\cof{2^\lambda} > \lambda$ (dove l'esponenziazione è cardinale)
		\item Se $\kappa$ è un cardinale limite e $\lambda \ge \cof{\kappa}$ allora $\kappa^\lambda = {\left( \cup_{\mu < \kappa} \mu^\lambda \right)}^\cof{\kappa}$ dove $\mu$ scorre sui cardinali
		\item ({\bf Hausdorff}) Se $\kappa$ e $\lambda$ sono cardinali infiniti, allora $(\kappa^+)^\lambda = \kappa^\lambda \cdot \kappa^+$
		\item Siano $\kappa$ e $\lambda$ cardinali con $2 \le \kappa$ e $\lambda \ge \omega$. Allora si ha
			\begin{enumerate}
				\item Se $\kappa \le \lambda$ allora $\kappa^\lambda = 2^\lambda$
				\item Se $\kappa$ è infinito ed $\exists \mu < \kappa$ tale che $\mu^\lambda \ge \kappa$ allora $\kappa^\lambda = \mu^\lambda$
				\item Assumiamo che $\kappa$ sia infinito e $\mu^\lambda < \kappa$ per tutti i $\mu < \kappa$. Allora $\lambda < \kappa$ e:
					\begin{itemize}
						\item Se $\cof{\kappa} > \lambda$ allora $\kappa^\lambda = \kappa$
						\item Se $\cof{\kappa} \le \lambda$ allora $\kappa^\lambda = \kappa^\cof{\kappa}$
					\end{itemize}
			\end{enumerate}
		\item Se assumiamo GCH e supponiamo che $\kappa$ e $\lambda$ siano cardinali con $2 \le \kappa$ e $\lambda$ infinito allora si ha:
			\begin{enumerate}
				\item Se $\kappa \le \lambda$ allora $\kappa^\lambda = \lambda^+$
				\item Se $\cof{\kappa} \le \lambda < \kappa$ allora $\kappa^\lambda = \kappa^+$
				\item Se $\lambda < \cof{\kappa}$ allora $\kappa^\lambda = \kappa$
			\end{enumerate}
	\end{itemize}
	
	\section*{Successore o Limite?}
	Elenchiamo di seguito, a seconda se $\alpha$ e $\beta$ sono limiti o successori, che cosa sono quelli ottenuti dalle operazioni elementari
	\vskip 0.5cm \begin{tabular}{ccccc}
	$\alpha$             & S & S & L & L \\
	$\beta$              & S & L & S & L \\
	$\alpha + \beta$     & S & L & S & L \\
	$\alpha \cdot \beta$ & S & L & L & L \\
	$\alpha ^ \beta$     &   & L & L & L \\
	\end{tabular} \vskip 0.5cm
	Inoltre $\alpha^\beta$ è successore $\sse$ $\alpha$ è successore e $\beta$ è finito.
	
	\section*{Disuguaglianze stupide ma da dimostrare}
	Con gli ordinali dovrebbero valere (devo ancora verificarle) le seguenti disuguaglianze stupide
	\begin{itemize}
		\item $\forall \alpha \neq 0, \beta \ge 2 \quad \alpha \beta > \alpha + 1$
		\item $\forall \chi > \alpha, \gamma \neq 0 \quad \gamma^\chi > \alpha$
		\item $\forall \alpha \ge 2, \beta \ge 2 \quad \alpha^\beta > \alpha \beta$ (con uguaglianza solo nel caso $\alpha = \beta = 2$)
	\end{itemize}
	
	\section*{Assiomi Utilizzati}
	Viene di seguito riportata una tabella con i principali teoremi di Insiemi, e gli assiomi necessari per dimostrarli (per come li abbiamo dimostrati in classe)
	\vskip 0.5cm
	\begin{tabular}{cccc}
	Teorema                    & Scelta     & Parti      & Infinito   \\
	Cantor-Bernstein           & \crossmark & \crossmark & \checkmark \\
	\end{tabular} \vskip 0.5cm
	
\end{document}
